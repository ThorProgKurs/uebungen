\begin{aufg} Implementiere eine c-Datei zu folgender Header-Datei: 
\begin{codelisting}
\begin{lstlisting}[numbers=left,numberstyle=\tiny,frame=tlrb]
/* gibt die Länge eines Strings zurück */
int str_len(char *s); 

/* gibt 0 zurück, wenn zwei strings gleich 
 * sind und 1 sonst */
int str_cmp(char *s1, char *s2);

/* kopiert s nach d und gibt d zurück */
char *str_cpy(char *d, char *s);

/* hänge s2 and s1 an und gib s1 zurück */ 
char *str_cat(char* s1, char* s2);
\end{lstlisting}
\end{codelisting}
und teste deinen Code mit folgendem Modul:
\begin{codelisting}
\begin{lstlisting}[numbers=left,numberstyle=\tiny,frame=tlrb]
#include <stdio.h>
#include "mystrings.h"

int main() {
	char p[100] = "Pepsi ";
	char c[100] = "Coca ";
	char suffix[10] = "Cola";
	char out[100];
	str_cpy(out,p); 
	str_cat(out,suffix); 
	str_cpy(p,out);
	str_cpy(out,c);
	str_cat(out,suffix);
	str_cpy(c,out);
	if (str_cmp(p,c)) {
		printf("%s",p);
		printf(" is not ");
		printf("%s",c);
		printf("\n");
	}
	return 0;
}
\end{lstlisting}
\end{codelisting}
\end{aufg}
