\begin{aufg}
\begin{enumerate}
\item Implementiere eine C-Datei zu folgender Header-Datei:
\begin{codelisting}
\begin{lstlisting}[numbers=left,numberstyle=\tiny,frame=tlrb]
/* gibt die Laenge eines Strings zurueck */
int str_len(char *s); 

/* gibt 0 zurueck, wenn zwei strings gleich 
 * sind und 1 sonst */
int str_cmp(char *s1, char *s2);

/* kopiert s nach d und gibt d zurueck */
char *str_cpy(char *d, char *s);

/* haenge s2 and s1 an und gib s1 zurueck */ 
char *str_cat(char* s1, char* s2)

/* allokiere neuen Speicher
 * fuer eine Kopie von s */
char *str_dup(char *s);
\end{lstlisting}
\end{codelisting}
\item Implementiere nun noch folgende String-Funktionen, die man in der Praxis \emph{niemals} benutzen wuerde:
\begin{codelisting}
\begin{lstlisting}[numbers=left,numberstyle=\tiny,frame=tlrb]
/* schreibt s rueckwaerts in s
 * und gibt es zurueck */
char *str_reverse(char *s);

/* gibt 1 zurueck, wenn ein String ein Palindrom
 * ist und 0 sonst */
int str_ispalin(char *s);

/* haenge s2 an s1 an und stelle sicher, dass
 * dazu genug Speicher in s1 zur
 * Verfuegung steht */
char *str_smartcat(char *s1, char *s2)

/* verkleinert den Speicher auf den s zeigt
 * auf die Laenge von s */
char *str_compress(char *s);
\end{lstlisting}
\end{codelisting}
\end{enumerate}
Ein Palindrom ist ein String, der rückwärts gelesen der Gleiche ist, zum
Beispiel ``anna''.
\end{aufg}
