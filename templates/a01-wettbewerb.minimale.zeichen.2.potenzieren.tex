\begin{aufg}
Wettbewerb: Gegeben ist folgender Programmrumpf:
\begin{codelisting}
\begin{lstlisting}[numbers=left,numberstyle=\tiny,frame=tlrb]
#include <stdio.h>
int main() {
	int x = 2;
	/* dein Code hier */
	printf("%i\n", x);
	return 0;
}
\end{lstlisting}
\end{codelisting}
Füge an der markierten Stelle C-Code ein, sodass der Wert von $2^{\left(3^3\right)}$ ausgegeben wird. Wer in diesem Raum am wenigsten Zeichen dafür benötigt bekommt eine Dose Cola. Erlaubt sind aber nur die Zeichen 
\begin{center}
	x \quad + \quad - \quad * \quad / \quad =
\end{center}
und das Semikolon. Zeilenumbrüche und Leerzeichen können natürlich nach belieben verwendet werden, da sie vom Compiler ignoriert werden.
\end{aufg}
