\begin{aufg}
Schreibe ein Modul \verb|arrayhelpers|, das einige nützliche Funktion zum \verb|int|-Array-Handling enthält:
\begin{enumerate}
\item Array zeilenweise oder mit Kommata getrennt ausgeben
\item Array sortieren
\item Alle Felder eines Arrays mit einem Wert initialisieren
\item Array um $1$ rotieren (d.h. das hinterste Element an erste Stelle schreiben und alle anderen Elemente um eins nach hinten schieben)
\item Array um $k$ rotieren
\item Array umdrehen
\item Ein Array in einem anderen suchen und die Position zurück geben. Sollte das Array nicht im anderen enthalten sein, so soll der Rückgabewert $-1$ sein.

\emph{Beispiel:} 
\begin{codelisting}
\begin{lstlisting}[numbers=left,numberstyle=\tiny,frame=tlrb]
int A[10] = {1, 2, 3, 4, 5, 6, 7, 8, 9, 10};
int B[3] = {4, 5, 6}
int C[2] = {5, 7}
int D[2] = {9, 10}
\end{lstlisting}
\end{codelisting}
Hier gilt: \verb|B| ist an $3$-ter Stelle in \verb|A| enthalten und \verb|D| an $8$-ter. Das Array \verb|C| ist garnicht in \verb|A| enthalten, darum wird der Rückgabewert $-1$ sein.

\end{enumerate}
\end{aufg}