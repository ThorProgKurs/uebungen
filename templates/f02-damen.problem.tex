\begin{aufg}
Das $8$-Damen Problem ist wie folgt definiert: Platziere $8$ Damen auf einem gewöhnlichen Schachbrett so, dass sie sich paarweise nicht bedrohen.

Das $n$-Damen Problem ist analog definiert (auf einem $n \times n$-Schachbrett). Schreibe ein Programm, dass das $n$-Damen Problem löst (falls möglich). Es hilft bei solchen Fragestellung häufig, sich Spezialfälle aufzumalen (betrachte beispielsweise das $3$- oder $4$-Damen Problem). 
\end{aufg}