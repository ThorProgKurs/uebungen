\begin{aufg}
Brainfuck ist eine sogenannte esoterische Programmiersprache, das sind Sprachen, die meist zu wissenschaftlichen oder theoretischen Zwecken, oder einfach zum Spaß entwickelt wurden. 

Brainfuck besteht nur aus $8$ Befehlen: $>$ $<$ $+$ $-$ $,$ $.$ $[$ $]$ alle anderen Zeichen werden als Kommentar interpretiert. Diese Befehle werden, wie bei C auch, nacheinander ausgeführt. Sie operieren auf einem (potentiell unendlich langen) Band (welches aus Zellen besteht in denen jeweils ein \verb|char| steht) indem sie einen Lese-/Schreibkopf über das Band bewegen und Zeichen lesen / schreiben lassen. Das Band ist überall mit \verb|'\0'| vorinitialisiert und der Lese-/Schreibkopf startet an ``Position $0$'' des Bandes. Die Befehle haben folgendee Bedeutung:

\begin{tabular}{|c|p{10cm}|} \hline
$>$ bzw. $<$ & schiebt den Lese-/Schreibkopf eins nach rechts bzw. links \\\hline
$+$ bzw. $-$ & in- bzw. dekrementiert den Bandwert unter dem Lese-/Schreibkopf um $1$ \\\hline
$.$ & gibt den Wert unter dem Lese-/Schreibkopf aus \\\hline
$,$ & liest ein Zeichen vom Benutzer ein und schreibt es unter den Lese-/Schreibkopf \\\hline
$[$ & springt zum zugehörigen $]$-Befehl, wenn der Wert unter dem Lese-/Schreibkopf $0$ ist, sonst soll nichts passieren\\\hline
$]$ & springt zum zugehörigen $[$-Befehl, wenn der Wert unter dem Lese-/Schreibkopf verschieden von $0$ ist\\\hline
\end{tabular}

\newpage
So sieht ein ``Hallo-Welt''-Programm in Brainfuck aus:

\begin{codelisting}
\begin{lstlisting}[numbers=left,numberstyle=\tiny,frame=tlrb,mathescape=true]
++++++++++
[
   >+++++++>++++++++++>+++>+<<<<-
]
>++.
>+.
+++++++..
+++.>++.
<<+++++++++++++++.
>.+++.
------.--------.
>+.>.
\end{lstlisting}
\end{codelisting}
Deine Aufgabe ist es nun, ein Programm zu schreiben, welches Brainfuck-Programme einlesen und ausführen kann. 
\end{aufg}