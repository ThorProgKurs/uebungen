\begin{aufg}
Schreibe ein Programm, dass zu einer in einer Datei gegebenen Adjazenzliste einen gerichteten Graphen einliest. Die erste Zeile der Datei enthält (mit einem Leerzeichen getrennt) die Anzahl der Knoten und die Anzahl der Kanten im Graphen und alle folgenden Zeilen (von denen es so viele gibt wie Kanten) sind ein Pärchen (wieder mit Leerzeichen getrennt) von Zahlen, das die Knotennummer angibt für die beiden Knoten die verbunden sein sollen. Beispiel für einen Graphen mit 7 Knoten und 5 Kanten:
\begin{codelisting}
\begin{lstlisting}[numbers=left,numberstyle=\tiny,frame=tlrb]
7 5
1 2
2 3
4 2
5 4
7 5
\end{lstlisting}
\end{codelisting}
\end{aufg}