\begin{aufg}
Denke dir einen sinnvollen Namen aus für ein Modul, dass Vektorrechnung auf $\R^n$ implementiert. Vektoren sollen \verb|double|-Arrays mit der Länge $n$ sein. Implementiere die nachfolgenden Funktionen:
\begin{enumerate}
	\item eine Funktion, die genügend Speicher für einen Vektor reserviert und einen Pointer darauf zurück gibt
	\item Vektoraddition
	\item Vektorsubtraktion
	\item Produkt eines Vektors mit einer skalaren Größe
	\item Skalarprodukt zweier Vektoren
	\item Kreuzprodukt zweier Vektoren (falls existent)
	\item eine Funktion, die prüft, ob zwei Vektoren orthogonal zueinander stehen
	\item eine Funktion, die prüft, ob zwei Vektoren parallel zueinander sind
	\item eine Funktion, die einen Vektor auf der Konsole aus gibt
\end{enumerate}
Der Rückgabetyp der Funktionen soll \verb|void| sein und das letzte Argument soll ein Vektor sein in dem das Ergebnis gespeichert wird. Zur Verdeutlichung hier ein Beispiel einer Funktion, die einen Vektor aufnimmt und ihn mit dem Nullvektor initialisiert:
\begin{codelisting}
\begin{lstlisting}[numbers=left,numberstyle=\tiny,frame=tlrb]
void make0(double *a, int n) { 
	int i;
	for(i=0; i<n; i++) a[i] = 0;
}
\end{lstlisting}
\end{codelisting}
\end{aufg}