\begin{aufg}
Diese Aufgabe läuft auf die Implementierung des Merge-Sort Algorithmus hinaus.
\begin{enumerate}
\item Implementiere eine Funktion \verb|merge|, die zwei bereits sortierte (eventuell verschieden große) Arrays als Argumente erhält, diese zu einem sortieren Array kombiniert und dieses zurück liefert. 
\item Die Funktion \verb|mergesort| selbst soll ein Array in zwei (möglichst gleich große) Teilarrays zerlegen, sich für diese Teilarrays selbst aufrufen und danach die dann sortierten Teilarrays mit der \verb|merge|-Funktion kombinieren. Erhält die Funktion ein Array mit keinem oder einem Element so belässt es dieses Array wie es ist, dann ist es nämlich bereits sortiert.
\item Besorge dir die Datei \verb|daten.h|, sortiere das darin definierte Array und schreibe es sortiert in eine Datei.
\end{enumerate}

Hier als Tipp ein Vorschlag für die Signaturen der beiden Funktionen:
\begin{codelisting}
\begin{lstlisting}[numbers=left,numberstyle=\tiny,frame=tlrb]
int *merge(int *list1, int n, int *list2, int m);
void mergesort(int *list, int n);
\end{lstlisting}
\end{codelisting}

\end{aufg}