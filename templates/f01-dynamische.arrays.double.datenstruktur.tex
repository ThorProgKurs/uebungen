\begin{aufg}
Implementiere "`dynamische Arrays"'. Also Funktionen, die es leicht ermöglichen mit dynamisch großen Arrays zu arbeiten. Verwende eine Header-Datei in folgendem Stil:
\begin{codelisting}
\begin{lstlisting}[numbers=left,numberstyle=\tiny,frame=tlrb]
typedef struct {
  double *data; /* eigentliches Array */
  int length;   /* erstes nicht verwendetes Element */
  int _size;    /* Menge der allokierten Elemente */
} DBLARRAY;

/* initialisiert eine Array-Datenstruktur */
DBLARRAY *dblarray_init();

/* gibt eine Array-Datenstruktur wieder frei */
void dblarray_free(DBLARRAY *);

/* setzt den Wert an der Stelle i auf val 
 * falls nötig wird neuer Speicher allokiert und 
 * alle Elemente bis dorthin mit 0 initialisiert */
int dblarray_set(DBLARRAY *, int i, double val);

/* setze das erste nicht initialisierte Element auf 
 * val sollte es noch nicht existieren wird neuer 
 * Speicher allokiert */
int dblarray_push(DBLARRAY *, double val);

/* gib den Wert an der Stelle i zurück */
double dblarray_get(DBLARRAY *, int i);

\end{lstlisting}
\end{codelisting}
Auf der Webseite findes du eine \verb|main.c|, die obiges Modul \emph{verwendet}. Damit kannst du es sozusagen testen.
\end{aufg}