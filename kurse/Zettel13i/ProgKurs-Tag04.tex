\documentclass{uebungszettel}
\newcommand{\utitle}{Tag 4}
\begin{document}
\begin{aufg}
Lege ein Modul \verb|mymath.c| / \verb|mymath.h| an, in dem du die bisher geschriebenen Funktionen auslagerst.
\end{aufg}


Wir brauchen im folgenden eine Potenzfunktion, die zwei Fließkommazahlen als Argumente akzeptiert. Falls du diese Funktion gestern geschrieben hast, sollte sie jetzt im \verb|mymath|-Modul verfügbar sein. Diese Funktion wird aber vermutlich zu langsam sein, daher gibt es die funktion 

\begin{verbatim}
double pow(double x, double y);
\end{verbatim}

in der Systemheader \verb|<math.h>|. Im Skript findest du im Anhang eine vollständige Referenz einiger Systembibliotheken.

\begin{aufg}
Implementiere die Riemann'sche Zeta-Funktion für $s \in \mathbb R$: $$
\zeta(s) := \sum_{k=1}^\infty \frac{1}{k^s} $$
Teste die Funktion für einige Werte $s \in ] 1, 3 [$. Für Werte $s \leq 1$
gilt $\zeta(s) = \infty$.
\end{aufg}

\begin{aufg}
Erweitere das "`mymath"'-Modul noch um eine Funktion, die zu den drei Koeffizienten $a, b, c \in \R$ einer quadratischen Gleichung $$
a \cdot x^2 + b \cdot x + c = 0 $$
die Lösungen berechnet und die größere Lösung zurück gibt.
\end{aufg}
\begin{aufg}
Implementiere ein Programm, dass ein \verb|int|-Array sortiert. Die naheliegenste Möglichkeit besteht wohl darin, zuerst das kleinste Element an die erste Stelle zu tauschen, dann das kleinste unter den Verbleibenden an die zweite Stelle zu tauschen, usw.
\end{aufg}

\end{document}
