\documentclass{uebungszettel}
\usepackage{algorithm,algorithmic,tabularx}
\floatname{algorithm}{Algorithmus}
\newcommand{\SET}{\textbf{set}\ }
\newcommand{\CHOOSE}{\textbf{choose}\ }
\newcommand{\GOTO}{\textbf{goto}\ }
\renewcommand{\algorithmicrequire}{\textbf{Input:}}
\renewcommand{\algorithmicensure}{\textbf{Output:}}
\renewcommand{\listalgorithmname}{Algorithms}
\renewcommand{\algorithmiccomment}[1]{\\/* #1 */}
\newcommand{\ah}[2]{\ \\* \emph{(#1, #2)}\\}
\newcommand{\utitle}{Tag 10}
\begin{document}
\begin{aufg}
Brainfuck ist eine sogenannte esoterische Programmiersprache: Das sind Sprachen, die meist zu wissenschaftlichen oder theoretischen Zwecken, oder einfach zum Spa� entwickelt wurden. 

Brainfuck besteht nur aus $8$ Befehlen: \verb|> < + - , . [ ]| -- alle anderen Zeichen werden als Kommentar interpretiert. Diese Befehle werden, wie bei C auch, nacheinander ausgef�hrt. Sie operieren auf einem (potentiell unendlich langen) Speicherband (welches aus Zellen besteht in denen jeweils ein \verb|char| steht) indem sie einen Lese-/Schreibkopf �ber das Band bewegen und Zeichen lesen/schreiben lassen. Das Band ist �berall mit \verb|0| vorinitialisiert und der Lese-/Schreibkopf startet an ``Position \verb|0|'' des Bandes. Die Befehle haben folgendee Bedeutung:

\medskip\noindent \begin{tabularx}{\textwidth}{@{}|c|X|} \hline
\verb|>| & schiebt den Lese-/Schreibkopf um eine Position nach rechts \\\hline
\verb|<| & schiebt den Lese-/Schreibkopf um eine Position nach links \\\hline
\verb|+| & inkrementiert den Bandwert unter dem Lese-/Schreibkopf um \verb|1| \\\hline
\verb|-| & dekrementiert den Bandwert unter dem Lese-/Schreibkopf um \verb|1| \\\hline
\verb|.| & gibt den Wert unter dem Lese-/Schreibkopf aus \\\hline
\verb|,| & liest ein Zeichen vom Benutzer ein und schreibt es unter den Lese-/Schreibkopf \\\hline
\verb|[| & springt zum zugeh�rigen \verb|]|-Befehl, wenn der Wert unter dem Lese-/Schreibkopf gleich \verb|0| ist, sonst soll nichts passieren\\\hline
\verb|]| & springt zum zugeh�rigen \verb|[|-Befehl, wenn der Wert unter dem Lese-/Schreibkopf verschieden von \verb|0| ist\\\hline
\end{tabularx}

\bigskip
\noindent So sieht ein ``Hallo-Welt''-Programm in Brainfuck aus:

\medskip \begin{codelisting}
\begin{lstlisting}[numbers=left,numberstyle=\tiny,frame=tlrb,mathescape=true]
++++++++++
[ >+++++++>++++++++++>+++>+<<<<- ]
>++.
>+.
+++++++..
+++.>++.
<<+++++++++++++++.
>.+++.
------.--------.
>+.>.
\end{lstlisting}
\end{codelisting}
Deine Aufgabe ist es nun, ein Programm zu schreiben, welches Brainfuck-Programme einlesen und ausf�hren kann. 
\end{aufg}


\rofoot{SS12}
\end{document}
