\documentclass{uebungszettel}
\usepackage{algorithm,algorithmic}
\floatname{algorithm}{Algorithmus}
\newcommand{\SET}{\textbf{set}\ }
\newcommand{\CHOOSE}{\textbf{choose}\ }
\newcommand{\GOTO}{\textbf{goto}\ }
\renewcommand{\algorithmicrequire}{\textbf{Input:}}
\renewcommand{\algorithmicensure}{\textbf{Output:}}
\renewcommand{\listalgorithmname}{Algorithms}
\renewcommand{\algorithmiccomment}[1]{\\/* #1 */}
\newcommand{\ah}[2]{\ \\* \emph{(#1, #2)}\\}
\newcommand{\utitle}{Tag 4}
\begin{document}

\begin{aufg}
Erweitere die Funktion zur L�sung einer quadratischen Gleichung von gestern: Es soll m�glich sein beide L�sungen weiter zu verwenden.
\end{aufg}

\begin{aufg}
Implementiere eine Funktion, die den Inhalt zweier \verb|int|-Variablen vertauscht.
\end{aufg}

\begin{aufg}
Schreibe ein Modul \verb|arrayhelpers|, das einige n�tzliche Funktion zum \verb|int|-Array-Handling enth�lt:
\begin{enumerate}
\item Array zeilenweise oder mit Kommata getrennt ausgeben
\item Array sortieren
\item Alle Felder eines Arrays mit einem Wert initialisieren
\item Array um $1$ rotieren (d.h. das hinterste Element an erste Stelle schreiben und alle anderen Elemente um eins nach hinten schieben)
\item Array um $k$ rotieren 
\item Array umdrehen
\item Ein Array in einem anderen suchen und die Position zur�ck geben. Sollte das Array nicht im anderen enthalten sein, so soll der R�ckgabewert $-1$ sein.

\emph{Beispiel:} 
\begin{codelisting}
\begin{lstlisting}[numbers=left,numberstyle=\tiny,frame=tlrb]
int A[10] = {1, 2, 3, 4, 5, 6, 7, 8, 9, 10};
int B[3] = {4, 5, 6}
int C[2] = {5, 7}
int D[2] = {9, 10}
\end{lstlisting}
\end{codelisting}
Hier gilt: \verb|B| ist an $3$-ter Stelle in \verb|A| enthalten und \verb|D| an $8$-ter. Das Array \verb|C| ist garnicht in \verb|A| enthalten, darum wird der R�ckgabewert $-1$ sein.

\end{enumerate}
\end{aufg}

\begin{aufg} L�se das $N$-Damen-Problem f�r eine Konstante $N$: Platziere $N$ Damen auf einem Schachbrett so, dass sie sich paarweise nicht bedrohen. 
\end{aufg}

\newpage

\begin{aufg} Wir wolle ein Array mit $n$ Eintr�gen als Permutation interpretieren, wenn jede Zahl von $0$ bis $n-1$ darin vor kommt.
\begin{enumerate}
\item Schreibe eine Funktion, die pr�ft, ob ein Array eine Permutation ist.
\item[b*)] Schreibe eine Funktion, die ein Array und ein Permutationsarray entgegennimmt und das Array entsprechend permutiert.
\end{enumerate}
\end{aufg}

\rofoot{SS12}
\end{document}
