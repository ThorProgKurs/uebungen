\documentclass{uebungszettel}
\usepackage{algorithm,algorithmic}
\floatname{algorithm}{Algorithmus}
\newcommand{\SET}{\textbf{set}\ }
\newcommand{\CHOOSE}{\textbf{choose}\ }
\newcommand{\GOTO}{\textbf{goto}\ }
\renewcommand{\algorithmicrequire}{\textbf{Input:}}
\renewcommand{\algorithmicensure}{\textbf{Output:}}
\renewcommand{\listalgorithmname}{Algorithms}
\renewcommand{\algorithmiccomment}[1]{\\/* #1 */}
\newcommand{\ah}[2]{\ \\* \emph{(#1, #2)}\\}
\newcommand{\utitle}{Tag 9}
\begin{document}

\begin{aufg} Spiele ein wenig mit OpenMP herum:
\begin{itemize}
\item Verstehe die Beispiele aus dem Skript und aus der OpenMP-Dokumentation; siehe die Webseite f�r die entsprechenden Links.
\item Forke und gib ein paar Zeilen auf der Kommandozeile aus. Pr�fe, ob bei mehreren Threads ein Durcheinander entsteht und repariere das Problem mit Hilfe von Locks und/oder kritischen Regionen.
\item �berlege bei vergangenen Aufgaben, ob sich Schleifen parallelisieren lassen und tue es, falls m�glich.
\end{itemize}
\end{aufg}

\begin{aufg} Parallelisiere deine Matrix-Multiplikation. Wenn du bereit bist, deinem Verstand endg�ltig lebwohl zu sagen, implementiere den Strassen-Algorithmus zur Matrix-Multiplikation mit Parallelisierung.
\end{aufg}


\rofoot{SS12}
\end{document}
