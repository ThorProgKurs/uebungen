\documentclass{uebungszettel}
\usepackage{algorithm,algorithmic}
\floatname{algorithm}{Algorithmus}
\newcommand{\SET}{\textbf{set}\ }
\newcommand{\CHOOSE}{\textbf{choose}\ }
\newcommand{\GOTO}{\textbf{goto}\ }
\renewcommand{\algorithmicrequire}{\textbf{Input:}}
\renewcommand{\algorithmicensure}{\textbf{Output:}}
\renewcommand{\listalgorithmname}{Algorithms}
\renewcommand{\algorithmiccomment}[1]{\\/* #1 */}
\newcommand{\ah}[2]{\ \\* \emph{(#1, #2)}\\}
\newcommand{\utitle}{Tag 5}
\begin{document}

\begin{aufg} Implementiere eine c-Datei zu folgender Header-Datei: 
\begin{codelisting}
\begin{lstlisting}[numbers=left,numberstyle=\tiny,frame=tlrb]
/* gibt die L�nge eines Strings zur�ck */
int str_len(char *s); 

/* gibt 0 zur�ck, wenn zwei strings gleich 
 * sind und 1 sonst */
int str_cmp(char *s1, char *s2);

/* kopiert s nach d und gibt d zur�ck */
char *str_cpy(char *d, char *s);

/* h�nge s2 and s1 an und gib s1 zur�ck */ 
char *str_cat(char* s1, char* s2)
\end{lstlisting}
\end{codelisting}
und teste deinen code mit folgendem modul:
\begin{codelisting}
\begin{lstlisting}[numbers=left,numberstyle=\tiny,frame=tlrb]
#include <stdio.h>
#include "mystrings.h"

int main() {
	char p[100] = "Pepsi ";
	char c[100] = "Coca ";
	char suffix[10] = "Cola";
	char out[100];
	str_cpy(out,p); 
	str_cat(out,suffix); 
	str_cpy(p,out);
	str_cpy(out,c);
	str_cat(out,suffix);
	str_cpy(c,out);
	if (str_cmp(p,c)) {
		printf("%s",p);
		printf(" is not ");
		printf("%s",c);
		printf("\n");
	}
	return 0;
}
\end{lstlisting}
\end{codelisting}
\end{aufg}



\begin{aufg} Implementiere einige Funktionen um mit quadratischen Matrizen umzugehen:
\begin{enumerate}
\item Eine Funktion, die Speicher f�r eine quadratische Matrix allokiert, eine um ihn freizugeben, eine um sie auszugeben und eine um sie zur Einheitsmatrix zu initialisieren (das ist die Matrix mit $1$en auf der Hauptdiagonale und $0$en sonst):
\begin{codelisting}
\begin{lstlisting}[numbers=left,numberstyle=\tiny,frame=tlrb]
double **matrix_alloc(int n);
void     matrix_free(double **A, int n);
void     matrix_print(double **A, int n);
double **matrix_id(double **A, int n);
\end{lstlisting}
\end{codelisting}
\item Eine Funktion um eine Matrix zu transponieren (d.h. an der Hauptdiagonale ``zu spiegeln'')
\item Eine Funktion, die zwei solche Matrizen miteinander multipliziert und eine neue Matrix zur�ck gibt. F�r zwei $n \times n$-Matrizen $A = (a_{ij})$ und $B = (b_{ij})$ ist $A \cdot B = C = (c_{ij})$ durch $c_{ij} = \sum_{k=1}^n a_{ik} b_{kj}$ definiert.
\end{enumerate}
\end{aufg}

\begin{aufg}
Diese Aufgabe l�uft auf die Implementierung des Merge-Sort Algorithmus hinaus.
\begin{enumerate}
\item Implementiere eine Funktion \verb|merge|, die zwei bereits sortierte (eventuell verschieden gro�e) Arrays als Argumente erh�lt, diese zu einem sortieren Array kombiniert und dieses zur�ck liefert. 
\item Die Funktion \verb|mergesort| selbst soll ein Array in zwei (m�glichst gleich gro�e) Teilarrays zerlegen, sich f�r diese Teilarrays selbst aufrufen und danach die dann sortierten Teilarrays mit der \verb|merge|-Funktion kombinieren. Erh�lt die Funktion ein Array mit keinem oder einem Element so bel�sst es dieses Array wie es ist, dann ist es n�mlich bereits sortiert.
\item Besorge dir die Datei \verb|daten.h|, sortiere das darin definierte Array und schreibe es sortiert in eine Datei.
\end{enumerate}

Hier als Tipp ein Vorschlag f�r die Signaturen der beiden Funktionen:
\begin{codelisting}
\begin{lstlisting}[numbers=left,numberstyle=\tiny,frame=tlrb]
int *merge(int *list1, int n, int *list2, int m);
void mergesort(int *list, int n);
\end{lstlisting}
\end{codelisting}

\end{aufg}

\rofoot{SS12}
\end{document}
