\documentclass{uebungszettel}
\usepackage{algorithm,algorithmic}
\floatname{algorithm}{Algorithmus}
\newcommand{\SET}{\textbf{set}\ }
\newcommand{\CHOOSE}{\textbf{choose}\ }
\newcommand{\GOTO}{\textbf{goto}\ }
\renewcommand{\algorithmicrequire}{\textbf{Input:}}
\renewcommand{\algorithmicensure}{\textbf{Output:}}
\renewcommand{\listalgorithmname}{Algorithms}
\renewcommand{\algorithmiccomment}[1]{\\/* #1 */}
\newcommand{\ah}[2]{\ \\* \emph{(#1, #2)}\\}
\newcommand{\utitle}{Tag 7}
\begin{document}


\begin{aufg} Die coole Labyrinth-Aufgabe vom letzten Mal machen. Aber nicht das mit den Matrizen, nicht umbedingt jedenfalls, nein das muss nicht sein. Kann aber.
\end{aufg}

\begin{aufg}~
\begin{enumerate}
\item
Implementiere die Addition, Multiplikation, Potenzen und Division komplexer Zahlen. Verwende dazu folgende Header-Datei:
\begin{codelisting}
\begin{lstlisting}[numbers=left,numberstyle=\tiny,frame=tlrb]
#ifndef _COMPLEX__H
#define _COMPLEX__H
typedef struct {
	double real;
	double imag;
} COMPLEX;

COMPLEX cplx_add(COMPLEX a, COMPLEX b);
COMPLEX cplx_mul(COMPLEX a, COMPLEX b);
COMPLEX cplx_pot(COMPLEX a, unsigned long n);
COMPLEX cplx_div(COMPLEX a, COMPLEX b);
#endif
\end{lstlisting}
\end{codelisting}
\item Implementiere die Addition, Multiplikation und Division sowie das K�rzen rationaler Zahlen. Schreibe dazu erst die Header-Datei.
\end{enumerate}
\end{aufg}

\newpage
\begin{aufg}
Implementiere "`dynamische Arrays"'. Also Funktionen, die es leicht erm�glichen mit dynamisch gro�en Arrays zu arbeiten. Verwende eine Header-Datei in folgendem Stil:
\begin{codelisting}
\begin{lstlisting}[numbers=left,numberstyle=\tiny,frame=tlrb]
#ifndef _DBLARRAY__H
#define _DBLARRAY__H

typedef struct {
  double *data; /* eigentliches Array */
  int length;   /* erstes nicht verwendetes Element */
  int _size;    /* Menge der allokierten Elemente */
} DBLARRAY;

/* initialisiert eine Array-Datenstruktur */
DBLARRAY *dblarray_init();

/* gibt eine Array-Datenstruktur wieder frei */
void dblarray_free(DBLARRAY *);

/* setzt den Wert an der Stelle i auf val 
 * falls n�tig wird neuer Speicher allokiert und 
 * alle Elemente bis dorthin mit 0 initialisiert */
int dblarray_set(DBLARRAY *, int i, double val);

/* setze das erste nicht initialisierte Element auf 
 * val sollte es noch nicht existieren wird neuer 
 * Speicher allokiert */
int dblarray_push(DBLARRAY *, double val);

/* gib den Wert an der Stelle i zur�ck */
double dblarray_get(DBLARRAY *, int i);

#endif
\end{lstlisting}
\end{codelisting}
Teste deine pr�chtigen Arrays ausgiebig!
\end{aufg}

\rofoot{SS12}
\end{document}
