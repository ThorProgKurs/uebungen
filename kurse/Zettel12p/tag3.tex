\documentclass{uebungszettel}
\usepackage{algorithm,algorithmic}
\floatname{algorithm}{Algorithmus}
\newcommand{\SET}{\textbf{set}\ }
\newcommand{\CHOOSE}{\textbf{choose}\ }
\newcommand{\GOTO}{\textbf{goto}\ }
\renewcommand{\algorithmicrequire}{\textbf{Input:}}
\renewcommand{\algorithmicensure}{\textbf{Output:}}
\renewcommand{\listalgorithmname}{Algorithms}
\renewcommand{\algorithmiccomment}[1]{\\/* #1 */}
\newcommand{\ah}[2]{\ \\* \emph{(#1, #2)}\\}
\newcommand{\utitle}{Tag 3}
\begin{document}
\begin{aufg}
\begin{enumerate}
\item Implementiere f�r $x \in \R$ und $n \in \N$ eine Potenzfunktion $x^n=$ \verb|power(x, n)| mit der Double-and-Add-Methode: \[
	power(x, n) = \left\{ \begin{array}{ll}
	1 & \text{wenn } n = 0 \\
	x \cdot power\left(x^2, \frac{n-1}{2}\right) & \text{wenn } n \text{ ungerade} \\
	power\left(x^2, \frac{n}{2}\right) & \text{wenn } n \text{ gerade} \\
	\end{array}
	\right.
\]
zuerst mal rekursiv. 
\item Implementiere eine Potenzfunktion \verb|naiv_power(x, n)|, indem du eine Schleife von $1$ bis $n$ laufen l�sst und bei jedem Durchlauf eine mit $1$ Initialisierte Variable mit $x$ multipliziert. Berechne $0,9999999999^{2000000000}$ einmal mit \verb|power(x, n)| von oben und einmal mit \verb|naiv_power(x, n)| (es sollte ca. $0,818731$ raus kommen). 
\item* Implementiere die Double-and-Add-Methode iterativ, also ohne rekursiven Aufruf.
\item* Frage einen Tutor wie man Zeit messen kann und vergleiche die Laufzeiten der 3 Funktionen.
\end{enumerate}
\end{aufg}
\begin{aufg}
Diese Aufgabe wird auf eine \verb|power(x, y)|-Funktion f�hren, die f�r beliebige $x \in \R^+$ und $y \in \R$ den Wert von $x^y$ berechnet. Du kannst mit dieser Funktion dein "`mymath"'-Modul um eine weitere Funktion erweitern.
\begin{itemize}
\item Implementiere die Exponential-Funktion \verb|expo(x)|, die $e^x$ mithilfe folgender Reihendarstellung: \[
e^x = \sum_{k=0}^\infty \frac{x^k}{k!} \]
\item Implementiere eine Logarithmus-Funktion \verb|logarithm(x)|, die $\ln(x)$ mithilfe folgender Reihedarstellung berechnet: \[
\ln(x) = 2\cdot \sum_{k = 0}^\infty \left(\frac{x-1}{x+1}\right)^{2k + 1} \frac{1}{2k + 1} \]
\item Verwende die Formel \[
x^y = e^{y \cdot \ln(x)} \] um \verb|power(x, y)| zu bestimmen.
\end{itemize}
\end{aufg}

\begin{aufg} Lege ein Modul \verb|mymath.c| / \verb|mymath.h| an, in dem du die bisher geschriebenen Funktionen auslagerst.
\end{aufg}
\begin{aufg} Wir brauchen im folgenden eine Potenzfunktion, die zwei Flie�kommazahlen als Argumente akzeptiert. Falls du diese Funktion gestern geschrieben hast, sollte sie jetzt im \verb|mymath|-Modul verf�gbar sein. Andernfalls gibt es die funktion 

\begin{verbatim}
double pow(double x, double y);
\end{verbatim}

in der Systemheader \verb|<math.h>|. Im Skript findest du im Anhang eine vollst�ndige Referenz einiger Systembibliotheken.


Implementiere die Riemann'sche Zeta-Funktion f�r $s \in \N$: $$
\zeta(s) := \sum_{k=1}^\infty \frac{1}{k^s} $$
\emph{Tipp:} Vielleicht kannst du die zuletzt implementierte Funktion \verb|power(x, n)| daf�r verwenden.
\end{aufg}
\begin{aufg}
Erweitere das "`mymath"'-Modul noch um eine Funktion, die zu den drei Koeffizienten $a, b, c \in \R$ einer quadratischen Gleichung $$
a \cdot x^2 + b \cdot x + c = 0 $$
die L�sungen berechnet und die gr��ere L�sung zur�ck gibt.
\end{aufg}
\rofoot{SS12}
\end{document}