\documentclass{uebungszettel}
\usepackage{algorithm,algorithmic}
\floatname{algorithm}{Algorithmus}
\newcommand{\SET}{\textbf{set}\ }
\newcommand{\CHOOSE}{\textbf{choose}\ }
\newcommand{\GOTO}{\textbf{goto}\ }
\renewcommand{\algorithmicrequire}{\textbf{Input:}}
\renewcommand{\algorithmicensure}{\textbf{Output:}}
\renewcommand{\listalgorithmname}{Algorithms}
\renewcommand{\algorithmiccomment}[1]{\\/* #1 */}
\newcommand{\ah}[2]{\ \\* \emph{(#1, #2)}\\}
\newcommand{\utitle}{Tag 6}
\begin{document}


\begin{aufg}
Gegeben sei eine Datei, in der ausschlie�lich Zahlen stehen. In der ersten Zeile stehe eine nat�rliche Zahl, die angibt wie viele Zahlen noch folgen. 
\begin{enumerate}
  \item Schreibe ein Programm, dass diese Datei einliest, die Zahlen sortiert und die Datei mit der sortierten Liste �berschreibt. 
  \item Modifiziere dein Programm nun so, dass in der ersten Zeile nicht mehr stehen muss, wie viele Zeilen noch folgen.
\end{enumerate}
\end{aufg}



\begin{aufg}
Erstelle ein Programm \verb|uniq|, dass zwei Dateinamen als Argumente erh�lt, die erste Datei wortweise\footnote{Ein ``Wort'' ist eine Zeichenfolge, die keine Whitespaces enth�t} einliest und ohne doppelte Eintr�ge in die zweite Datei schreibt (in irgendeiner Reihenfolge).
\end{aufg}



\begin{aufg}
Schreibe ein Programm, dass ein Labyrinth aus einer Datei einliest:
{\tt
\lstset{language=Delphi}
\begin{lstlisting}
XXXXXXXXXXXXXXXX
X X XXXXXXXXXX*X
X$X XX     XXX X
X X XX XXX XXX X
X   XX XXX XXX X
XXX X   XX XXX X
XXX   X        X
XXXXXXXXXXXXXXXX
\end{lstlisting}
}
\emph{Bemerkung: } Wir spezifizieren das Labyrinth hier nicht viel n�her, entscheide dich selbst vorher f�r ein Format. Soll die Gr��e des Labyrinths variabel sein oder fest? Soll die Gr��e in der ersten Zeile der Datei stehen oder nicht? Soll das Labyrinth quadratisch sein oder nicht? Soll es au�en herum immer mit $X$en begrenzt sein oder hast du vielleicht eine andere L�sung?

\vspace{1.5ex} Das Programm soll einen Weg vom Startpunkt (dem Stern) zum Schatz (dem Dollarzeichen) finden. Die $X$e sind W�nde und Leerzeichen sind Pfade. Markiere einen Weg mit Punkten und gebe das Labyrinth mit Weg in der Konsole aus.

\vspace{\fill}
\newpage


{\tt
\lstset{language=Delphi}
\begin{lstlisting}
XXXXXXXXXXXXXXXX
X X XXXXXXXXXX*X
X$X XX     XXX.X
X.X XX XXX XXX.X
X...XX XXX XXX.X
XXX.X...XX XXX.X
XXX...X........X
XXXXXXXXXXXXXXXX
\end{lstlisting}
}
\end{aufg}

\begin{aufg}
Implementiere den Strassen-Algorithmus zur schnellen Matrixmultiplikation: F�r $C = A \cdot B$ mit $A, B, C \in \R^{2^n \times 2^n}$ sei folgende Partitionierung gegeben: \[
A = \begin{pmatrix} A_{1,1} & A_{1,2} \\ A_{2,1} & A_{2,2} \end{pmatrix}; 
B = \begin{pmatrix} B_{1,1} & B_{1,2} \\ B_{2,1} & B_{2,2} \end{pmatrix}; 
C = \begin{pmatrix} C_{1,1} & C_{1,2} \\ C_{2,1} & C_{2,2} \end{pmatrix}; 
\] wobei alle Untermatrizen gleich gro� sind. Definiere 7 neue Matrizen: 
\[ \begin{array}{rl}
M_1 := & (A_{1,1} + A_{2,2}) \cdot (B_{1,1} + B_{2,2}) \\
M_2 := & (A_{2,1} + A_{2,2}) \cdot B_{1,1} \\
M_3 := & A_{1,1} \cdot (B_{1,2} - B_{2,2}) \\
M_4 := & A_{2,2} \cdot (B_{2,1} - B_{1,1}) \\
M_5 := & (A_{1,1} + A_{1,2}) \cdot B_{2,2} \\
M_6 := & (A_{2,1} - A_{1,1}) \cdot (B_{1,1} + B_{1,2}) \\
M_7 := & (A_{1,2} - A_{2,2}) \cdot (B_{2,1} + B_{2,2}) \\
\end{array} \] daraus ergibt sich dann f�r die L�sung $C$: \[ \begin{array}{rl}
C_{1,1} = & M_1 + M_4 - M_5 + M_7 \\
C_{1,2} = & M_3 + M_5 \\
C_{2,1} = & M_2 + M_4 \\
C_{2,2} = & M_1 - M_2 + M_3 + M_6 \\
\end{array} \] Dabei soll zur Berechnung der $M_i$ nat�rlich wieder der gleiche Algorithmus zur Matrixmultiplikation verwendet werden. Man spart insgesamt eine Matrixmultiplikation und verringert so den Aufwand von $O(n^3) = O(n^{\log_2(8)})$ auf $O(n^{\log_2(7)})$ also ungef�hr $O(n^{2,807})$. 
\end{aufg}


\rofoot{SS12}
\end{document}
