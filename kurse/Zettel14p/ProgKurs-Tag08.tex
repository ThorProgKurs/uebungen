\documentclass{uebungszettel}
\begin{document}

\begin{aufg}
\begin{enumerate}
\item Implementiere die Addition, Multiplikation, Potenzen und Division komplexer Zahlen. 
\item Implementiere die Addition, Multiplikation und Division sowie das Kürzen rationaler Zahlen. Schreibe dazu erst die Header-Datei.
\end{enumerate}
\end{aufg}


\begin{aufg}
Implementiere "`dynamische Arrays"'. Also Funktionen, die es leicht ermöglichen mit dynamisch großen Arrays zu arbeiten. Verwende -- wenn du möchtest -- eine Header-Datei in folgendem Stil:
\begin{codelisting}
\begin{lstlisting}[numbers=left,numberstyle=\tiny,frame=tlrb]
typedef struct {
  double *data; /* eigentliches Array */
  int length;   /* erstes nicht verwendetes Element */
  int _size;    /* Menge der allokierten Elemente */
} DBLARRAY;

/* initialisiert eine Array-Datenstruktur */
DBLARRAY *dblarray_init();

/* gibt eine Array-Datenstruktur wieder frei */
void dblarray_free(DBLARRAY *);

/* setzt den Wert an der Stelle i auf val 
 * falls noetig wird neuer Speicher allokiert und 
 * alle Elemente bis dorthin mit 0 initialisiert */
int dblarray_set(DBLARRAY *, int i, double val);

/* setze das erste nicht initialisierte Element auf 
 * val sollte es noch nicht existieren wird neuer 
 * Speicher allokiert */
int dblarray_push(DBLARRAY *, double val);

/* gib den Wert an der Stelle i zurueck */
double dblarray_get(DBLARRAY *, int i);

\end{lstlisting}
\end{codelisting}
Es ist gelegentlich günstig, sich vorher zu überlegen, wie man ein Modul verwendet, bevor man es implementiert. Schreibe dir also eine passende \verb|main.c|, die dein Modul benutzt um es zu testen.
\end{aufg}

\newpage

\begin{aufg}
Implementiere doppelt verkettete Listen, die \verb|double|-Variablen speichern.

\medskip\begin{codelisting}
\begin{lstlisting}[numbers=left,numberstyle=\tiny,frame=tlrb]
/* Definiere hier angemessene Strukturen fuer einen
   einzelnen Listeneintrag und die Liste selbst. */

/* Leere Liste erstellen */
LIST *list_create();

/* Element hinter E einfuegen, NULL heisst am Anfang */
LISTNODE *list_insert(LIST *L, LISTNODE *E, double p);

/* Element am Anfang bzw. Ende einfuegen */
LISTNODE *list_unshift(LIST *L, double p);
LISTNODE *list_push(LIST *L, double p);

/* Element am Anfang bzw. Ende entfernen und 
   die Daten zurueck geben */
double list_shift(LIST *L);
double list_pop(LIST *L);

/* eine Element aus der Liste entfernen */
void list_delete(LIST *L, LISTNODE *E);

/* zwei Listen zusammenfuegen */
LIST *list_merge(LIST *L, LIST *M);

/* Liste inklusive allen Elementen frei geben */
void list_free(LIST *L);

\end{lstlisting}
\end{codelisting}
\end{aufg}

\end{document}
