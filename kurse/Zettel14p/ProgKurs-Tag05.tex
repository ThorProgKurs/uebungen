\documentclass{uebungszettel}
\begin{document}

\begin{aufg}
Implementiere die Funktion, die den Inhalt zweier \verb|int|-Variablen vertauscht.
\end{aufg}

\begin{aufg}
Schreibe Funktionen \verb|square_to| und \verb|root_to|, die einen \verb|double|-Pointer entgegen nehmen, die dort stehende Variable quadrieren bzw. daraus die Wurzel ziehen und das Ergebnis sowohl zurück geben als auch an die gleiche Speicherstelle schreiben.
\end{aufg}

\begin{aufg}
Erweitere das "`mymath"'-Modul noch um eine Funktion, die zu den drei Koeffizienten $a, b, c \in \R$ einer quadratischen Gleichung $$
a \cdot x^2 + b \cdot x + c = 0 $$
die Lösungen berechnet.

\medskip \noindent \emph{Tip:} Eine Funktion kann nur einen Wert als Rückgabewert haben. Um mehr als einen Wert zurück zu geben, kann man Pointer verwenden.
\end{aufg}

\begin{aufg} Implementiere eine c-Datei zu folgender Header-Datei: 
\begin{codelisting}
\begin{lstlisting}[numbers=left,numberstyle=\tiny,frame=tlrb]
/* gibt die Laenge eines Strings zurueck */
int str_len(char *s); 

/* gibt 0 zurueck, wenn zwei strings gleich 
 * sind und 1 sonst */
int str_cmp(char *s1, char *s2);

/* kopiert s nach d und gibt d zurueck */
char *str_cpy(char *d, char *s);

/* haenge s2 and s1 an und gib s1 zurueck */ 
char *str_cat(char* s1, char* s2);
\end{lstlisting}
\end{codelisting}
und teste deinen Code mit der c-Datei auf der folgenden Seite.

\newpage

\begin{codelisting}
\begin{lstlisting}[numbers=left,numberstyle=\tiny,frame=tlrb]
#include <stdio.h>
#include "mystrings.h"

int main() {
	char p[100] = "Pepsi ";
	char c[100] = "Coca ";
	char suffix[10] = "Cola";
	char out[100];
	str_cpy(out,p); 
	str_cat(out,suffix); 
	str_cpy(p,out);
	str_cpy(out,c);
	str_cat(out,suffix);
	str_cpy(c,out);
	if (str_cmp(p,c)) {
		printf("%s",p);
		printf(" is not ");
		printf("%s",c);
		printf("\n");
	}
	return 0;
}
\end{lstlisting}
\end{codelisting}
\end{aufg}


\end{document}
