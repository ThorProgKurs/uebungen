\documentclass{uebungszettel}
\usepackage{algorithm,algorithmic}

\floatname{algorithm}{Algorithmus}
\newcommand{\SET}{\textbf{set}\ }
\newcommand{\CHOOSE}{\textbf{choose}\ }
\newcommand{\GOTO}{\textbf{goto}\ }
\renewcommand{\algorithmicrequire}{\textbf{Input:}}
\renewcommand{\algorithmicensure}{\textbf{Output:}}
\renewcommand{\listalgorithmname}{Algorithms}
\renewcommand{\algorithmiccomment}[1]{\\/* #1 */}

\newcommand{\utitle}{Tag 2}

\begin{document}
\newcommand{\ah}[2]{\ \\* \emph{(#1, #2)}\\}


\begin{aufg} Implementiert eine Datenstruktur f�r Graphen als Adjazenzliste. Schreibt eine Funktion, um Graphen mit Kantengewichten aus einer Datei einzulesen. Die Datei soll im folgenden Format sein:
\begin{itemize}
\item In der ersten Zeile der Datei stehen die Knotenzahl $n$, gefolgt von der Kantenanzahl $m$.
\item Danach folgen $m$ Zeilen, jede beschreibt eine Kante: Die Zeile beginnt mit der Nummer des ersten Knoten $v$, gefolgt von der Nummer des zweiten Knoten $w$, gefolgt vom Gewicht der Kante $(v,w)$. Das Gewicht sollte als \verb|double| eingelesen werden. 
\end{itemize}
\end{aufg}

\begin{aufg}
Implementiere eine Heap-Datenstruktur und den Dijkstra Algorithmus. Wir werden auf der Homepage einige Beispielinstanzen von Graphen haben. Der Startknoten f�r Dijkstra ist stets der Knoten mit der Nummer $0$.
\end{aufg}



\end{document}
