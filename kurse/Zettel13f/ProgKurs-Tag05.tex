\documentclass{uebungszettel}
\usepackage{hyperref}
\usepackage{algorithm,algorithmic}

\floatname{algorithm}{Algorithmus}
\newcommand{\SET}{\textbf{set}\ }
\newcommand{\CHOOSE}{\textbf{choose}\ }
\newcommand{\GOTO}{\textbf{goto}\ }
\renewcommand{\algorithmicrequire}{\textbf{Input:}}
\renewcommand{\algorithmicensure}{\textbf{Output:}}
\renewcommand{\listalgorithmname}{Algorithms}
\renewcommand{\algorithmiccomment}[1]{\\/* #1 */}

\newcommand{\utitle}{Tag 5}

\begin{document}
\newcommand{\ah}[2]{\ \\* \emph{(#1, #2)}\\}


\begin{aufg} Lade dir \href{http://www.cygwin.com}{Cygwin} herunter und installiere das Paket f�r \href{http://openmp.org/wp/openmp-specifications/}{OpenMP}. Kompiliere folgendes Beispielprogramm 

\smallskip\begin{codelisting}
\begin{lstlisting}[numbers=left,numberstyle=\tiny,frame=tlrb]
#include <stdio.h>
#include <omp.h>
int main() {
  double a [10000];
  #pragma omp parallel
  {
    int i;
    #pragma omp for
    for (i=0; i< 10000; i++)
      a[i] = i*i/2.0;
    #pragma omp master
    {
      printf("%f\n", a[3432]);
    }
  }
  return 0;
}
\end{lstlisting}
\end{codelisting}
\end{aufg}

\begin{aufg} Schreibe eine parallelisierte Funktion zur Multiplikation von Matrizen.
\end{aufg}

\begin{aufg}
Implementiere den Strassen-Algorithmus zur schnellen Matrixmultiplikation, nat�rlich parallelisiert.

F�r $C = A \cdot B$ mit $A, B, C \in \R^{2^n \times 2^n}$ sei folgende Partitionierung gegeben: 
\begin{align*}
A &= \begin{pmatrix} A_{1,1} & A_{1,2} \\ A_{2,1} & A_{2,2} \end{pmatrix}
&B &= \begin{pmatrix} B_{1,1} & B_{1,2} \\ B_{2,1} & B_{2,2} \end{pmatrix}
&C &= \begin{pmatrix} C_{1,1} & C_{1,2} \\ C_{2,1} & C_{2,2} \end{pmatrix} 
\end{align*}
wobei alle Untermatrizen gleich gro� sind. Definiere 7 neue Matrizen: 
\[ \begin{array}{rl}
M_1 := & (A_{1,1} + A_{2,2}) \cdot (B_{1,1} + B_{2,2}) \\
M_2 := & (A_{2,1} + A_{2,2}) \cdot B_{1,1} \\
M_3 := & A_{1,1} \cdot (B_{1,2} - B_{2,2}) \\
M_4 := & A_{2,2} \cdot (B_{2,1} - B_{1,1}) \\
M_5 := & (A_{1,1} + A_{1,2}) \cdot B_{2,2} \\
M_6 := & (A_{2,1} - A_{1,1}) \cdot (B_{1,1} + B_{1,2}) \\
M_7 := & (A_{1,2} - A_{2,2}) \cdot (B_{2,1} + B_{2,2}) \\
\end{array} \] daraus ergibt sich dann f�r die L�sung $C$: \[ \begin{array}{rl}
C_{1,1} = & M_1 + M_4 - M_5 + M_7 \\
C_{1,2} = & M_3 + M_5 \\
C_{2,1} = & M_2 + M_4 \\
C_{2,2} = & M_1 - M_2 + M_3 + M_6 \\
\end{array} \] Dabei soll zur Berechnung der $M_i$ nat�rlich wieder der gleiche Algorithmus zur Matrixmultiplikation verwendet werden. Man spart insgesamt eine Matrixmultiplikation und verringert so den Aufwand von $O(n^3) = O(n^{\log_2(8)})$ auf $O(n^{\log_2(7)})$ also ungef�hr $O(n^{2,807})$. 
\end{aufg}



\end{document}
