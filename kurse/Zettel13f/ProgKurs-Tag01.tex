\documentclass{uebungszettel}
\usepackage{algorithm,algorithmic}

\floatname{algorithm}{Algorithmus}
\newcommand{\SET}{\textbf{set}\ }
\newcommand{\CHOOSE}{\textbf{choose}\ }
\newcommand{\GOTO}{\textbf{goto}\ }
\renewcommand{\algorithmicrequire}{\textbf{Input:}}
\renewcommand{\algorithmicensure}{\textbf{Output:}}
\renewcommand{\listalgorithmname}{Algorithms}
\renewcommand{\algorithmiccomment}[1]{\\/* #1 */}

\newcommand{\utitle}{Tag 1}

\begin{document}
\newcommand{\ah}[2]{\ \\* \emph{(#1, #2)}\\}
\begin{aufg}
Implementiere doppelt verkettete Listen, die beliebige Daten speichern k�nnen (als \verb|void *|). Die Liste soll zus�tzlich einen Stack enthalten, auf dem gel�schte Elemente abgelegt werden, damit man sie wiederherstellen kann.

\begin{codelisting}
\begin{lstlisting}[numbers=left,numberstyle=\tiny,frame=tlrb]
/* Leere Liste erstellen */
LIST *list_create();

/* Element hinter E einf�gen, NULL hei�t am Anfang */
LISTNODE *list_insert(LIST *L, LISTNODE *E, void *p);
/* Element hinter E einf�gen, dessen datenpointer
   auf n Zellen allokierten Speicher zeigt. Sollte bei
   Speichermangel einen Nullpointer liefern. */
LISTNODE *list_insert_alloc(LIST *L, LISTNODE *E, 
	unsigned int n);

/* Element am Anfang bzw. Ende entfernen und 
   die Daten zur�ck geben */
void *list_shift(LIST *L);
void *list_pop(LIST *L);

/* eine Element aus der Liste entfernen */
void list_delete(LIST *L, LISTNODE *E);

/* Liste inklusive allen Elementen frei geben */
void list_free(LIST *L);

/* Das eben gel�schte Element wiederherstellen */
void list_undelete(LIST *L);
/* Alle gel�schten Elemente wiederherstellen */
void list_dance(LIST *L);

\end{lstlisting}
\end{codelisting}
\end{aufg}

\begin{aufg} Schreibe ein Programm, dass Worte aus einer Datei einliest, diese in eine Liste einf�gt, anschlie�end die Liste lexikographisch sortiert und dann die Worte sortiert ausgibt.
\end{aufg}

\end{document}
