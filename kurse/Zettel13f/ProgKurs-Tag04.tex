\documentclass{uebungszettel}
\usepackage{hyperref}
\usepackage{algorithm,algorithmic}

\floatname{algorithm}{Algorithmus}
\newcommand{\SET}{\textbf{set}\ }
\newcommand{\CHOOSE}{\textbf{choose}\ }
\newcommand{\GOTO}{\textbf{goto}\ }
\renewcommand{\algorithmicrequire}{\textbf{Input:}}
\renewcommand{\algorithmicensure}{\textbf{Output:}}
\renewcommand{\listalgorithmname}{Algorithms}
\renewcommand{\algorithmiccomment}[1]{\\/* #1 */}

\newcommand{\utitle}{Tag 4}

\begin{document}
\newcommand{\ah}[2]{\ \\* \emph{(#1, #2)}\\}


\begin{aufg} Lade dir \href{http://www.cygwin.com}{Cygwin} herunter und installiere das Paket f�r \href{http://www.gmplib.org/}{GMP}. Kompiliere folgendes Beispielprogramm 

\smallskip\begin{codelisting}
\begin{lstlisting}[numbers=left,numberstyle=\tiny,frame=tlrb]
/* Gebe einige Zufallszahlen aus. */
#include <stdio.h>
#include <gmp.h>
#include <time.h>

int main() {
  mpz_t a;
  mpz_init(a); 
  mpz_set_str(a,"856490000",10);
  mpz_mul(a,a,a);
  gmp_printf("%Z\n",a);
  mpz_clear(a);
  return 0;
}
\end{lstlisting}
\end{codelisting}
\end{aufg}

\noindent In den folgenden Aufgaben geht es um elliptische Kurven. Zum Testen empfehlen wir
\[ E := \left\{\, (x,y) \in \mathbb{F}_p^2 \, \middle\vert \,
   y^2 = x^3 + a x + b \, \right\} \cup \left\{\, \mathcal{O} \, \right\} \]
mit den folgenden Parametern aus dem Brainpool Standard \footnote{\href{http://www.ecc-brainpool.org/download/draft-lochter-pkix-brainpool-ecc-00.txt}{http://www.ecc-brainpool.org/download/draft-lochter-pkix-brainpool-ecc-00.txt}}:

\smallskip\begin{codelisting}
\begin{lstlisting}[numbers=left,numberstyle=\tiny,frame=tlrb]
Curve-ID: brainpoolP256r1
p = A9FB57DBA1EEA9BC3E660A909D838D726E3BF623D526202820
    13481D1F6E5377
a = 7D5A0975FC2C3057EEF67530417AFFE7FB8055C126DC5C6CE9
    4A4B44F330B5D9
b = 26DC5C6CE94A4B44F330B5D9BBD77CBF958416295CF7E1CE6B
    CCDC18FF8C07B6
x = 8BD2AEB9CB7E57CB2C4B482FFC81B7AFB9DE27E1E3BD23C23A
    4453BD9ACE3262
y = 547EF835C3DAC4FD97F8461A14611DC9C27745132DED8E545C
    1D54C72F046997
q = A9FB57DBA1EEA9BC3E660A909D838D718C397AA3B561A6F790
    1E0E82974856A7
h = 1
\end{lstlisting}
\end{codelisting}

\noindent  Die Parameter sind im Hexadezimalsystem angegeben. Hier ist au�erdem $g = (x,y)$ ein Punkt auf der Kurve und $q = |\left<g\right>|$ die Anzahl Punkte in der von $g$ erzeugten Untergruppe.

\begin{aufg} Implementiere unter Verwendung der GMP ein Modul, welches die Punktegruppe einer elliptischen Kurve implementiert. Es sollten mindestens Funktionen zur Verf�gung stehen, Punkte auf einer durch Parameter $a$ und $b$ gegebenen Kurve zu addieren.
\end{aufg}

\begin{aufg} Implementiere eine Funktion, die zu einem Punkt $p \in E$ und einer Zahl $n \in \mathbb{Z}_+$ den Punkt $n \cdot p \in E$ auf der Kurve berechnet. Eine \emph{naive} M�glichkeit ist es selbstverst�ndlich, den Punkt $p$ genau $n$ mal aufzuaddieren. Wir kennen jedoch eine Methode zur schnellen Potenzierung von Zahlen, die prinzipiell auch hier anwendbar ist. (Stichwort: Double \& Add) \end{aufg}

\begin{aufg} Implementiere Lenstras EC-Faktorisierungsalgorithmus. Eine Zahl zum testen wird auf der \href{http://ah-effect.net/}{Kurshomepage} zum Download verf�gbar sein.

\begin{algorithm}[H]
\caption{Lenstras EC-Faktorisierungsalgorithmus}
\label{alg2}
\algsetup{indent=1.5em}
\begin{algorithmic}[1]
\REQUIRE Zahl $n \in \mathbb{Z}_+$.
\ENSURE Ein Teiler $d$ von $n$
\STATE W�hle zuf�llige $x_0,y_0,a \in \mathbb{Z}_n$
\STATE \SET $b := y_0^2 - x_0^3 - ax_0$
\STATE Sei $E$ die Kurve zu den Parametern $a$ und $b$.
\STATE \SET $p := (x_0,y_0) \in E$
\FOR{$m = 2,3,4,\ldots$}
	\STATE \SET $p := m \cdot p \pmod n$
	\IF{Berechnung schl�gt fehl}
		\STATE Wir haben ein $x$ gefunden, welches kein Inverses modulo $n$ hat.
		\STATE \SET $d := \gcd(x,n)$
		\IF{$d < n$}
			\RETURN $d$
		\ELSIF{$d = n$}
			\STATE Starte Algorithmus neu
		\ENDIF
	\ENDIF
\ENDFOR
\end{algorithmic}
\end{algorithm}
\end{aufg}


\end{document}
