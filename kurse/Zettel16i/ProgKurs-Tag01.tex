\documentclass{uebungszettel}
\begin{document}

\begin{aufg}
Installiere einen Compiler auf deinem Computer und kompiliere ein Hallo-Welt-Programm. Informationen dazu und einen Download-Link für Cygwin findest du im Skript auf dem USB-Stick oder der Webseite des Kurses:
\begin{center}
	\verb|http://www.ah-effect.net/ |
\end{center}
\end{aufg}

\begin{aufg}
Schreibe ein Programm, dass den Wert der folgenden Funktion ausgibt (für eine fest in den Quellcode geschriebene \verb|int|-Variable):
\[
	f(n) = \left\{ \begin{array}{ll}
	\frac{n}{2} & \text{wenn } n \text{ gerade} \\
	\frac{n+1}{2} & \text{wenn } n \text{ ungerade} \\
	\end{array}
	\right.
\]
Und das geht natürlich nur mit Wissen aus der Vorlesung.
\end{aufg}

\begin{aufg}
    Was machen folgende Algorithmen? (nächste Seite)

    \begin{algorithm}[H]
        \caption{}
        \algsetup{indent=1.5em}
        Eingabe ist eine ganze Zahl $c \in \mathbb N$. Dies ist am einfachsten
        als Variable bereitzustellen. Ausgabe ist entweder »Ja« oder »Nein«.

        Schritte:
        \begin{itemize}
            \item
                Setze $n$ auf 2.
            \item
                Wiederhole unendlich oft …
                \begin{itemize}
                    \item
                        Falls $n > \sqrt c$, so gib »Ja« aus und beende das
                        Programm.
                    \item
                        Falls $n$ ein Teiler von $c$ ist, so gib »Nein« aus und
                        beende das Programm.
                    \item
                        Erhöhe $n$ um eins.
                \end{itemize}
        \end{itemize}
    \end{algorithm}

    \begin{algorithm}[H]
        \caption{}
        \algsetup{indent=1.5em}
        Eingabe sind die ganzen Zahlen $a, b \in \mathbb N$. Ausgabe ist eine
        ganze Zahl $k \in \mathbb N$.

        Schritte:
        \begin{itemize}
            \item Wiederhole unendlich oft …
                \begin{itemize}
                    \item
                        Falls $a = 0$ ist, gib $b$ aus und beende das Programm.
                    \item
                        Falls $b = 0$ ist, gib $a$ aus und beende das Programm.
                    \item
                        Falls $a > b$ ist, setze $a$ auf $a - b$.
                    \item
                        Falls obige Bedingung nicht erfüllt ist, setze $b$ auf
                        $b - a$.
                \end{itemize}
        \end{itemize}
    \end{algorithm}

    \begin{algorithm}[H]
        \caption{}
        \algsetup{indent=1.5em}
        Eingabe ist eine reelle nicht-negative Zahl $a \in \mathbb R_0^+$.
        Ausgabe ist eine reelle Zahl $x \in \mathbb R$.

        Schritte:
        \begin{itemize}
            \item
                Setze $x$ auf 2 und $y$ auf 1.
            \item
                Wiederhole unendlich oft …
                \begin{itemize}
                    \item 
                        Falls $|x - y| \leq 10^{-10}$ ist, gib $x$ aus und
                        beende das Programm.
                    \item
                        Setze $x$ auf $y$.
                    \item
                        Setze $y$ auf $\frac12 (x + \frac ax)$.
                \end{itemize}
        \end{itemize}
    \end{algorithm}
\end{aufg}

\begin{aufg}
Wettbewerb: Gegeben ist folgender Programmrumpf:
\begin{codelisting}
\begin{lstlisting}[numbers=left,numberstyle=\tiny,frame=tlrb]
#include <stdio.h>
int main(int argc, char **argv) {
	int x = 2;
	/* dein Code hier */
	printf("%i\n", x);
	return 0;
}
\end{lstlisting}
\end{codelisting}
Füge an der markierten Stelle C-Code ein, sodass der Wert von $2^{\left(3^3\right)}$ ausgegeben wird. Wer in diesem Raum am wenigsten Zeichen dafür benötigt bekommt eine Dose Cola\footnote{Wenn er sie sich kauft.}. Erlaubt sind aber nur die Zeichen 
\begin{center}
	x \quad + \quad - \quad * \quad / \quad =
\end{center}
und das Semikolon. Zeilenumbrüche und Leerzeichen können natürlich nach belieben verwendet werden, da sie vom Compiler ignoriert werden.
\end{aufg}

\end{document}
