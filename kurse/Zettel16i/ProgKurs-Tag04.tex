\documentclass{uebungszettel}
\begin{document}

\begin{aufg} Lege ein Modul \verb|mymath.c| / \verb|mymath.h| an, in dem du die bisher geschriebenen Funktionen 
auslagerst. Implementiere ausserdem die folgenden Funktionen:
\begin{enumerate}
  \item $\log_a(x)$ mit beliebigen Argumenten $a \in \R$ und $x \in \R^+$
  \item Fakultät $n!$ für $n \in \N$ (falls nicht schon lange geschehen)
  \item für die $k$-te Wurzel aus $x\in \R^+_0$ mit $k \in \R^+$
  \item eine Funktion, die zu zwei Seiten eines Dreiecks und ihrem eingeschlossenen Winkel die Länge der dritten Seite zurück gibt
\end{enumerate}
\end{aufg}

\noindent Wir brauchen im folgenden eine Potenzfunktion, die zwei Fließkommazahlen als Argumente akzeptiert. Falls du diese 
Funktion gestern geschrieben hast, sollte sie jetzt im \verb|mymath|-Modul verfügbar sein. Diese Funktion wird aber 
vermutlich zu langsam sein, daher gibt es die Funktion 

\begin{verbatim}
double pow(double x, double y);
\end{verbatim}

in der Systemheader \verb|<math.h>|. Im Skript findest du bei Interesse im Anhang eine Referenz einiger Systembibliotheken.

\begin{aufg}
Implementiere die Riemann'sche Zeta-Funktion für $s \in \R$: $$
\zeta(s) := \sum_{k=1}^\infty \frac{1}{k^s} $$
\end{aufg}

\begin{aufg}
Erweitere das "`mymath"'-Modul noch um eine Funktion, die zu den drei Koeffizienten $a, b, c \in \R$ einer quadratischen Gleichung $$
a \cdot x^2 + b \cdot x + c = 0 $$
die Lösungen berechnet und die größere Lösung zurück gibt.
\end{aufg}

\end{document}
