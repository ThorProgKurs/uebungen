\documentclass{uebungszettel}
\begin{document}
\begin{aufg} Auf der Homepage gibt es ein Modul \verb|rationals|, in dem rationale Zahlen implementiert sind. Binde es in ein neues Projekt ein und deklariere ein Array von Brüchen und sortiere es mit Hilfe von \verb|qsort|.
\end{aufg}

\begin{aufg}
Das $8$-Damen Problem ist wie folgt definiert: Platziere $8$ Damen auf einem gewöhnlichen Schachbrett so, dass sie sich paarweise nicht bedrohen.

Das $n$-Damen Problem ist analog definiert (auf einem $n \times n$-Schachbrett). Schreibe ein Programm, dass das $n$-Damen Problem löst (falls möglich). Es hilft bei solchen Fragestellung häufig, sich Spezialfälle aufzumalen (betrachte beispielsweise das $3$- oder $4$-Damen Problem). 
\end{aufg}

\begin{aufg}
Verwende deine Listen-Implementierung um eine sog. \verb|int|-Stack und eine \verb|double|-Queue zu implementieren.

Ein \emph{Stack} ist eine Datenstruktur zum speichern mehrerer Einträge, welche nur die Operationen push und pop unterstützt: das heißt, man kann immer nur am Ende des Stacks neue Daten anfügen und auch nur von dort Daten entfernen (Natürlich würde man einen Stack für gewöhnlich mit einer Array implementieren)

Eine \emph{Queue} ist eine Datenstruktur, welche ausschließlich die Operationen unshift und pop unterstützt: Man kann nur am Anfang Daten anfügen und nur am Ende Daten entfernen.

\end{aufg}

\begin{aufg} Verallgemeinere dein Listen-Modul von gestern, damit es beliebige Daten (also \verb|void *|) speichern kann.
\end{aufg}
\end{document}
