\documentclass{uebungszettel}
\usepackage{algorithm,algorithmic}

\floatname{algorithm}{Algorithmus}
\newcommand{\SET}{\textbf{set}\ }
\newcommand{\CHOOSE}{\textbf{choose}\ }
\newcommand{\GOTO}{\textbf{goto}\ }
\renewcommand{\algorithmicrequire}{\textbf{Input:}}
\renewcommand{\algorithmicensure}{\textbf{Output:}}
\renewcommand{\listalgorithmname}{Algorithms}
\renewcommand{\algorithmiccomment}[1]{\\/* #1 */}

\newcommand{\utitle}{Tag 3}

\begin{document}
\newcommand{\ah}[2]{\ \\* \emph{(#1, #2)}\\}


\begin{aufg} Implementiert eine Datenstruktur f�r Graphen als Adjazenzliste. Schreibt eine Funktion, um Graphen mit Kantengewichten aus einer Datei nach dem in der Vorlesung besprochenen Format einzulesen. In weiser Voraussicht k�nnte diese Einleseroutine den Index des Knotens \verb|s| zur�ckliefern, auch wenn dieser f�r die unmittelbar folgenden Aufgaben noch nicht relevant ist.
\end{aufg}

\begin{aufg}
Schreibe eine Funktion, die irgendeinen gerichteten Kreis in einem Graphen in linearer Zeit findet und diesen ausgibt. Implementiere dazu \textbf{D}epth \textbf{F}irst \textbf{S}earch (DFS) mit einem Callback-Argument:
\begin{codelisting}
\begin{lstlisting}[numbers=left,numberstyle=\tiny,frame=tlrb]
typedef int (*DFS_CALLBACK)(
	GRAPH *G,    
	int v,      /* Knotenindex */
	void *data  /* Zusatzdaten (optional) */
);

/* F�hrt DFS auf dem Graphen G durch. Bei jedem Knoten 
   wird (*cb) aufgerufen, wobei der Knotenindex und
   der data-Pointer �bergeben werden. Wenn die 
   R�ckgabe dieses Aufrufs 0 ist, so soll die Tiefen-
   suche abgebrochen werden. */
void graph_dfs(GRAPH* G, DFS_CALLBACK cb, void *data);
\end{lstlisting}
\end{codelisting}
\end{aufg}

\begin{aufg} Verwendet eure Heaps von gestern, um den Dijkstra-Algorithmus zu implementieren. Auf der Homepage findet ihr eine Graphendatei im besprochenen Format zum Testen.
\medskip

\noindent\textbf{Bonus!} Wir werden morgen zuf�llig Knotenindizes abfragen, und wer zuerst den Abstand von \verb|s| zu diesem Knoten nennen kann, bekommt eine Coke. \textbf{Bonus-Coke!!!}
\end{aufg}

\begin{aufg} Kauft euch Firefly und seht es euch an.
\end{aufg}

\begin{aufg}[$\ast$] Implementiere auch Breitensuche auf deinem Graphen mit Callback. Dazu musst du deine Listenimplementierung als Queue verwenden.
\end{aufg}

\end{document}
