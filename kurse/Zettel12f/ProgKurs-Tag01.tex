\documentclass{uebungszettel}
\usepackage{algorithm,algorithmic}

\floatname{algorithm}{Algorithmus}
\newcommand{\SET}{\textbf{set}\ }
\newcommand{\CHOOSE}{\textbf{choose}\ }
\newcommand{\GOTO}{\textbf{goto}\ }
\renewcommand{\algorithmicrequire}{\textbf{Input:}}
\renewcommand{\algorithmicensure}{\textbf{Output:}}
\renewcommand{\listalgorithmname}{Algorithms}
\renewcommand{\algorithmiccomment}[1]{\\/* #1 */}

\newcommand{\utitle}{Tag 1}

\begin{document}
\newcommand{\ah}[2]{\ \\* \emph{(#1, #2)}\\}
\begin{aufg}
Implementiere doppelt verkettete Listen, die beliebige Daten speichern k�nnen (als \verb|void *|). Die Liste soll zus�tzlich einen Stack enthalten, auf dem gel�schte Elemente abgelegt werden, damit man sie wiederherstellen kann.

\begin{codelisting}
\begin{lstlisting}[numbers=left,numberstyle=\tiny,frame=tlrb]
/* Leere Liste erstellen */
LIST *list_create();

/* Element hinter E einf�gen, NULL hei�t am Anfang */
LISTNODE *list_insert(LIST *L, LISTNODE *E, void *p);
/* Element hinter E einf�gen, dessen datenpointer
   auf n Zellen allokierten Speicher zeigt. Sollte bei
   Speichermangel einen Nullpointer liefern. */
LISTNODE *list_insert_alloc(LIST *L, LISTNODE *E, 
	unsigned int n);

/* Element am Anfang bzw. Ende einf�gen */
LISTNODE *list_unshift(LIST *L, void *p);
LISTNODE *list_push(LIST *L, void *p);

/* Element am Anfang bzw. Ende entfernen und 
   die Daten zur�ck geben */
void *list_shift(LIST *L);
void *list_pop(LIST *L);

/* eine Element aus der Liste entfernen */
void list_delete(LIST *L, LISTNODE *E);

/* zwei Listen zusammenf�gen */
LIST *list_merge(LIST *L, LIST *M);

/* Liste inklusive allen Elementen frei geben */
void list_free(LIST *L);

/* Das eben gel�schte Element wiederherstellen */
void list_undelete(LIST *L);
/* Alle gel�schten Elemente wiederherstellen */
void list_dance(LIST *L);

\end{lstlisting}
\end{codelisting}
\end{aufg}

\begin{aufg} Erstelle eine Liste, f�ge einige Strings in sie ein und sortiere die Liste lexikographisch.
\end{aufg}

\begin{aufg}
\begin{enumerate}
\item
Implementiere die Addition, Multiplikation, Potenzen und Division komplexer Zahlen. Verwende dazu folgende Header-Datei:
\begin{codelisting}
\begin{lstlisting}[numbers=left,numberstyle=\tiny,frame=tlrb]
typedef struct _COMPLEX {
	double real;
	double imag;
} COMPLEX;

COMPLEX cplx_add(COMPLEX a, COMPLEX b);
COMPLEX cplx_mult(COMPLEX a, COMPLEX b);
COMPLEX cplx_pot(COMPLEX a, unsigned long n);
COMPLEX cplx_div(COMPLEX a, COMPLEX b);
\end{lstlisting}
\end{codelisting}
\item Implementiere die Addition, Multiplikation und Division sowie das K�rzen rationaler Zahlen. Schreibe dazu erst die Header-Datei.
\end{enumerate}
\end{aufg}
\end{document}
