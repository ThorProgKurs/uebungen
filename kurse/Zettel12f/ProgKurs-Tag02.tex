\documentclass{uebungszettel}
\usepackage{algorithm,algorithmic}

\floatname{algorithm}{Algorithmus}
\newcommand{\SET}{\textbf{set}\ }
\newcommand{\CHOOSE}{\textbf{choose}\ }
\newcommand{\GOTO}{\textbf{goto}\ }
\renewcommand{\algorithmicrequire}{\textbf{Input:}}
\renewcommand{\algorithmicensure}{\textbf{Output:}}
\renewcommand{\listalgorithmname}{Algorithms}
\renewcommand{\algorithmiccomment}[1]{\\/* #1 */}

\newcommand{\utitle}{Tag 2}

\begin{document}
\newcommand{\ah}[2]{\ \\* \emph{(#1, #2)}\\}

\begin{aufg}
Implementiere einen Heap in einem eigenen Modul, wobei die Knoten im Heap beliebige Informationen speichern k�nnen.
\end{aufg}

\begin{aufg}
Schreibe eine Funktion, die den Heap graphisch auf der Kommandozeile ausgibt. Zum Beispiel etwa so, oder kl�ger:
\begin{lstlisting}[numbers=left,numberstyle=\tiny,frame=tlrb]
                     2
             ________|________
             3               4
         ____|____       ____|____
         6       6       5       5
       __|__   __|__   __|__   __|__
       9   9   7   9   6   7   6
\end{lstlisting}
\end{aufg}

\begin{aufg} Schreibe eine Funktion, die eine Liste von Punkten
\begin{codelisting}
\begin{lstlisting}[numbers=left,numberstyle=\tiny,frame=tlrb]
typedef struct POINT {
	double x;
	double y;
} POINT;
\end{lstlisting}
\end{codelisting}
nach ihrem Abstand von einem vorgegebenen Punkt sortiert. Verwende dazu den Heap aus Aufgabe 1.
\end{aufg}

\end{document}
