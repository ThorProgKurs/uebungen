\documentclass{uebungszettel}
\usepackage{algorithm,algorithmic}
\usepackage{enumitem}
\usepackage{tabularx}
\floatname{algorithm}{Algorithmus}
\newcommand{\SET}{\textbf{set}\ }
\newcommand{\CHOOSE}{\textbf{choose}\ }
\newcommand{\GOTO}{\textbf{goto}\ }
\renewcommand{\algorithmicrequire}{\textbf{Input:}}
\renewcommand{\algorithmicensure}{\textbf{Output:}}
\renewcommand{\listalgorithmname}{Algorithms}
\renewcommand{\algorithmiccomment}[1]{\\/* #1 */}

\newcommand{\utitle}{Tag 10}

\begin{document}
\newcommand{\ah}[2]{\ \\* \emph{(#1, #2)}\\}
\newcommand{\power}{\mathrm{power}}




\begin{aufg} Implementiere eine Funktion die zu einem gegebenen Funktionenpointer $f:\R \rightarrow \R$, einem Dateinamen, einer Schrittweite $s \in \R$, einer Startstelle $x_1$ und einer Endstelle $x_2$ die Wertetabelle der Funktion zwischen $x_1$ und $x_2$ zur Schrittweite $s$ speichert. Dabei sollen $x$ und $f(x)$ durch einen Tabulator getrennt werden und jedes Paar $(x, f(x))$ in einer eigenen Zeile stehen. Etwa wäre die Ausgabe für $f=\cos$ zwischen $x_1=0$ und $x_2=0$ mit Schrittweite $s=0.1$ die folgende:
\begin{codelisting}
\begin{lstlisting}[numbers=left,numberstyle=\tiny,frame=tlrb]
0.0 1.0
0.1 0.995004165278
0.2 0.980066577841
0.3 0.955336489126
0.4 0.921060994003
0.5 0.87758256189
0.6 0.82533561491
0.7 0.764842187284
0.8 0.696706709347
0.9 0.621609968271
1.0 0.540302305868
\end{lstlisting}
\end{codelisting}
\end{aufg}

\begin{aufg}
In dieser Aufgabe geht es um numerische Integration.
\begin{enumerate}
\item Implementiere eine Integrationsfunktion, die das Intervall $[a, b]$ in $n$ gleich große Teile aufteilt, für diese jeweils die Trapezsumme (aus der Vorlesung) berechnet und diese aufsummiert:

\begin{codelisting}
\begin{lstlisting}[numbers=left,numberstyle=\tiny,frame=tlrb]
double integrate(double a, double b, 
  double (*f)(double), unsigned int n); 
\end{lstlisting}
\end{codelisting}

\item Schreibe nun eine Funktion, die nicht die Anzahl der Teilintervalle erhält, sondern eine ``Fehlertoleranz'' $e$. Die Funktion die Aufteilung solange verfeinern, bis sich der approximierte Wert für das Integral durch eine Verfeinerung nur noch um weniger als $e$ ändern würde. 
\end{enumerate}
\end{aufg}



\end{document}
