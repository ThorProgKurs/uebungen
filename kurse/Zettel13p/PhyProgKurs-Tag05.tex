\documentclass{uebungszettel}
\usepackage{algorithm,algorithmic}
\usepackage{enumitem}
\floatname{algorithm}{Algorithmus}
\newcommand{\SET}{\textbf{set}\ }
\newcommand{\CHOOSE}{\textbf{choose}\ }
\newcommand{\GOTO}{\textbf{goto}\ }
\renewcommand{\algorithmicrequire}{\textbf{Input:}}
\renewcommand{\algorithmicensure}{\textbf{Output:}}
\renewcommand{\listalgorithmname}{Algorithms}
\renewcommand{\algorithmiccomment}[1]{\\/* #1 */}

\newcommand{\utitle}{Tag 5}

\begin{document}
\newcommand{\ah}[2]{\ \\* \emph{(#1, #2)}\\}
\newcommand{\power}{\mathrm{power}}


\begin{aufg}
Schreibe ein Modul \verb|arrayhelpers|, das einige nützliche Funktion zum \verb|int|-Array-Handling enthält:
\begin{enumerate}
\item Array zeilenweise oder mit Kommata getrennt ausgeben
\item Array sortieren
\item Alle Felder eines Arrays mit einem Wert initialisieren
\item Array um $1$ rotieren (d.h. das hinterste Element an erste Stelle schreiben und alle anderen Elemente um eins nach hinten schieben)
\item Array um $k$ rotieren 
\item Array umdrehen
\item Ein Array in einem anderen suchen und die Position zurück geben. Sollte das Array nicht im anderen enthalten sein, so soll der Rückgabewert $-1$ sein.

\emph{Beispiel:} 
\begin{codelisting}
\begin{lstlisting}[numbers=left,numberstyle=\tiny,frame=tlrb]
int A[10] = {1, 2, 3, 4, 5, 6, 7, 8, 9, 10};
int B[3] = {4, 5, 6}
int C[2] = {5, 7}
int D[2] = {9, 10}
\end{lstlisting}
\end{codelisting}
Hier gilt: \verb|B| ist an $3$-ter Stelle in \verb|A| enthalten und \verb|D| an $8$-ter. Das Array \verb|C| ist garnicht in \verb|A| enthalten, darum wird der Rückgabewert $-1$ sein.

\end{enumerate}
\end{aufg}

\noindent Bevor du dir Stundenlang den Kopf über die Sternchen-Aufgabe zerbrichst, mache lieber Aufgabe 3.

\begin{aufg} Wir wolle ein Array mit $n$ Einträgen als Permutation interpretieren, wenn jede Zahl von $0$ bis $n-1$ darin vor kommt.
\begin{enumerate}
\item Schreibe eine Funktion, die prüft, ob ein Array eine Permutation ist.
\item * Schreibe eine Funktion, die ein Array und ein Permutationsarray entgegennimmt und das Array entsprechend permutiert.
\end{enumerate}
\end{aufg}

\newpage

\begin{aufg} Implementiere eine c-Datei zu folgender Header-Datei: 
\begin{codelisting}
\begin{lstlisting}[numbers=left,numberstyle=\tiny,frame=tlrb]
/* gibt die Länge eines Strings zurück */
int str_len(char *s); 

/* gibt 0 zurück, wenn zwei strings gleich 
 * sind und 1 sonst */
int str_cmp(char *s1, char *s2);

/* kopiert s nach d und gibt d zurück */
char *str_cpy(char *d, char *s);

/* hänge s2 and s1 an und gib s1 zurück */ 
char *str_cat(char* s1, char* s2);
\end{lstlisting}
\end{codelisting}
und teste deinen Code mit folgendem Modul:
\begin{codelisting}
\begin{lstlisting}[numbers=left,numberstyle=\tiny,frame=tlrb]
#include <stdio.h>
#include "mystrings.h"

int main() {
	char p[100] = "Pepsi ";
	char c[100] = "Coca ";
	char suffix[10] = "Cola";
	char out[100];
	str_cpy(out,p); 
	str_cat(out,suffix); 
	str_cpy(p,out);
	str_cpy(out,c);
	str_cat(out,suffix);
	str_cpy(c,out);
	if (str_cmp(p,c)) {
		printf("%s",p);
		printf(" is not ");
		printf("%s",c);
		printf("\n");
	}
	return 0;
}
\end{lstlisting}
\end{codelisting}
\end{aufg}



\end{document}
