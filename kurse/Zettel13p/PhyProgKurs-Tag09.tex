\documentclass{uebungszettel}
\usepackage{algorithm,algorithmic}
\usepackage{enumitem}
\usepackage{tabularx}
\floatname{algorithm}{Algorithmus}
\newcommand{\SET}{\textbf{set}\ }
\newcommand{\CHOOSE}{\textbf{choose}\ }
\newcommand{\GOTO}{\textbf{goto}\ }
\renewcommand{\algorithmicrequire}{\textbf{Input:}}
\renewcommand{\algorithmicensure}{\textbf{Output:}}
\renewcommand{\listalgorithmname}{Algorithms}
\renewcommand{\algorithmiccomment}[1]{\\/* #1 */}

\newcommand{\utitle}{Tag 9}

\begin{document}
\newcommand{\ah}[2]{\ \\* \emph{(#1, #2)}\\}
\newcommand{\power}{\mathrm{power}}




\begin{aufg}
~\begin{enumerate}
\item
Implementiere die Addition, Multiplikation, Potenzen und Division komplexer Zahlen. Verwende dazu folgende Header-Datei:
\begin{codelisting}
\begin{lstlisting}[numbers=left,numberstyle=\tiny,frame=tlrb]
#ifndef _COMPLEX__H
#define _COMPLEX__H

typedef struct _COMPLEX {
	double real;
	double imag;
} COMPLEX;

COMPLEX cplx_add(COMPLEX a, COMPLEX b);
COMPLEX cplx_mul(COMPLEX a, COMPLEX b);
COMPLEX cplx_pow(COMPLEX a, unsigned long n);
COMPLEX cplx_div(COMPLEX a, COMPLEX b);

#endif
\end{lstlisting}
\end{codelisting}
\item Implementiere die Addition, Multiplikation und Division sowie das Kürzen rationaler Zahlen. Schreibe dazu erst die Header-Datei.
\end{enumerate}
\end{aufg}


\begin{aufg}
Eine von den beiden Aufgaben vom letzten Zettel fertig machen. Aber dann schon die Listen machen.
\end{aufg}

\vfil
\begin{center}\bf FLIP ME\end{center}
\vfil\newpage

\begin{aufg} Lese noch einmal im Skript die Sektion 7.5 und implementiere doppelt verkettete Listen, die \verb|double|-Variablen speichern.

\medskip\begin{codelisting}
\begin{lstlisting}[numbers=left,numberstyle=\tiny,frame=tlrb]
/* Definiere hier angemessene Strukturen für einen
   einzelnen Listeneintrag und die Liste selbst. */

/* Leere Liste erstellen */
LIST *list_create();

/* Element hinter E einfügen, NULL heißt am Anfang */
LISTNODE *list_insert(LIST *L, LISTNODE *E, double p);

/* Element am Anfang bzw. Ende einfügen */
LISTNODE *list_unshift(LIST *L, double p);
LISTNODE *list_push(LIST *L, double p);

/* Element am Anfang bzw. Ende entfernen und 
   die Daten zurück geben */
double list_shift(LIST *L);
double list_pop(LIST *L);

/* eine Element aus der Liste entfernen */
void list_delete(LIST *L, LISTNODE *E);

/* zwei Listen zusammenfügen */
LIST *list_merge(LIST *L, LIST *M);

/* Liste inklusive allen Elementen frei geben */
void list_free(LIST *L);

\end{lstlisting}
\end{codelisting}
\end{aufg}



\end{document}
