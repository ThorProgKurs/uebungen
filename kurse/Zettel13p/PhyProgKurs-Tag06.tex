\documentclass{uebungszettel}
\usepackage{algorithm,algorithmic}
\usepackage{enumitem}
\floatname{algorithm}{Algorithmus}
\newcommand{\SET}{\textbf{set}\ }
\newcommand{\CHOOSE}{\textbf{choose}\ }
\newcommand{\GOTO}{\textbf{goto}\ }
\renewcommand{\algorithmicrequire}{\textbf{Input:}}
\renewcommand{\algorithmicensure}{\textbf{Output:}}
\renewcommand{\listalgorithmname}{Algorithms}
\renewcommand{\algorithmiccomment}[1]{\\/* #1 */}

\newcommand{\utitle}{Tag 6}

\begin{document}
\newcommand{\ah}[2]{\ \\* \emph{(#1, #2)}\\}
\newcommand{\power}{\mathrm{power}}


\begin{aufg} Implementiere einige Funktionen um mit quadratischen Matrizen umzugehen:
\begin{enumerate}
\item Eine Funktion, die Speicher für eine quadratische Matrix allokiert, eine um ihn freizugeben, eine um sie auszugeben und eine um sie zur Einheitsmatrix zu initialisieren (das ist die Matrix mit $1$en auf der Hauptdiagonale und $0$en sonst):
\begin{codelisting}
\begin{lstlisting}[numbers=left,numberstyle=\tiny,frame=tlrb]
double **matrix_alloc(int n);
void     matrix_free(double **A, int n);
void     matrix_print(double **A, int n);
double **matrix_id(double **A, int n);
\end{lstlisting}
\end{codelisting}
\item Eine Funktion um eine Matrix zu transponieren (d.h. an der Hauptdiagonale ``zu spiegeln'')
\item Eine Funktion, die zwei solche Matrizen miteinander multipliziert und eine neue Matrix zurück gibt. Für zwei $n \times n$-Matrizen $A = (a_{ij})$ und $B = (b_{ij})$ ist $A \cdot B = C = (c_{ij})$ durch $c_{ij} = \sum_{k=1}^n a_{ik} b_{kj}$ definiert.
\end{enumerate}
\end{aufg}

\begin{aufg}
Implementiere die folgenden Funktionen für Matrizen:
\begin{enumerate}
  \item Multiplikation einer Matrix mit einem Vektor
  \item Den Gauß-Algorithmus, der einer Matrix und einem Vektor einen Lösungsvektor zuordnet. Dazu kannst du annehmen, dass man während des gesamten Ablaufs keine Spaltenvertauschungen durchführen muss und dass es genau eine Lösung für das zugehörige Gleichungssystem geben wird. 
\end{enumerate}
\end{aufg}

\vfill
\begin{center} flip me. \end{center}
\vfill 
\newpage

\begin{aufg}
Diese Aufgabe läuft auf die Implementierung des Merge-Sort Algorithmus hinaus.
\begin{enumerate}
\item Implementiere eine Funktion \verb|merge|, die zwei bereits sortierte (eventuell verschieden große) Arrays als Argumente erhält, diese zu einem sortieren Array kombiniert und dieses zurück liefert. 
\item Die Funktion \verb|mergesort| selbst soll ein Array in zwei (möglichst gleich große) Teilarrays zerlegen, sich für diese Teilarrays selbst aufrufen und danach die dann sortierten Teilarrays mit der \verb|merge|-Funktion kombinieren. Erhält die Funktion ein Array mit keinem oder einem Element so belässt es dieses Array wie es ist, dann ist es nämlich bereits sortiert.
\item Besorge dir die Datei \verb|daten.h|, sortiere das darin definierte Array und schreibe es sortiert in eine Datei.
\end{enumerate}

Hier als Tipp ein Vorschlag für die Signaturen der beiden Funktionen:
\begin{codelisting}
\begin{lstlisting}[numbers=left,numberstyle=\tiny,frame=tlrb]
int *merge(int *list1, int n, int *list2, int m);
void mergesort(int *list, int n);
\end{lstlisting}
\end{codelisting}

\end{aufg}



\end{document}
