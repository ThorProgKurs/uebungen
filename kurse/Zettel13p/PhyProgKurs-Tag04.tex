\documentclass{uebungszettel}
\usepackage{algorithm,algorithmic}
\usepackage{enumitem}
\floatname{algorithm}{Algorithmus}
\newcommand{\SET}{\textbf{set}\ }
\newcommand{\CHOOSE}{\textbf{choose}\ }
\newcommand{\GOTO}{\textbf{goto}\ }
\renewcommand{\algorithmicrequire}{\textbf{Input:}}
\renewcommand{\algorithmicensure}{\textbf{Output:}}
\renewcommand{\listalgorithmname}{Algorithms}
\renewcommand{\algorithmiccomment}[1]{\\/* #1 */}

\newcommand{\utitle}{Tag 4}

\begin{document}
\newcommand{\ah}[2]{\ \\* \emph{(#1, #2)}\\}
\newcommand{\power}{\mathrm{power}}



\begin{aufg} Lege ein Modul \verb|mymath.c| / \verb|mymath.h| an, in dem du die bisher geschriebenen Funktionen auslagerst.
\end{aufg}

 Wir brauchen im folgenden eine Potenzfunktion, die zwei Fließkommazahlen als Argumente akzeptiert. Falls du diese Funktion gestern geschrieben hast, sollte sie jetzt im \verb|mymath|-Modul verfügbar sein. Diese Funktion wird aber vermutlich zu langsam sein, daher gibt es die funktion 

\begin{verbatim}
double pow(double x, double y);
\end{verbatim}

in der Systemheader \verb|<math.h>|. Im Skript findest du im Anhang eine vollständige Referenz einiger Systembibliotheken.


\begin{aufg}
Implementiere die Riemann'sche Zeta-Funktion für $s \in \N$: $$
\zeta(s) := \sum_{k=1}^\infty \frac{1}{k^s} $$
\end{aufg}

\begin{aufg}
Implementiere die Funktion, die den Inhalt zweier \verb|int|-Variablen vertauscht.
\end{aufg}

\begin{aufg}
Schreibe Funktionen \verb|square_to| und \verb|root_to|, die einen \verb|double|-Pointer entgegen nehmen, die dort stehende Variable quadrieren bzw. daraus die Wurzel ziehen und das Ergebnis sowohl zurück geben als auch an die gleiche Speicherstelle schreiben.
\end{aufg}

\begin{aufg}
Erweitere das "`mymath"'-Modul noch um eine Funktion, die zu den drei Koeffizienten $a, b, c \in \R$ einer quadratischen Gleichung $$
a \cdot x^2 + b \cdot x + c = 0 $$
die Lösungen berechnet.

\medskip \noindent \emph{Tip:} Eine Funktion kann nur einen Wert als Rückgabewert haben. Um mehr als einen Wert zurück zu geben, kann man Pointer verwenden.
\end{aufg}
\end{document}
