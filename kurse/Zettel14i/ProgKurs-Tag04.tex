\documentclass{uebungszettel}
\begin{document}

\begin{aufg} Lege ein Modul \verb|mymath.c| / \verb|mymath.h| an, in dem du die bisher geschriebenen Funktionen auslagerst.
\end{aufg}

Wir brauchen im folgenden eine Potenzfunktion, die zwei Fließkommazahlen als Argumente akzeptiert. Falls du diese Funktion gestern geschrieben hast, sollte sie jetzt im \verb|mymath|-Modul verfügbar sein. Diese Funktion wird aber vermutlich zu langsam sein, daher gibt es die Funktion 

\begin{verbatim}
double pow(double x, double y);
\end{verbatim}

in der Systemheader \verb|<math.h>|. Im Skript findest du bei Interesse im Anhang eine Referenz einiger Systembibliotheken.

\begin{aufg}
Implementiere die Riemann'sche Zeta-Funktion für $s \in \R$: $$
\zeta(s) := \sum_{k=1}^\infty \frac{1}{k^s} $$
\end{aufg}

\begin{aufg}
Implementiere die Funktion, die den Inhalt zweier \verb|int|-Variablen vertauscht.
\end{aufg}

\begin{aufg}
\begin{enumerate}[leftmargin=*]
\item Erweitere das "`mymath"'-Modul noch um eine Funktion, die zu den drei Koeffizienten $a, b, c \in \R$ einer quadratischen Gleichung $$
a \cdot x^2 + b \cdot x + c = 0 $$
die größere Lösung berechnet.
\item Erweitere die Funktion zur Lösung einer quadratischen Gleichung: Es soll möglich sein beide Lösungen weiter zu verwenden. 

Eine Funktion kann nur einen Wert als Rückgabewert haben. Um mehr als einen Wert ``zurück'' zu geben, kann man Pointer verwenden.
\end{enumerate}
\end{aufg}

\end{document}
