\documentclass{uebungszettel}
\begin{document}

\begin{aufg}
Berechne die Summe der ersten $n$ ungeraden Zahlen mit einer \texttt{while}-Schleife. Wie ist die Ausgabe für $n = 1, \ldots ,15$. Was fällt dir auf? Könnte man diese Aufgabe nun also effizienter programmieren?
\end{aufg}

\begin{aufg}~
\begin{enumerate}
\item Implementiere den Primzahltest (Algorithmus 1) von gestern.
\item Schreibe ein Programm, dass jeweils die nächste Primzahl nach $20000$, $30000$ und $40000$ findet.
\end{enumerate}
\end{aufg}

\begin{aufg}
Für $a \in \mathbb{R}^+$ konvergiert die Folge $(a_n)$ mit $a_0 = a$ und
\[ a_{n+1} = \frac{1}{2}\left(a_n + \frac{a}{a_n}\right) \]
gegen $\sqrt{a}$. Implementiere damit einen Wurzellalgorithmus.
\end{aufg}

\begin{aufg}
Implementiere den Algorithmus 2 von gestern, welcher den größten gemeinsamen Teiler zweier Zahlen berechnet.
\end{aufg}

\begin{aufg}
Implementiere den Cosinus über seine Reihendarstellung mit einer Schleife. Du kannst die Formel bei Wikipedia nachschlagen, sie selbst entwickeln oder diese hier verwenden:
\[ \cos(x) = \sum_{k = 0}^{\infty}{(-1)^k \cdot \frac{x^{2k}}{(2k)!}} \]
\end{aufg}

\begin{aufg}
Schreibe ein Programm, um den Wert der Reihe
\[ \sum_{k = 1}^\infty \frac{1}{k^2} \]
zu berechnen. Er sollte $\frac{\pi^2}{6}$ sein. Wichtig ist, sich ein geeignetes Abbruchkriterium zu überlegen.
\end{aufg}


\end{document}
