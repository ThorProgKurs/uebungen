\documentclass{uebungszettel}
\begin{document}

\begin{aufg}
Installiere einen Compiler auf deinem Computer und kompiliere ein Hallo-Welt-Programm. Informationen dazu und einen Download-Link für Cygwin findest du im Skript auf dem USB-Stick oder der Webseite des Kurses:
\begin{center}
	\verb|http://www.ah-effect.net/ |
\end{center}
\end{aufg}

\begin{aufg}
Was machen folgende Algorithmen (kein C-Code)?

\begin{algorithm}[H]
\caption{}
\algsetup{indent=1.5em}
\begin{algorithmic}[1]
\REQUIRE Ganze Zahl $c\in\mathbb{N}$
\ENSURE Entweder \verb|Ja| oder \verb|Nein|.
\STATE \SET $n := 2$.
\IF{$n>\sqrt{c}$} \label{1Start}
\RETURN \verb|Ja|
\ENDIF
\IF{$n$ teilt $c$}
\RETURN \verb|Nein|
\ENDIF
\STATE \SET $n := n + 1$
\STATE \GOTO \ref{1Start}
\end{algorithmic}
\end{algorithm}

\begin{algorithm}[H]
\caption{}
\algsetup{indent=1.5em}
\begin{algorithmic}[1]
\REQUIRE Ganze Zahlen $a,b\in\mathbb{N}$
\ENSURE Eine ganze Zahl $k\in\mathbb{N}$
\IF{$a=0$} 
\RETURN $b$
\ENDIF
\IF{$b=0$} \label{3Start}
\RETURN $a$
\ENDIF
\IF{$a>b$}
\STATE \SET $a = a - b$
\ELSE 
\STATE \SET $b = b - a$
\ENDIF
\STATE \GOTO \ref{3Start}
\end{algorithmic}
\end{algorithm}

\begin{algorithm}[H]
\caption{}
\algsetup{indent=1.5em}
\begin{algorithmic}[1]
\REQUIRE Reelle Zahl $a\in\mathbb{R}_{\ge 0}$
\ENSURE Reine reelle Zahl $x\in\mathbb{R}$
\STATE \SET $x := 2$ und $y := 1$.
\IF{$|x-y|\le 10^{-10}$} \label{2Start} 
\RETURN $x$
\ENDIF
\STATE \SET $x := y$
\STATE \SET $y := \frac{1}{2} \cdot \left(x+\frac{a}{x}\right)$
\STATE \GOTO \ref{2Start}
\end{algorithmic}
\end{algorithm}
\end{aufg}

\begin{aufg}
Wettbewerb: Gegeben ist folgender Programmrumpf:
\begin{codelisting}
\begin{lstlisting}[numbers=left,numberstyle=\tiny,frame=tlrb]
#include <stdio.h>
int main(int argc, char **argv) {
	int x = 2;
	/* dein Code hier */
	printf("%i\n", x);
	return 0;
}
\end{lstlisting}
\end{codelisting}
Füge an der markierten Stelle C-Code ein, sodass der Wert von $2^{\left(3^3\right)}$ ausgegeben wird. Wer in diesem Raum am wenigsten Zeichen dafür benötigt bekommt eine Dose Cola\footnote{Wenn er sie sich kauft.}. Erlaubt sind aber nur die Zeichen 
\begin{center}
	x \quad + \quad - \quad * \quad / \quad =
\end{center}
und das Semikolon. Zeilenumbrüche und Leerzeichen können natürlich nach belieben verwendet werden, da sie vom Compiler ignoriert werden.
\end{aufg}

\end{document}
