\documentclass{uebungszettel}
\begin{document}

\begin{aufg}
Erstelle ein Programm \verb|uniq|, dass zwei Dateinamen als Kommandozeilenargumente erhält, die erste Datei zeilenweise als Integer-Variable einliest und ohne doppelte Einträge in die zweite Datei schreibt (in irgendeiner Reihenfolge).
\end{aufg}

\begin{aufg}
Gegeben sei eine Datei, in der ausschließlich Zahlen stehen. In der ersten Zeile stehe eine natürliche Zahl, die angibt, wie viele Zeilen noch folgen. Die noch folgenden Zeilen bestehen ebenfalls nur aus einer Zahl. 
\begin{enumerate}
\item Schreibe ein Programm, dass diese Datei einliest, die Zahlen sortiert und die Datei mit der sortierten Liste überswchreibt.
\item Modifiziere dein Programm nun so, dass in der ersten Zeile nicht mehr stehen muss, wie viele Zeilen noch folgen.\end{enumerate}
\end{aufg}

\begin{aufg}
Schreibe ein Programm, welches eine Datei im folgenden Format ausliest:
\medskip \begin{codelisting}
\begin{lstlisting}[numbers=left,numberstyle=\tiny,frame=tlrb,mathescape=true]
sin(0.4) = 0.389
sin(0.45) = 0.43496
sin(0.6) = 0.5346
\end{lstlisting}
\end{codelisting}
Wenn eine Zeile einen Fehler enthält, der größer als $10^{-3}$ ist, so soll dieser Fehler mit Zeilennummer auf der Konsole ausgegeben werden.
\end{aufg}

\begin{aufg} Erweitere dein Matrizen-Modul um eine Funktion, die Matrizen aus einer Datei lesen kann. Ein Format für diese Dateien darfst du dir selbst ausdenken.
\end{aufg}

\newpage
\begin{aufg}
*Schreibe ein Programm, dass ein Labyrinth aus einer Datei einliest:
{\tt
\lstset{language=Delphi}
\begin{lstlisting}
XXXXXXXXXXXXXXXX
X X XXXXXXXXXX*X
X$X XX     XXX X
X X XX XXX XXX X
X   XX XXX XXX X
XXX X   XX XXX X
XXX   X        X
XXXXXXXXXXXXXXXX
\end{lstlisting}
}
\noindent\emph{Bemerkung: } Wir spezifizieren das Labyrinth hier nicht viel näher, entscheide dich selbst vorher was für ein Format die Datei haben soll und welche Einschränkungen du daran stellst: Soll die Größe des Labyrinths variabel sein oder fest? Soll die Größe in der ersten Zeile der Datei stehen oder nicht? Soll das Labyrinth quadratisch sein oder nicht? Soll es außen herum immer mit $X$en begrenzt sein oder hast du vielleicht eine andere Lösung?\\
Das Programm soll einen Weg vom Startpunkt (dem Stern, dem Geburtsort) zum Dollar (dem Schatz) finden. Die $X$e sind Wände und Leerzeichen sind Pfade. Markiere einen Weg mit Punkten und gebe das Labyrinth mit Weg in der Konsole aus.

{\tt
\lstset{language=Delphi}
\begin{lstlisting}
XXXXXXXXXXXXXXXX
X X XXXXXXXXXX*X
X$X XX     XXX.X
X.X XX XXX XXX.X
X...XX XXX XXX.X
XXX.X...XX XXX.X
XXX...X........X
XXXXXXXXXXXXXXXX
\end{lstlisting}
}
\end{aufg}
\end{document}
