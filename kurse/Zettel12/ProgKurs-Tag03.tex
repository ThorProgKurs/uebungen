\documentclass{uebungszettel}
\usepackage{algorithm,algorithmic}

\floatname{algorithm}{Algorithmus}
\newcommand{\SET}{\textbf{set}\ }
\newcommand{\CHOOSE}{\textbf{choose}\ }
\newcommand{\GOTO}{\textbf{goto}\ }
\renewcommand{\algorithmicrequire}{\textbf{Input:}}
\renewcommand{\algorithmicensure}{\textbf{Output:}}
\renewcommand{\listalgorithmname}{Algorithms}
\renewcommand{\algorithmiccomment}[1]{\\/* #1 */}

\newcommand{\utitle}{Tag 3}

\begin{document}

{\noindent \textbf{Lies mich:} Werde erst mal mit den Aufgaben \textbf{von Tag 2} fertig. Wenn du das Konzept von Schleifen noch nicht verstanden hast, ist ein Fortkommen quasi nicht m�glich. Auf diesem Zettel wird dieses Konzept vertieft und in Zusammenhang mit der heutigen Vorlesung gebracht.}

\begin{aufg}
Lies den Text, vor dem ``Lies mich'' steht.
\end{aufg}


\begin{aufg}
Berechne die Summe aller nat�rlichen Zahlen, die kleiner als $1000$ und Vielfache von $3$ oder $5$ sind.
\end{aufg}

\begin{aufg} ~
\begin{enumerate}
\item Finde den gr��ten Primfaktor der Zahl $1073741821$.
\item Finde die $10001$-te Primzahl.
\end{enumerate}
\end{aufg}

\begin{aufg}
Berechne die kleinste nat�rliche Zahl, die durch alle Zahlen von $1$ bis $20$ teilbar ist.
\end{aufg}

\begin{aufg}
Finde das eindeutige Tripel $(a, b, c)$ mit
\begin{align*}
a^2+b^2&=c^2 &\text{und}&&
a+b+c&=1000.
\end{align*}
\end{aufg}



\end{document}
