\documentclass{uebungszettel}
\usepackage{algorithm,algorithmic}

\floatname{algorithm}{Algorithmus}
\newcommand{\SET}{\textbf{set}\ }
\newcommand{\CHOOSE}{\textbf{choose}\ }
\newcommand{\GOTO}{\textbf{goto}\ }
\renewcommand{\algorithmicrequire}{\textbf{Input:}}
\renewcommand{\algorithmicensure}{\textbf{Output:}}
\renewcommand{\listalgorithmname}{Algorithms}
\renewcommand{\algorithmiccomment}[1]{\\/* #1 */}

\newcommand{\utitle}{Tag 2}

\begin{document}
\newcommand{\ah}[2]{\ \\* \emph{(#1, #2)}\\}

\noindent{\bf Hinweis:} Insgesamt gilt: Wir k�nnen einem kompilierten Programm noch keine Eingabe �ber die Kommandozeile geben. Um eure Programme mit verschiedenen Werten zu testen, ver�ndert Ihr vorerst einfach die Konstanten im Quellcode.


\begin{aufg} ~
\begin{enumerate}
\item Implementiere den Primzahltest (Algorithmus 1) von gestern.
\item Schreibe ein Programm, dass jeweils die n�chste Primzahl nach $20000$, $30000$ und $40000$ findet.
\end{enumerate}
\end{aufg}

\begin{aufg}
F�r $a \in \mathbb{R}^+$ konvergiert die Folge $(a_n)$ mit $a_0 = a$ und
\[ a_{n+1} = \frac{1}{2}\left(a_n + \frac{a}{a_n}\right) \]
gegen $\sqrt{a}$. Implementiere damit einen Wurzelalgorithmus.
\end{aufg}

\begin{aufg}
Implementiere den Algorithmus 2 von gestern, welcher den gr��ten gemeinsamen Teiler zweier Zahlen berechnet.
\end{aufg}

\begin{aufg}
Implementiere den Cosinus �ber seine Reihendarstellung mit einer Schleife. Du kannst die Formel bei Wikipedia nachschlagen, sie selbst entwickeln oder diese hier verwenden:
\[ \cos(x) = \sum_{k = 0}^{\infty}{(-1)^k \cdot \frac{x^{2k}}{(2k)!}} \]
\end{aufg}

\begin{aufg}
Schreibe ein Programm, um den Wert der Reihe
\[ \sum_{k = 1}^\infty \frac{1}{k^2} \]
zu berechnen. Er sollte $\frac{\pi^2}{6}$ sein. Wichtig ist, sich ein geeignetes Abbruchkriterium zu �berlegen.
\end{aufg}


\end{document}
