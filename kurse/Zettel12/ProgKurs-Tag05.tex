\documentclass{uebungszettel}
\usepackage{algorithm,algorithmic}

\floatname{algorithm}{Algorithmus}
\newcommand{\SET}{\textbf{set}\ }
\newcommand{\CHOOSE}{\textbf{choose}\ }
\newcommand{\GOTO}{\textbf{goto}\ }
\renewcommand{\algorithmicrequire}{\textbf{Input:}}
\renewcommand{\algorithmicensure}{\textbf{Output:}}
\renewcommand{\listalgorithmname}{Algorithms}
\renewcommand{\algorithmiccomment}[1]{\\/* #1 */}

\newcommand{\utitle}{Tag 5}

\begin{document}
\newcommand{\ah}[2]{\ \\* \emph{(#1, #2)}\\}


\begin{aufg} Werde mit dem letzten Zettel fertig und lege ein Modul mit dem Namen \verb|mymath.c| an, in dem du die bisher geschriebenen Funktionen auslagerst. Die Header-Datei \verb|mymath.h| sollte (mindestens) wie folgt aussehen:

\medskip\begin{codelisting}
\begin{lstlisting}[numbers=left,numberstyle=\tiny,frame=tlrb]
#ifndef _MYMATH__H
#define _MYMATH__H

double logarithm(double);
double expo(double);
double power(double,double);

/* Ganzzahlige Potenzen mit Double&Add ausrechnen: */
double int_power(double,int);

double wurzel(double);
double betrag(double);

#endif
\end{lstlisting}
\end{codelisting}
\end{aufg}

\begin{aufg} Eine \verb|double|-Variable kann den symbolischen Wert $\infty$ (unendlich) haben. Es gibt auch $-\infty$. F�r diesen Wert und $a\in\R$ gelten folgende Regeln:
\begin{align*}
a + \infty &= \infty &
a \cdot \infty &= \infty &
\infty+\infty &= \infty
\end{align*}
Der Wert $\infty-\infty$ ergibt den weiteren symbolischen Wert $\mathrm{NaN}$ (Not a Number). Wenn das Ergebnis einer Rechnung mit nicht-symbolischen \verb|double|-Werten den Wertebereich verl�sst, so ist das Ergebnis $\pm\infty$. Schreibe eine Funktion, die den Wert $\infty$ berechnet und zur�ck gibt.
\end{aufg}

\newpage

\begin{aufg} Wir brauchen im folgenden eine Potenzfunktion, die zwei Flie�kommazahlen als Argumente akzeptiert. Falls du diese Funktion gestern geschrieben hast, sollte sie jetzt im \verb|mymath|-Modul verf�gbar sein. Andernfalls gibt es die funktion 

\begin{verbatim}
double pow(double x, double y);
\end{verbatim}

in der Systemheader \verb|<math.h>|. Im Skript findest du im Anhang  Referenzen einiger Systembibliotheken.
Implementiere die Riemann'sche Zeta-Funktion f�r $s > 1$: 
$$\zeta(s) := \sum_{k=1}^\infty \frac{1}{k^s} $$
F�r $s\le 1$ sollte die Funktion den Wert $\infty$ zur�ckliefern, siehe Aufgabe 2. 

Erweitere dein \verb|mymath|-Modul um die Zeta-Funktion und
schreibe ein Programm, das deren Werte f�r einige $s\in]1,3[$ ausgibt.

\medskip \textbf{Hinweis:} Unter Linux muss man dem gcc noch den Linker-Befehl \verb|-lm| �bergeben, um den Linker anzuweisen, die \verb|math|-Bibliothek mit zu verlinken. Beispiel:
\begin{center}
\verb|gcc -Wall -pedantic -o zeta zeta.c mymath.c -lm|
\end{center}
Der \verb|-lm| Parameter muss als letztes �bergeben werden, da es sich um einen Linker-Befehl handelt.

\end{aufg}

\begin{aufg}
Erweitere das mymath-Modul noch um eine Funktion, die zu den drei Koeffizienten $a, b, c \in \R$ einer quadratischen Gleichung $$
a \cdot X^2 + b \cdot X + c = 0 $$
die L�sungen berechnet und die gr��ere L�sung zur�ck gibt.
\end{aufg}


\end{document}
