\documentclass{uebungszettel}
\usepackage{algorithm,algorithmic}

\floatname{algorithm}{Algorithmus}
\newcommand{\SET}{\textbf{set}\ }
\newcommand{\CHOOSE}{\textbf{choose}\ }
\newcommand{\GOTO}{\textbf{goto}\ }
\renewcommand{\algorithmicrequire}{\textbf{Input:}}
\renewcommand{\algorithmicensure}{\textbf{Output:}}
\renewcommand{\listalgorithmname}{Algorithms}
\renewcommand{\algorithmiccomment}[1]{\\/* #1 */}

\newcommand{\utitle}{Tag 4}

\begin{document}
\newcommand{\ah}[2]{\ \\* \emph{(#1, #2)}\\}


\begin{aufg}
Implementiere den Absolutbetrag \verb|betrag(x)|, die Wurzelfunktion \verb|wurzel(x)| (mit dem Heron-Verfahren) und den Cosinus \verb|cos(x)| als Funktionen.
\end{aufg}

\begin{aufg}~
\begin{enumerate}
\item Implementiere f�r $x \in \R$ und $n \in \N$ eine Funktion \verb|power(x,n)|, die den Wert $x^n$ berechnet. Wir verwenden dazu die Double-and-Add-Methode, d.h. die rekursive Formel \[
	\mathrm{power}(x, n) = \left\{ \begin{array}{ll}
	1 & \text{wenn } n = 0 \\
	x \cdot \mathrm{power}\left(x^2, \frac{n-1}{2}\right) & \text{wenn } n \text{ ungerade} \\
	\mathrm{power}\left(x^2, \frac{n}{2}\right) & \text{wenn } n \text{ gerade} \\
	\end{array}
	\right.
\]
\item Implementiere eine Potenzfunktion \verb|naiv_power(x, n)|, indem du eine Schleife von $1$ bis $n$ laufen l�sst und bei jedem Durchlauf eine mit $1$ Initialisierte Variable mit $x$ multipliziert. Berechne $0,9999999999^{2000000000}$ einmal mit \verb|power(x, n)| aus a) und einmal mit \verb|naiv_power(x, n)| (es sollte ca. $0,818731$ raus kommen). 
\item* Implementiere die Double-and-Add-Methode iterativ, also ohne rekursiven Aufruf.
\item* Vergleiche die Laufzeiten der 3 Funktionen. Man kann wie folgt die Laufzeit messen:
\begin{codelisting}
\begin{lstlisting}[numbers=left,numberstyle=\tiny,frame=tlrb]
#include <stdio.h>
#include <time.h>
int main() {
	double zeit;
	/* Eigene Variablendeklarationen */
	zeit = clock();

	/* Programmcoder hier */

	/* Zeitdifferenz ausrechnen: */
	zeit = (clock()-zeit) / CLOCKS_PER_SEC;
	/* Zeitdifferenz ausgeben: */
	printf("%f\n",zeit);
	return 0;
}
\end{lstlisting}
\end{codelisting}
\end{enumerate}
\end{aufg}


\newpage

\begin{aufg}
Diese Aufgabe wird auf eine \verb|power(x, y)|-Funktion f�hren, die f�r beliebige $x \in \R^+$ und $y \in \R$ den Wert von $x^y$ berechnet.
\begin{itemize}
\item Implementiere die Exponentialfunktion \verb|expo(x)|, die $e^x$ mit Hilfe folgender Reihendarstellung berechnet: \[
e^x = \sum_{k=0}^\infty \frac{x^k}{k!} \]
\item Implementiere eine Logarithmus-Funktion \verb|logarithm(x)|, die $\ln(x)$ mit Hilfe folgender Reihendarstellung berechnet: \[
\ln(x) = 2\cdot \sum_{k = 0}^\infty \left(\frac{x-1}{x+1}\right)^{2k + 1} \frac{1}{2k + 1} \]
{\bf Hinweis:} Diese Approximation des Logarithmus ist um den Punkt $x=1$ herum sehr genau, f�r gr��ere Werte allerdings nicht. Man kann sich zu nutze machen, dass $\ln(x)=2r\ln(\sqrt{2})+\ln(2^{-r}x)$ gilt.
\item Verwende die Formel \[
x^y = e^{y \cdot \ln(x)} \] um \verb|power(x, y)| zu bestimmen.
\end{itemize}
\end{aufg}


\end{document}
