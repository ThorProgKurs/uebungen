\documentclass{uebungszettel}
\usepackage{algorithm,algorithmic}

\floatname{algorithm}{Algorithmus}
\newcommand{\SET}{\textbf{set}\ }
\newcommand{\CHOOSE}{\textbf{choose}\ }
\newcommand{\GOTO}{\textbf{goto}\ }
\renewcommand{\algorithmicrequire}{\textbf{Input:}}
\renewcommand{\algorithmicensure}{\textbf{Output:}}
\renewcommand{\listalgorithmname}{Algorithms}
\renewcommand{\algorithmiccomment}[1]{\\/* #1 */}

\newcommand{\utitle}{Tag 8}

\begin{document}
\newcommand{\ah}[2]{\ \\* \emph{(#1, #2)}\\}

\begin{aufg}
Gegeben sei eine Datei, in der ausschlie�lich Zahlen stehen. In der ersten Zeile stehe eine nat�rliche Zahl, die angibt wie viele Zahlen noch folgen. 
\begin{enumerate}
  \item Schreibe ein Programm, dass diese Datei einliest, die Zahlen sortiert und die Datei mit der sortierten Liste �berschreibt. 
  \item Modifiziere dein Programm nun so, dass in der ersten Zeile nicht mehr stehen muss, wie viele Zeilen noch folgen.
\end{enumerate}
\end{aufg}


\begin{aufg}
Erstelle ein Programm \verb|uniq|, dass zwei Dateinamen als Argumente erh�lt, die erste Datei zweilenweise als Integer-Variable einliest und ohne doppelte Eintr�ge in die zweite Datei schreibt (in irgendeiner Reihenfolge).
\end{aufg}



\begin{aufg}
Implementiere folgende Funktion, die zu einem gegebenen String einen l�ngsten Teilstring findet, der ein Palindrom\footnote{Ein Palindrom ist ein String, der r�ckw�rts gelesen das gleiche ergibt wie vorw�rts. Ein Beispiel w�re ``anna''.} ist. Speichere diesen Teilstring wieder in s und gib s zur�ck.

\medskip \begin{codelisting}
\begin{lstlisting}[numbers=left,numberstyle=\tiny,frame=tlrb]
/* Beschreibung, selber machen */
char *str_glsp(char *s);
\end{lstlisting}
\end{codelisting}

\medskip Das Problem ist in polynomieller Laufzeit l�sbar und auf die Idee kann man auch kommen. 
\end{aufg}

\newpage

\begin{aufg} Implementiere die folgenden Funktionen:

\medskip\begin{codelisting}
\begin{lstlisting}[numbers=left,numberstyle=\tiny,frame=tlrb]
/* pieces ist ein Array der L�nge len (von Strings).
   H�nge diese Strings, getrennt durch sep, aneinander 
   und liefere das Ergebnis. Beispiel:

   char *p[3] = { "Ene", "Mene", "Miste" };
   char *joined = str_join( p, 3, ", ");
   printf("%s", joined); 
   free(joined);

   gibt aus: Ene, Mene, Miste */
char  *str_join(char **pieces, int len, char *sep);

/* Spalte str bei jedem Auftreten von sep auf und 
   liefere das Array der entstehenden Strings.
   W�re etwa str der String "Ene, Mene, Miste" und
   sep der String "," so sollte das zur�ckgegebene
   Array die Strings

     "Ene"
     " Mene"
     " Miste"
   
  enthalten (Man bemerke die Leerzeichen). In len
  wird die L�nge des Arrays gespeichert, in diesem
  Fall 3. Der String str selbst sollte von der 
  Funktion nat�rlich nicht ver�ndert werden. */
char **str_split(char *str, char *sep, int *len);
\end{lstlisting}
\end{codelisting}
\end{aufg}



\end{document}
