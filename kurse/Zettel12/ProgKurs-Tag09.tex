\documentclass{uebungszettel}
\usepackage{algorithm,algorithmic}

\floatname{algorithm}{Algorithmus}
\newcommand{\SET}{\textbf{set}\ }
\newcommand{\CHOOSE}{\textbf{choose}\ }
\newcommand{\GOTO}{\textbf{goto}\ }
\renewcommand{\algorithmicrequire}{\textbf{Input:}}
\renewcommand{\algorithmicensure}{\textbf{Output:}}
\renewcommand{\listalgorithmname}{Algorithms}
\renewcommand{\algorithmiccomment}[1]{\\/* #1 */}

\newcommand{\utitle}{Tag 9}

\begin{document}
\newcommand{\ah}[2]{\ \\* \emph{(#1, #2)}\\}



\begin{aufg}
Schreibe ein Programm, dass ein Labyrinth aus einer Datei einliest:
\begin{center}\begin{minipage}{1.65in}{\tt
\lstset{language=Delphi}
\begin{lstlisting}
XXXXXXXXXXXXXXXX
X X XXXXXXXXXX*X
X$X XX     XXX X
X X XX XXX XXX X
X   XX XXX XXX X
XXX X   XX XXX X
XXX   X        X
XXXXXXXXXXXXXXXX
\end{lstlisting}
}\end{minipage}\end{center}
\emph{Bemerkung: } Wir spezifizieren das Labyrinth hier nicht viel n�her, entscheide dich selbst vorher f�r ein Format. Soll die Gr��e des Labyrinths variabel sein oder fest? Soll die Gr��e in der ersten Zeile der Datei stehen oder nicht? Soll das Labyrinth quadratisch sein oder nicht? Soll es au�en herum immer mit $X$en begrenzt sein oder hast du vielleicht eine andere L�sung?

\vspace{1.5ex} Das Programm soll einen Weg vom Startpunkt (dem Stern) zum Schatz (dem Dollarzeichen) finden. Die $X$e sind W�nde und Leerzeichen sind Pfade. Markiere einen Weg mit Punkten und gebe das Labyrinth mit Weg in der Konsole aus.

\begin{center}\begin{minipage}{1.65in}{\tt
\lstset{language=Delphi}
\begin{lstlisting}
XXXXXXXXXXXXXXXX
X X XXXXXXXXXX*X
X$X XX     XXX.X
X.X XX XXX XXX.X
X...XX XXX XXX.X
XXX.X...XX XXX.X
XXX...X........X
XXXXXXXXXXXXXXXX
\end{lstlisting}
}\end{minipage}\end{center}
\end{aufg}

\begin{aufg} Implementiere ein Modul, das Rechenoperationen f�r ganzzahlige Br�che bereitstellt:

\medskip \begin{codelisting}
\begin{lstlisting}[numbers=left,numberstyle=\tiny,frame=tlrb]
typedef struct {
      signed long int numerator;
    unsigned long int denominator;
} RATIONAL;
\end{lstlisting}
\end{codelisting}
Es sollte Funktionen zum addieren, subtrahieren und multiplizieren von Br�chen geben. Das Ergebnis einer Rechnung sollte immer vollst�ndig gek�rzt sein.
\end{aufg}

\begin{aufg} Lies im Skript den Teil 7.5 �ber verkettete Listen. 
Implementiere doppelt verkettete Listen, die anstatt einer \verb|double|-Variable beliebige Daten speichern k�nnen, als \verb|void*|. Die Header-Datei k�nnte etwa wie folgt aussehen:

\medskip \begin{codelisting}
\begin{lstlisting}[numbers=left,numberstyle=\tiny,frame=tlrb]
#ifndef _LIST__H
#define _LIST__H

LIST *list_create(); /* Leere Liste erstellen */

/* Element hinter E einf�gen, NULL hei�t am Anfang */
LISTNODE *list_insert(LIST *L, LISTNODE *E, void *p);

/* Element am Anfang bzw. Ende einf�gen */
LISTNODE *list_unshift(LIST *L, void *p);
LISTNODE *list_push(LIST *L, void *p);

/* Element am Anfang bzw. Ende entfernen und 
   die Daten zur�ck geben */
void *list_shift(LIST *L);
void *list_pop(LIST *L);

/* eine Element aus der Liste entfernen */
void list_delete(LIST *L, LISTNODE *E);

/* Die Elemente aus Liste M an Liste L anh�ngen und
   L zur�ckgeben. Nach Aufruf ist M leer. */
LIST *list_merge(LIST *L, LIST *M);

/* Liste inklusive allen Elementen frei geben */
void list_free(LIST *L);

#endif
\end{lstlisting}
\end{codelisting}
\end{aufg}


\end{document}
