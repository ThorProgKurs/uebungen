\documentclass{uebungszettel}
\begin{document}

\begin{aufg}
Implementiere die Funktion, die den Inhalt zweier \verb|int|-Variablen vertauscht.
\end{aufg}

\begin{aufg}
Schreibe Funktionen \verb|square_to| und \verb|root_to|, die einen \verb|double|-Pointer entgegen nehmen, die dort 
stehende Variable quadrieren bzw. daraus die Wurzel ziehen und das Ergebnis sowohl zurück geben als auch an die gleiche 
Speicherstelle schreiben.
\end{aufg}

\begin{aufg}
Erweitere das "`mymath"'-Modul noch um eine Funktion, die zu den drei Koeffizienten $a, b, c \in \R$ einer quadratischen
 Gleichung $$
a \cdot x^2 + b \cdot x + c = 0 $$
die Lösungen berechnet.
\end{aufg}

\begin{aufg}
Schreibe ein Modul \verb|arrayhelpers|, das einige nützliche Funktion zum Umgang mit statischen \verb|int|-Arrays enthält:
\begin{enumerate}
\item Array zeilenweise oder mit Kommata getrennt ausgeben
\item Alle Felder eines Arrays mit einem Wert initialisieren
\item Array um $1$ rotieren (d.h. das hinterste Element an erste Stelle schreiben und alle anderen Elemente um eins nach hinten schieben)
\item Array um $k$ rotieren 
\item Array umdrehen
\item Ein Array in einem anderen suchen und die Position zurück geben. Sollte das Array nicht im anderen enthalten sein, so soll der Rückgabewert $-1$ sein.

\emph{Beispiel:} 
\begin{codelisting}
\begin{lstlisting}[numbers=left,numberstyle=\tiny,frame=tlrb]
int A[10] = {1, 2, 3, 4, 5, 6, 7, 8, 9, 10};
int B[3] = {4, 5, 6}
int C[2] = {5, 7}
int D[2] = {9, 10}
\end{lstlisting}
\end{codelisting}
Hier gilt: \verb|B| ist an $3$-ter Stelle in \verb|A| enthalten und \verb|D| an $8$-ter. Das Array \verb|C| ist garnicht in \verb|A| enthalten, darum wird der Rückgabewert $-1$ sein.
\end{enumerate}
\end{aufg}

\begin{aufg} In dieser Aufgabe geht es um Sortieralgorithmen. Definiere dir ein Test-Array mit einer \emph{festen} 
Anzahl von Einträgen mit denen du den Algorithmus testest.
\begin{enumerate} 
\item Implementiere folgenden Sortieralgorithmus: Sortiere das kleines Element an die erste Stelle, dann das 
zweitekleinste Element an die zweite Stelle usw. Dieser Algorithmus ist in vielen Fällen relativ langsam (aber sehr 
schnell bei kurzen Listen) -- man spricht auch von einer Komplexität von $\mathcal O(n^2)$ (wobei $n$ die Anzahl der 
Elemente ist). 
\item Aus theoretischer Sicht sind Sortieralgorithmen bis zu $\mathcal O(n \log(n))$ realisierbar (von denen wir in den 
nächsten Tagen auch noch einige kennen lernen werden). Wenn man nun aber die Wertemenge der zu sortierenden Eintrag 
einschränkt (z.B. sei die größte zu sortierende Zahl $20000$) ist es sogar möglich einen \emph{linearen} 
Sortieralgorithmus zu implementieren, also $\mathcal O(n)$. Dazu stellt man sich für jede Zahl einen leeren ``Bucket'' 
(Korb) vor. Dann geht man die Liste der zu sortierenden Einträge durch und für ein Verkommen der Zahl $k$ einen Ball in 
den $k$-ten Bucket. Danach hat man alle Zahlen aus der Liste statt dessen in die Buckets sortiert und kann die 
ursprüngliche Liste überschreiben. Diese machen wir auf folgende Weise: Gehe die Buckets vom ersten bis zum letzten 
durch. Liegen im $k$-ten Bucket $j$ Bälle, schreibe $j$ mal die Zahl $k$ hintereinander in die Liste. Da die Bälle 
genau den zu sortierenden Zahlen entsprechen ist die Liste nachher sortiert. Diese Algorithmus heißt ``Bucket-Sort''. 

Implementiere den Bucket-Sort-Algorithmus für \verb|int|-Arrays mit Werten zwischen $0$ und $20000$.
\end{enumerate}
\end{aufg}



\end{document}
