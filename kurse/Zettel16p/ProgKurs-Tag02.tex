\documentclass{uebungszettel}
\begin{document}

\begin{aufg}
Das folgende Programm sollte die Summe der ersten $n$
Zahlen berechnen.
Allerdings enth{\"a}lt es $5$ Fehler.
Finde sie alle! 

\begin{codelisting}
\begin{lstlisting}[numbers=left,numberstyle=\tiny,frame=tlrb,showstringspaces=false]
/* Summe der ersten n Zahlen.
 * (c) 2015 Clelia und Johannes */

#include <stdio.h>

int main () {
    int n = 10;      /* Addiere bis zu dieser Zahl */
    int i;
    int summe;       /* speichert Zwischenergebnis */   
    
    i = 0;
    
    while (i < n) {
      summe =+ i     /* addiere ite Zahl auf summe */
    }    
    printf ("Das Ergebnis ist %f.\n", summe);
    return 0;
}
\end{lstlisting}
\end{codelisting}
Was fällt dir auf, wenn du das Programm nach Korrektur ausführst? Könnte man diese Aufgabe nun also effizienter implementieren?
\end{aufg}

\begin{aufg}~
\begin{enumerate}
\item Implementiere den Primzahltest (Algorithmus 1) von gestern.
\item Schreibe ein Programm, dass jeweils die nächste Primzahl nach $20000$, $30000$ und $40000$ findet.
\end{enumerate}
\end{aufg}

\begin{aufg}
Für $a \in \mathbb{R}^+$ konvergiert die Folge $(a_n)$ mit $a_0 = a$ und
\[ a_{n+1} = \frac{1}{2}\left(a_n + \frac{a}{a_n}\right) \]
gegen $\sqrt{a}$. Implementiere damit einen Wurzellalgorithmus.
\end{aufg}

\begin{aufg}
Implementiere den Algorithmus 2 von gestern, welcher den größten gemeinsamen Teiler zweier Zahlen berechnet.
\end{aufg}

\begin{aufg}
Implementiere den Cosinus über seine Reihendarstellung mit einer for-Schleife. Du kannst die Formel bei Wikipedia nachschlagen, sie selbst entwickeln oder diese hier verwenden:
\[ \cos(x) = \sum_{k = 0}^{\infty}{(-1)^k \cdot \frac{x^{2k}}{(2k)!}} \]
\end{aufg}

\begin{aufg}
Schreibe ein Programm, um den Wert der Reihe
\[ \sum_{k = 1}^\infty \frac{1}{k^2} \]
zu berechnen. Er sollte $\frac{\pi^2}{6}$ sein. Wichtig ist, sich ein geeignetes Abbruchkriterium zu überlegen.
\end{aufg}

\end{document}
