\documentclass{uebungszettel}
\newcommand{\utitle}{Tag 5}

\begin{document}

\begin{aufg} Durch Pointer k�nnen nun Funktionen auf Variablen einer anderen Funktion zugreifen. Implementiere eine Funktion \verb|swap|, die den Inhalt zweier \verb|int|-Variablen vertauscht.
\end{aufg}

\begin{aufg}
Implementiere ein Programm, dass ein \verb|int|-Array sortiert. Die naheliegenste M�glichkeit besteht wohl darin, zuerst das kleinste Element an die erste Stelle zu tauschen, dann das kleinste unter den Verbleibenden an die zweite Stelle zu tauschen, usw.
\end{aufg}

\begin{aufg}
Erweitere das "`mymath"'-Modul noch um eine Funktion, die zu den drei Koeffizienten $a, b, c \in \R$ einer quadratischen Gleichung
\[ a \cdot x^2 + b \cdot x + c = 0 \]
die L�sungen berechnet. Sofern die Gleichung zwei L�sungen hat, sollten danach beide zur Verf�gung stehen. Dies kann realisiert werden, indem man der Funktion auch einen Pointer �bergibt, an dessen Adresse der Wert geschrieben wird.
\end{aufg}


\begin{aufg}
Schreibe Funktionen \verb|square_to| und \verb|root_to|, die einen \verb|double|-Pointer entgegen nehmen, die dort stehende Variable quadrieren bzw. daraus die Wurzel ziehen und das Ergebnis sowohl zur�ck geben als auch an die gleiche Speicherstelle schreiben.
\end{aufg}

\end{document}
