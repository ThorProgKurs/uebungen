\documentclass{uebungszettel}
\newcommand{\utitle}{Tag 8}

\begin{document}

\begin{aufg} Lese noch einmal im Skript die Sektion 7.5 und implementiere doppelt verkettete Listen, die \verb|double|-Variablen speichern.

\begin{codelisting}
\begin{lstlisting}[numbers=left,numberstyle=\tiny,frame=tlrb]
/* Definiere hier angemessene Strukturen f�r einen
   einzelnen Listeneintrag und die Liste selbst. */

/* Leere Liste erstellen */
LIST *list_create();

/* Element hinter E einf�gen, NULL hei�t am Anfang */
LISTNODE *list_insert(LIST *L, LISTNODE *E, double p);

/* Element am Anfang bzw. Ende einf�gen */
LISTNODE *list_unshift(LIST *L, double p);
LISTNODE *list_push(LIST *L, double p);

/* Element am Anfang bzw. Ende entfernen und 
   die Daten zur�ck geben */
double list_shift(LIST *L);
double list_pop(LIST *L);

/* eine Element aus der Liste entfernen */
void list_delete(LIST *L, LISTNODE *E);

/* zwei Listen zusammenf�gen */
LIST *list_merge(LIST *L, LIST *M);

/* Liste inklusive allen Elementen frei geben */
void list_free(LIST *L);

\end{lstlisting}
\end{codelisting}
\end{aufg}

\vfil 
\begin{center}
{\bf FLIP ME}
\end{center}
\vfil


\newpage

\begin{aufg}
Implementiere "`dynamische Arrays"' von \verb|double|-Variablen. Also Funktionen, die es leicht erm�glichen mit dynamisch gro�en Arrays zu arbeiten. Verwende eine Header-Datei in folgendem Stil:
\begin{codelisting}
\begin{lstlisting}[numbers=left,numberstyle=\tiny,frame=tlrb]
typedef struct {
  double *data; /* eigentliches Array */
  int length;   /* erstes nicht verwendetes Element */
  int _size;    /* Menge der allokierten Elemente */
} DBLARRAY;

/* initialisiert eine Array-Datenstruktur. Liefert 1
   bei Erfolg und andernfalls 0. */
int dblarray_init(DBLARRAY *, int initial_size);

/* gibt eine Array-Datenstruktur wieder frei */
void dblarray_free(DBLARRAY *);

/* Setzt den Wert an der Stelle i auf val. 
   Falls n�tig, wird neuer Speicher allokiert und 
   alle Elemente bis dorthin mit 0 initialisiert */
int dblarray_set(DBLARRAY *, int i, double val);

/* gib den Wert an der Stelle i zur�ck */
double dblarray_get(DBLARRAY *, int i);

/* setze das erste nicht initialisierte Element auf 
   val. Sollte es noch nicht existieren, wird neuer 
   Speicher allokiert */
int dblarray_push(DBLARRAY *, double val);

/* Liefere das letzte initialisierte Element und
   verringere die L�nge des Arrays um 1. */
double dblarray_pop(DBLARRAY *);

\end{lstlisting}
\end{codelisting}
\end{aufg}

\begin{aufg} Schreibe Funktionen, die aus einer verketteten Liste ein \verb|DBLARRAY| und umgekehrt machen.
\end{aufg}


\end{document}
