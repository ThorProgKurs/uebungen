\documentclass{uebungszettel}
\newcommand{\utitle}{Tag 7}

\begin{document}
\begin{aufg} Implementiere einige Funktionen um mit quadratischen Matrizen umzugehen:
\begin{enumerate}
\item Eine Funktion, die Speicher f�r eine quadratische Matrix allokiert, eine um ihn freizugeben, eine um sie auszugeben, eine und sie aus einer Datei einzulesen und eine, um sie zur Einheitsmatrix zu initialisieren (das ist die Matrix mit $1$en auf der Hauptdiagonalen und $0$en sonst):
\begin{codelisting}
\begin{lstlisting}[numbers=left,numberstyle=\tiny,frame=tlrb]
#include <stdio.h>

double **matrix_alloc(int n);
void     matrix_free(double **A, int n);
void     matrix_print(double **A, int n);
double **matrix_read(FILE *fp, int *n);
double **matrix_id(double **A, int n);
\end{lstlisting}
\end{codelisting}
\item Eine Funktion, die zwei solche Matrizen miteinander multipliziert und eine neue Matrix zur�ck gibt. F�r zwei $n \times n$-Matrizen $A = (a_{ij})$ und $B = (b_{ij})$ ist $A \cdot B = C = (c_{ij})$ �ber folgende Formel definiert: \[
c_{ij} = \sum_{k=1}^n a_{ik} b_{kj}
\]
\item Eine Funktion, die die Spur einer Matrix ausrechnet: Das ist die Summe der Eintr�ge auf der Hauptdiagonalen.
\item Auf der Webseite des Kurses gibt es zwei Dateien mit Matrizen. Lese die beiden Matrizen ein, multipliziere sie, und bestimme die Spur des Ergebnisses. Wenn dein Ergebnis stimmt, bekommt du eine Dose Cola, wenn wir beeindruckt sind.
\end{enumerate}
\end{aufg}

\begin{aufg}
Gegeben sei eine Datei, in der ausschlie�lich Zahlen stehen. In der ersten Zeile stehe eine nat�rliche Zahl, die angibt wie viele Zahlen noch folgen. 
\begin{enumerate}
  \item Schreibe ein Programm, dass diese Datei einliest, die Zahlen sortiert und die Datei mit der sortierten Liste �berschreibt. 
  \item Modifiziere dein Programm nun so, dass in der ersten Zeile nicht mehr stehen muss, wie viele Zeilen noch folgen.
\end{enumerate}
\end{aufg}

\end{document}
