\documentclass{uebungszettel}
\newcommand{\utitle}{Tag 4}

\begin{document}

\begin{aufg} \ 
\begin{enumerate}
\item Implementiere eine Funktion \verb|presentation(n, r)| in einem eigenen Modul, die zu einer nat�rlichen Zahl (im Zehnersystem) $n$ und einer Basis $0 < r < 10$ mit nachfolgendem Algorithmus die Darstellung von $n$ in der Basis $r$ ausgibt: Dividiere durch r, merke dir den Rest, den die Zahl bei der Division l�sst. Verfahre mit dem Ergebniss der Division (abgerundet) so weiter. Die Reste geben die Zahl in der neuen Basis. Hier ein Beispiel, wie der Algorithmus f�r die 99 in der Basis 8 aussehen soll: 
\begin{eqnarray*}
99 & = & 12 \cdot 8 + 3 \\
12 & = & 1 \cdot 8 + 4 \\
1 & = & 0 \cdot 8 +1
\end{eqnarray*}
Die Darstellung von 99 zur Basis 8 ist also 341 (oder "`richtig"' 143). Lagere die Funktion in ein eigenes Modul aus!
\item Was macht dieser Programmschnipsel? Warum? \\
\begin{codelisting}
\begin{lstlisting}[numbers=left,numberstyle=\tiny,frame=tlrb]
for(; n != 0; n /= r) {
	printf("%i\n",n%r);
}
\end{lstlisting}
\end{codelisting}
\item{*} Implementiere \verb|presentation2(n, r)| rekursiv s.d. die Ziffern in richtiger (umgekehrter) Reihenfolge ausgegeben werden. Implementiere diese Funktion im gleichen Modul wie \verb|presentation(n, r)|!
\end{enumerate}
\end{aufg}


\begin{aufg} Lege ein Modul \verb|mymath.c| / \verb|mymath.h| an, in dem du die bisher geschriebenen Funktionen auslagerst. Erweitere das Modul um die Funktion \verb|sgn(x)| und den Absolutbetrag \verb|betrag(x)|.
\end{aufg}

Wir brauchen im folgenden eine Potenzfunktion, die zwei Flie�kommazahlen als Argumente akzeptiert. Falls du diese Funktion gestern geschrieben hast, sollte sie jetzt im \verb|mymath|-Modul verf�gbar sein. Andernfalls gibt es die funktion 

\begin{verbatim}
double pow(double x, double y);
\end{verbatim}

in der Systemheader \verb|<math.h>|. Im Skript findest du im Anhang eine vollst�ndige Referenz einiger Systembibliotheken.

\begin{aufg}
Implementiere die Riemann-Zeta-Funktion f�r $s\in\mathbb{R}$ und $s>1$: 
\[ \zeta(s) := \sum_{k=1}^\infty \frac{1}{k^s} \]
und f�ge sie zu deinem \verb|mymath|-Modul hinzu.

\noindent\emph{Warnung}: Diese Reihe konvergiert nicht f�r $s \le 1$.
\end{aufg}

\begin{aufg} \ 
\begin{enumerate}
\item Vielleicht kennt ihr das Cantor'sche Diagonalverfahren, man verwendet es beispielsweise um zu zeigen, dass $\N$ und $\Q$ gleichm�chtig sind. Eine Funktion $\pi: \N^2 \rightarrow \N$, die einem P�rchen nat�rlicher Zahlen ihre Nummer auf dem "`Weg"' durch das Diagonalschema zuordnet ist gegeben durch: \[
	\pi(x, y) = \frac{(x+y) \cdot (x+y+1)}{2} + y \]
Implementiere die Funktion $\pi$ in einem eigenen Modul "`CantorDiag"'.
\item Es l�sst sich zeigen, dass diese Funktion invertierbar ist. Die Werte der Inversen $\pi^{-1}: \N \rightarrow \N^2$ lassen sich mit folgender Methode berechnen: Zu einem $z = \pi(x, y)$ erh�lt man $x$ und $y$ zur�ck mit \[
w = \left\lfloor \frac{\sqrt{8\cdot z + 1}-1}{2} \right\rfloor;
t = \frac{w^2 + w}{2};
y = z - t;
x = w -y \]
Zeige: $\forall z \in \N \cap [0, 1000]: \pi\left(\pi^{-1}\left(z\right)\right) = z$.
\end{enumerate}
\end{aufg}


\end{document}
