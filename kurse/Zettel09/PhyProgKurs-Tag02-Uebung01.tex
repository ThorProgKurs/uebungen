\documentclass{uebungszettel}

\begin{document} 
\fontfamily{arial}\selectfont
\cohead{Tag 2 - �bung 1}

\begin{aufg}
Berechne die Summe der ersten $n$ ungeraden Zahlen mit einer \texttt{for}-Schleife. Wie ist die Ausgabe f�r $n = 1, \ldots ,15$. Was f�llt dir auf? K�nnte man diese Aufgabe nun also effizienter programmieren?
\end{aufg}

\begin{aufg}
Berechne die Reihe
\[ \sum_{k = 1}^\infty \frac{1}{k^2} \]
Ihr Wert sollte $\frac{\pi^2}{6}$ sein.
\emph{Bemerkung: �berlege dir vielleicht ein geeignetes Abbruchkriterium}
\end{aufg}

\begin{aufg}
Implementiere den Euklidischen Algorithmus zur Berechnung des gr��ten gemeinsamen Teilers zweier Zahlen als eigene Funktion.
\end{aufg}

\begin{aufg}
Implementiere den Cosinus �ber seine Reihendarstellung mit einer for-Schleife. Du kannst die Formel bei wiki nachschlagen, sie gerade selbst entwickeln oder diese hier verwenden:
\[ cos(x) = \sum_{\underset{k \text{ STEP } 2}{k = 0}}^{\infty}{(-1)^{\frac{k}{2}} \cdot \frac{x^k}{k!}} \]
Du solltest dies selbstverst�ndlich als eigene Funktion schreiben.
\end{aufg}


\end{document}
