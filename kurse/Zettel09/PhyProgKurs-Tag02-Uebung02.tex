\documentclass{uebungszettel}

\begin{document} 
\fontfamily{arial}\selectfont
\cohead{Tag 2 - �bung 2}

\begin{aufg}
Implementiere die Signumsfunktion \texttt{sgn(x)}, den Absolutbetrag \texttt{betrag(x)}, \texttt{cos(x)} und die Wurzelfunktion \texttt{wurzel(x)} als Funktionen in einem eigenen Modul.
\end{aufg}

\begin{aufg}
\begin{enumerate}
\item Implementiere f�r $x \in \R$ und $n \in \N$ eine Potenzfunktion $x^n=$ \texttt{power(x, n)} mit der Double-and-Add-Methode: \[
	power(x, n) = \left\{ \begin{array}{ll}
	1 & \text{wenn } n = 0 \\
	x \cdot power\left(x^2, \frac{n-1}{2}\right) & \text{wenn } n \text{ ungerade} \\
	power\left(x^2, \frac{n}{2}\right) & \text{wenn } n \text{ gerade} \\
	\end{array}
	\right.
\]
zuerst mal rekursiv. 
\item Implementiere eine Potenzfunktion \texttt{naiv\_power(x, n)}, indem du eine Schleife von $1$ bis $n$ laufen l�sst und bei jedem Durchlauf eine mit $1$ Initialisierte Variable mit $x$ multipliziert. Berechne $0,9999999999^{2000000000}$ einmal mit \texttt{power(x, n)} von oben und einmal mit \texttt{naiv\_power(x, n)} (es sollte ca. $0,818731$ raus kommen). 
\item* Implementiere die Double-and-Add-Methode iterativ, also ohne rekursiven Aufruf.
\end{enumerate}
\end{aufg}

\begin{aufg}
Implementiere die Riemann'sche Zeta-Funktion f�r $s \in \N$: $$
\zeta(s) := \sum_{k=1}^\infty \frac{1}{k^s} $$
\emph{Tipp:} Vielleicht kannst du die zuletzt implementierte Funktion \texttt{power(x, n)} daf�r verwenden.
\end{aufg}

\end{document}
