\documentclass{uebungszettel}

\begin{document} 
\fontfamily{arial}\selectfont
\cohead{Tag 1 - �bung 2}

\begin{aufg}
Implementiere den Primzahltest von eben.
\end{aufg}

\begin{aufg}
Implementiere den Wurzelalgorithmus von eben.
\end{aufg}

\begin{aufg}
Implementiere den euklidischen Algorithmus von eben.
\end{aufg}

\begin{aufg}
Wettbewerb: Gegeben ist folgender Programmrumpf:
\begin{codelisting}
\begin{lstlisting}[numbers=left,numberstyle=\tiny,frame=tlrb]
#include <stdio.h>
int main(int argc, char **argv) {
	int x = 2;
	/* dein Code hier */
	printf("%i", x);
	return 0;
}
\end{lstlisting}
\end{codelisting}
F�ge an der markierten Stelle C-Code ein, sodass der Wert von $2^{\left(3^3\right)}$ ausgegeben wird. Wer in diesem Raum am wenigsten Zeichen daf�r ben�tigt bekommt morgen eine �berraschung von uns. Erlaubt sind aber nur die Zeichen 
\begin{center}
	x \quad + \quad - \quad * \quad / \quad =
\end{center}
und das Semikolon. Zeilenumbr�che und Leerzeichen k�nnen nat�rlich nach belieben verwendet werden, da sie vom Compiler ignoriert werden.
\end{aufg}



\end{document}
