\documentclass{uebungszettel}
\usepackage{algorithm,algorithmic}
\usepackage[pdftex]{hyperref}

\floatname{algorithm}{Algorithmus}
\newcommand{\SET}{\textbf{set}\ }
\renewcommand{\algorithmicrequire}{\textbf{Input:}}
\renewcommand{\algorithmicensure}{\textbf{Output:}}

\definecolor{linkcolor}{rgb}{0.5,0.08,0.08}
\hypersetup{pdftex=true,colorlinks=true,breaklinks=true,linkcolor=linkcolor,menucolor=linkcolor,citecolor=linkcolor,filecolor=linkcolor,urlcolor=linkcolor,frenchlinks=false}

\begin{document}
\cohead{Tag 8}

\noindent Siehe R�ckseite f�r Algorithmen im Pseudocode.

\begin{aufg} Implementiere Pollards $p-1$ Methode um die folgende Zahl zu faktorisieren. Du kannst dir von der \href{http://www.uni-bonn.de/~rattle/course/now}{Kurshomepage} diese und weitere Zahlen herunterladen, an denen du deinen Algorithmus testen kannst.

\begin{codelisting}
\begin{lstlisting}[numbers=left,numberstyle=\tiny,frame=tlrb]
0x2D42080FCFBC71D2EA9A7BE2CCB954C8749C90CAB5547E2D2C17
  23B5C9BBE302ED0405CCCBD4B7CDAF35F98B7094D420D77BDA9C
  966BC3394783502595C9C86BB0DA603826C159F23D73057E4865
  527B437418A0D83ABC5AE9DE0E30DB7A5FF6F420682E79E1CC01
  45A09C4CDDDF7F73AEBA1D013226835AA7CDEC03D3D4B4181212
  A5EAD2C1C6AC321B7DD93F93BF90B792FED921ACEF4304526218
  171CED29428D7627F20085D95EF8260F84BD2CF4F46DE29BB7C2
  19381F68F61A19CF3EE725C54E9531808912F635FD37EA50A5BD
  9C33215B03219AA78981EEE9B46360B555250FF5A80615A4FD97
  76A887BA17DE38E63005AC42EF8552334E36E909E5427
\end{lstlisting}
\end{codelisting}
\end{aufg}

\begin{aufg} Implementiere Lenstras EC-Faktorisierungsalgorithmus und faktorisiere zun�chst die folgende Zahl. Auch diese ist auf der \href{http://www.uni-bonn.de/~rattle/course/now}{Kurshomepage} zum Download verf�gbar.

\begin{codelisting}
\begin{lstlisting}[numbers=left,numberstyle=\tiny,frame=tlrb]
0x591EE21000D2C0F86EBFCF0BA1D7D7B4BA6D1F3368D68684F719
  A4BC4FB05FD57A3CA48539B9B705A5363EE670F4C8143EB21986
  CFD46110AAC75389C0B6F10F25E9FFCA39CBC3E233F18D2E16D2
  629CF72812DDE952AA975A7433715F71CBD790A0819F6D9BD836
  0554253E59A20E1E546278A052EACBC6A5D18DE6E802ADE566D7
  B6232B837CE6B67A2B8CDC9246556437EDFB18D34AEE3FD57318
  040E42B20F6A56E0613FFB8BD824493203F9BFF5BFC115DCFFC3
  BE946E193C86BD81ECF166C4E365A533428422FF285B469D300A
  775711F345FCDBEF25661F6008C2D46DF281BB6C6391E8A107D7
  11A2D0C2C1425A504E69BFBC3245A45F6EEE0D62D6C1B16677C6
  2F4B6F01F368EA6B9014367F49A726143366EEE80DF6C1E1A5F3
  C652761D9AE3E2D3E079D259BC2D868301EAD8D7B9F34C702449
  5C8064D45D04B79DF2A8F7996DE4E7E672816CAF638AF70B39B6
  1A677C5E9EEA41968DFF50757738EDC6047CAF5CF6A5281213C0
  439B924E32C102D1F939EDCCD1143554386E610F6EC1B
\end{lstlisting}
\end{codelisting}
Du kannst danach auch versuchen, die anderen Zahlen aus Aufgabe 1 zu faktorisieren.
\end{aufg}

\pagebreak[4]

\begin{algorithm}[H]
\caption{Pollards $p-1$ Algorithmus}
\label{alg2}
\algsetup{indent=1.5em}
\begin{algorithmic}[1]
\REQUIRE Zahlen $a,n \in \mathbb{Z}_+$.
\ENSURE Eine Zahl $d > 1$ mit $\exists m: \gcd(a^{m!}-1,n) = d$ 
\FOR{$m = 2,3,4,\ldots$}
	\STATE \SET $a := a^m \bmod n$
	\STATE \SET $d := \gcd(a-1,n)$
	\IF{$d \ne 1$}
		\RETURN d
	\ENDIF
\ENDFOR
\end{algorithmic}
\end{algorithm}

\begin{algorithm}[H]
\caption{Lenstras EC-Faktorisierungsalgorithmus}
\label{alg2}
\algsetup{indent=1.5em}
\begin{algorithmic}[1]
\REQUIRE Zahl $n \in \mathbb{Z}_+$.
\ENSURE Ein Teiler $d$ von $n$
\STATE W�hle zuf�llige $x_0,y_0,a \in \mathbb{Z}_n$
\STATE \SET $b := y_0^2 - x_0^3 - ax_0$
\STATE Sei $E$ die Kurve zu den Parametern $a$ und $b$.
\STATE \SET $p := (x_0,y_0) \in E$
\FOR{$m = 2,3,4,\ldots$}
	\STATE \SET $p := m \cdot p \pmod n$
	\IF{Berechnung schl�gt fehl}
		\STATE Wir haben ein $x$ gefunden, welches kein Inverses modulo $n$ hat.
		\STATE \SET $d := \gcd(x,n)$
		\IF{$d < n$}
			\RETURN $d$
		\ELSIF{$d = n$}
			\STATE Starte Algorithmus neu
		\ENDIF
	\ENDIF
\ENDFOR
\end{algorithmic}
\end{algorithm}


\end{document}
