\documentclass{uebungszettel}

\begin{document}
\cohead{Tag 3}

\begin{aufg} Finde heraus was ein Sudoku ist, danach schreibe ein Modul mit Funktionen, die
\begin{enumerate}
	\item pr�ft, ob ein Sudoku korrekt gel�st ist
	\item ein teilweise ausgef�lltes Sudoku l�st
\end{enumerate}
\end{aufg}

\begin{aufg}
Schreibe ein Programm, dass ein Labyrinth aus einer Textdatei einliest, es k�nnte etwa wie folgt aussehen:
{\tt
\lstset{language=Delphi}
\begin{lstlisting}
XXXXXXXXXXXXXXXX
X X XXXXXXXXXX*X
X$X XX     XXX X
X X XX XXX XXX X
X   XX XXX XXX X
XXX X   XX XXX X
XXX   X        X
XXXXXXXXXXXXXXXX
\end{lstlisting}
}
\emph{Bemerkung: } Wir spezifizieren das Labyrinth hier nicht viel n�her, entscheide dich selbst vorher was f�r ein Format die Datei haben soll und welche Einschr�nkungen du daran stellst: Soll die Gr��e des Labyrinths variabel sein oder fest? Soll die Gr��e in der ersten Zeile der Datei stehen oder nicht? Soll das Labyrinth quadratisch sein oder nicht? Soll es au�en herum immer mit $X$en begrenzt sein oder hast du vielleicht eine andere L�sung?\\
Das Programm soll einen Weg vom Startpunkt (dem Stern, dem Geburtsort) zum Dollar (dem Schatz) finden. Die $X$e sind W�nde und Leerzeichen sind Pfade. Markiere einen Weg mit Punkten und gebe das Labyrinth mit Weg in der Konsole aus.

{\tt
\lstset{language=Delphi}
\begin{lstlisting} 
XXXXXXXXXXXXXXXX
X X XXXXXXXXXX*X
X$X XX     XXX.X
X.X XX XXX XXX.X
X...XX XXX XXX.X
XXX.X...XX XXX.X
XXX...X........X
XXXXXXXXXXXXXXXX
\end{lstlisting}
}

\end{aufg}


\end{document}
