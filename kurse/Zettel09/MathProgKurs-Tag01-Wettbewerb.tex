\documentclass{wettbewerbszettel}

\begin{document}

%Quick Sort
%T�rme von Hanoi
%Mergesort
%k-Mergesort
%Median
%Game of life
%Max Flow Game
%Picross Solver
%Sudoku Solver
%n-Damen Problem
%Tangled Tale (2 di-graph, hamilton path)

\newcommand{\ah}[2]{\ \\* \emph{(#1, #2)}\\}

\begin{aufg} \ah{Exponentialfunktion f�r Matrizen}{rot, 5}
F�r eine quadratische Matrix $A \in \mathbb{R}^{n \times n}$ definiert man
\[ \exp(A) := \sum_{k=0}^\infty \frac{A^k}{k!} \]
wie auch f�r relle Zahlen. Hierbei bezeichnet $A^k$ das $k$--fache Matrixprodukt
von $A$ mit sich selbst, wobei $A^0 = \mathbb{I}$ gilt. Man kann zeigen, dass
dies stets konvergiert. Implementiere diese Funktion in C.
\end{aufg}\newpage

\begin{aufg} \ah{Vielfache von $3$ und $5$}{gr�n, 1} %PE1
Addiere alle nat�rlichen Zahlen, die kleiner als $1.000$ und Vielfaches von $3$
oder $5$ sind.
\end{aufg}\newpage

\begin{aufg} \ah{Gerade Fibonacci-Zahlen}{gr�n, 1} %PE2
Berechne die Summe aller geradzahligen Elemente der Folge der Fibonacci-Zahlen,
die kleiner als $4.000.000$ sind.
\end{aufg}\newpage

\begin{aufg} \ah{Gr��ter Primfaktor}{gelb, 2} %PE3
Schreibe eine Funktion, die den gr��ten Primfaktor einer �bergebenen Zahl zur�ck
gibt, und teste sie an der Zahl $$1073741821$$
\end{aufg}\newpage

\begin{aufg} \ah{Palindrom-Zahlen}{gelb, 3} %PE4
Eine Palindrom-Zahl liest sich (im Dezimalsystem) vorw�rts wie r�ckw�rts. Die
gr��te Palindrom-Zahl, die Produkt zweier zweistelliger Zahlen ist, ist
\[9009 = 91 \cdot 99\]
Finde die gr��te Palindromzahl, die Produkt zweier dreistelliger Zahlen ist.
\end{aufg}\newpage

\begin{aufg} \ah{Teilbarkeit}{gelb, 2} %PE5
Berechne die kleinste Zahl, die durch die Zahlen $1$ bis $20$ teilbar ist.
\end{aufg}\newpage

\begin{aufg} \ah{Summe der Quadrate}{gr�n, 1} %PE6
Schreibe eine Funktion, die zu einer Zahl $n$ die Differenz des Quadrats der Summe und der Summe der Quadrate der Zahlen von $1$ bis $n$ zur�ck gibt. Beispiel ($n=3$): \[
(1+2+3)^2 - (1^2 + 1^2 + 3^2)= 22
\]
Teste sie f�r $n = 100$.
\end{aufg}\newpage

\begin{aufg} \ah{Primzahl}{gr�n, 1} %PE7 
Finde die $10.001$te Primzahl.
\end{aufg}\newpage

\begin{aufg} \ah{Gr��tes Produkt}{rot, 4} %PE8
Finde das gr��te Produkt von f�nf aufeinanderfolgenden Ziffern in der Zahl:
{\tiny
\begin{verbatim}
73167176531330624919225119674426574742355349194934
96983520312774506326239578318016984801869478851843
85861560789112949495459501737958331952853208805511
12540698747158523863050715693290963295227443043557
66896648950445244523161731856403098711121722383113
62229893423380308135336276614282806444486645238749
30358907296290491560440772390713810515859307960866
70172427121883998797908792274921901699720888093776
65727333001053367881220235421809751254540594752243
52584907711670556013604839586446706324415722155397
53697817977846174064955149290862569321978468622482
83972241375657056057490261407972968652414535100474
82166370484403199890008895243450658541227588666881
16427171479924442928230863465674813919123162824586
17866458359124566529476545682848912883142607690042
24219022671055626321111109370544217506941658960408
07198403850962455444362981230987879927244284909188
84580156166097919133875499200524063689912560717606
05886116467109405077541002256983155200055935729725
71636269561882670428252483600823257530420752963450
\end{verbatim}}
(Download von der Kursseite)
\end{aufg}\newpage

\begin{aufg} \ah{Pythagor�isches Tripel}{gelb, 2} %PE9
Finde das eindeutige Tripel $(a, b, c)$ mit
\[ a^2+b^2=c^2 \]
und
\[ a+b+c=1000 \]
\end{aufg}\newpage

\begin{aufg} \ah{Gr��tes Produkt die Zweite}{rot, 5}
Finde die vier benachbarten Zahlen (nach oben, unten, links, rechts und in alle
diagonalen Richtungen) mit dem gr��ten Produkt:
{\tiny
\begin{verbatim}
08 02 22 97 38 15 00 40 00 75 04 05 07 78 52 12 50 77 91 08
49 49 99 40 17 81 18 57 60 87 17 40 98 43 69 48 04 56 62 00
81 49 31 73 55 79 14 29 93 71 40 67 53 88 30 03 49 13 36 65
52 70 95 23 04 60 11 42 69 24 68 56 01 32 56 71 37 02 36 91
22 31 16 71 51 67 63 89 41 92 36 54 22 40 40 28 66 33 13 80
24 47 32 60 99 03 45 02 44 75 33 53 78 36 84 20 35 17 12 50
32 98 81 28 64 23 67 10 26 38 40 67 59 54 70 66 18 38 64 70
67 26 20 68 02 62 12 20 95 63 94 39 63 08 40 91 66 49 94 21
24 55 58 05 66 73 99 26 97 17 78 78 96 83 14 88 34 89 63 72
21 36 23 09 75 00 76 44 20 45 35 14 00 61 33 97 34 31 33 95
78 17 53 28 22 75 31 67 15 94 03 80 04 62 16 14 09 53 56 92
16 39 05 42 96 35 31 47 55 58 88 24 00 17 54 24 36 29 85 57
86 56 00 48 35 71 89 07 05 44 44 37 44 60 21 58 51 54 17 58
19 80 81 68 05 94 47 69 28 73 92 13 86 52 17 77 04 89 55 40
04 52 08 83 97 35 99 16 07 97 57 32 16 26 26 79 33 27 98 66
88 36 68 87 57 62 20 72 03 46 33 67 46 55 12 32 63 93 53 69
04 42 16 73 38 25 39 11 24 94 72 18 08 46 29 32 40 62 76 36
20 69 36 41 72 30 23 88 34 62 99 69 82 67 59 85 74 04 36 16
20 73 35 29 78 31 90 01 74 31 49 71 48 86 81 16 23 57 05 54
01 70 54 71 83 51 54 69 16 92 33 48 61 43 52 01 89 19 67 48
\end{verbatim}}
(Download von der Kursseite)
\end{aufg}\newpage

\begin{aufg} \ah{Burrows Wheeler Transformation}{rot, 7}
Die \emph{BWT} wird in Kompressionsalgorithmen verwendet. Berechne f�r einen
String $s$ der L�nge $n$ die $n \times n$-Matrix $A$ mit $A_{i,j} =
s[(j+i)\bmod n]$ ($0 \le i, j < n$). Sortiere die Zeilen lexikographisch, und gib
dann den String $t$ mit $t[i] = \tilde{A}_{i,n-1}$ zur�ck, sowie den Index $k$ mit
$\tilde{A}_{k,j} = s[j]$ der sortierten Matrix $\tilde{A}$ f�r alle $j$.
\begin{verbatim}
BANANE                 ANANEB
EBANAN                 ANEBAN
NEBANA    sortieren    BANANE (*)
ANEBAN    -------->    EBANAN
NANEBA                 NANEBA
ANANEB                 NEBANA
\end{verbatim}
Ausgabe: ``BNENAA'', $k = 2$
\end{aufg}\newpage

\begin{aufg} \ah{Max in Array}{gr�n, 1}
Schreibe eine Funktion, die das Maximum eines \ic{double}-Arrays zur�ck gibt.
\end{aufg}\newpage

\begin{aufg} \ah{Levenshtein Distanz}{rot, 6}
Berechne die minimale Anzahl von Einf�gen/L�schen/Ersetzen Operationen, die
n�tig sind, um einen gegebenen String in einen anderen zu �berf�hren:

\xymatrix@C=4em{
    *++[F-,]{gips} \ar[r]^-{ersetzen}
&   *++[F-,]{gibs} \ar[r]^-{einf"ugen}
&   *++[F-,]{gibts}
}\vspace{.3em}

F�r zwei Strings $s_1$, $s_2$ der L�ngen $n_1$, $n_2$, berechne folgende
$(n_1+1) \times (n_2+1)$-Matrix $D$:
\vspace{-.5em}
\begin{eqnarray*}
    \forall\ 0 \le i \le n_1:& D_{i,0} = i \\
    \forall\ 0 \le j \le n_2:& D_{0,j} = j
\end{eqnarray*}
\[ \forall\ 1 \le i \le n_1, 1 \le j \le n_2: \]
\vspace{-1em}
\[
    D_{i,j} = min
    \begin{cases}
        0+D_{i-1,j-1} &{\rm{falls}} \ s_1[i] = s_2[j] \\
        1+D_{i-1,j-1} &{\rm{(Ersetzen)}} \\
        1+D_{i  ,j-1} &{\rm{(Einf"ugen)}} \\
        1+D_{i-1,j  } &{\rm{(L"oschen)}}
    \end{cases}
\]
Gib dann $D_{n_1,n_2}$ zur�ck.
\end{aufg}\newpage

\begin{aufg} \ah{Max Subsequence}{gelb, 2/3}
Finde die maximale Summe eines zusammenh�ngenden Teilarrays, d.h. f�r
ein Array $(a_i)_{i=1}^n, a_i \in \mathbb{Z} $, finde
\[ \max \left\{ \sum_{i=l}^{u} a_i \ \middle|\  1 \le l \le u \le n \right\} \]
\end{aufg}\newpage

\end{document}
