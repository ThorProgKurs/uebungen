\documentclass{uebungszettel}

\begin{document}
\cohead{Tag 2}


\begin{aufg}
In dieser Aufgabe geht es um numerische Integration.
\begin{enumerate}
\item Implementiere eine Integrationsfunktion, die das Intervall $[a, b]$ in $n$ gleich gro�e Teile aufteilt, f�r diese jeweils die Trapzsumme berechnet und diese aufsummiert:

\begin{codelisting}
\begin{lstlisting}[numbers=left,numberstyle=\tiny,frame=tlrb]
double integrate(double a, double b, 
  double (*f)(double), unsigned int n); 
\end{lstlisting}
\end{codelisting}

Die Trapezsumme k�nnte etwa wie folgt implementiert werden:

\begin{codelisting}
\begin{lstlisting}[numbers=left,numberstyle=\tiny,frame=tlrb]
#include <math.h>

/* Definiere REALFUNC: Funktionen, welche ein
   double erhalten und ein double liefern */
typedef double (*REALFUNC)(double);

double Trapez( REALFUNC f, double a, double b ) {
  return fabs( (b-a) * (f(a)+f(b))/2.0 ); 
}
\end{lstlisting}
\end{codelisting}

\item Schreibe nun eine Funktion, die nicht die Anzahl der Teilintervalle erh�lt, sondern eine ``Fehlertoleranz'' $e$. Die Funktion die Aufteilung solange verfeinern, bis sich der approximierte Wert f�r das Integral durch eine Verfeinerung nur noch um weniger als $e$ �ndern w�rde. 
\end{enumerate}
\end{aufg}


\begin{aufg} Implementiere eine Funktion die zu einem gegebenen Funktionenpointer $f:\R \rightarrow \R$, einem Dateinamen, einer Schrittweite $s \in \R$, einer Startstelle $x_1$ und einer Endstelle $x_2$ die Wertetabelle der Funktion zwischen $x_1$ und $x_2$ zur Schrittweite $s$ speichert. Dabei sollen $x$ und $f(x)$ durch einen Tabulator getrennt werden und jedes Paar $(x, f(x))$ in einer eigenen Zeile stehen.
\end{aufg}

\begin{aufg}
\begin{enumerate}
	\item L�se das $8$-Damen-Problem: Platziere $8$ Damen auf einem gew�hnlichen Schachbrett so, dass sie sich paarweise nicht bedrohen.
	\item L�se das $n$-Damen Problem (auf einem $n \times n$-Schachbrett (mit einem Programm (auf einem Computer))).
\end{enumerate}
\end{aufg}

\end{document}
