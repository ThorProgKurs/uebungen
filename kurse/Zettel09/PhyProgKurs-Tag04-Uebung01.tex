\documentclass{uebungszettel}

\begin{document} 
\fontfamily{arial}\selectfont
\cohead{Tag 4 - �bung 1}

\begin{aufg}
Denke dir einen sinnvollen Namen aus f�r ein Modul, dass Vektorrechnung auf $\R^n$ implementiert. Vektoren sollen \verb|double|-Arrays mit der L�nge $n$ sein. Implementiere die nachfolgenden Funktionen:
\begin{enumerate}
	\item eine Funktion, die gen�gend Speicher f�r einen Vektor reserviert und einen Pointer darauf zur�ck gibt
\item Vektoraddition und --subtraktion
\item Produkt eines Vektors mit einer skalaren Gr��e
\item Skalarprodukt zweier Vektoren
\item Kreuzprodukt zweier Vektoren (falls existent)
\item eine Funktion, die pr�ft, ob zwei Vektoren orthogonal zueinander stehen
\item eine Funktion, die pr�ft, ob zwei Vektoren parallel zueinander sind
\item eine Funktion, die einen Vektor auf der Konsole aus gibt
\end{enumerate}
Der R�ckgabetyp der Funktionen soll \verb|void| sein und das letzte Argument soll ein Vektor sein in dem das Ergebnis gespeichert wird. Zur Verdeutlichung hier ein Beispiel einer Funktion, die einen Vektor aufnimmt und ihn mit dem Nullvektor initialisiert:
\begin{codelisting}
\begin{lstlisting}[numbers=left,numberstyle=\tiny,frame=tlrb]
void make0(double *a, int n) { 
	int i;
	for(i=0; i<n; i++) a[i] = 0;
}
\end{lstlisting}
\end{codelisting}
\end{aufg}
\begin{aufg}
Implementiere den Bucket-Sort Algorithmus: Unter der Annahme, dass die zu sortierenden Daten aus einem endlichen Wertebereich stammen kann man dies theoretisch sehr schnell machen (man sagt: in Linearzeit). Ohne gro�e Einschr�nkung der Universalit�t des Algorithmus betrachten wir hier nur ein \verb|unsigned short|-Array $A$. Bucketsort besteht nun aus zwei Durchl�ufen:
\begin{enumerate}
\item Sei $n$ das Maximum aus $A$, erstelle dann ein \verb|unsigned|-Array $B$ mit $n+1$ Elementen und sorge daf�r, dass an der $i$-ten Stelle die Anzahl der $i$s steht, die in $A$ vorkommen.
\item Gehe nun $B$ durch und �berschreibe $A$. Schreibe dabei $k$ mal das Element $i$, wenn an der $i$-ten Stelle von $B$ ein $k$ steht.
\end{enumerate}
\end{aufg}


\end{document}
