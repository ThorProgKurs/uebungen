\documentclass{uebungszettel}

\begin{document} 
\fontfamily{arial}\selectfont
\cohead{Tag 5 - �bung 2}

\begin{aufg}
\begin{enumerate}
\item
Implementiere die Addition, Multiplikation, Potenzen und Division komplexer Zahlen. Verwende dazu folgende Header-Datei:
\begin{codelisting}
\begin{lstlisting}[numbers=left,numberstyle=\tiny,frame=tlrb]
struct COMPLEX {
	double real;
	double imag;
};

typedef struct COMPLEX COMPLEX;

COMPLEX cplx_add(COMPLEX a, COMPLEX b);
COMPLEX cplx_mult(COMPLEX a, COMPLEX b);
COMPLEX cplx_pot(COMPLEX a, unsigned long n);
COMPLEX cplx_div(COMPLEX a, COMPLEX b);
\end{lstlisting}
\end{codelisting}
\item Implementiere die Addition, Multiplikation und Division sowie das K�rzen rationaler Zahlen. Schreibe dazu erst die Header-Datei.
\end{enumerate}
\end{aufg}

\pagebreak[4]

\begin{aufg}
Implementiere ``dynamische Arrays''. Also Funktionen, die es leicht erm�glichen mit dynamisch gro�en Arrays zu arbeiten. Verwende eine Header-Datei in folgendem Stil:
\begin{codelisting}
\begin{lstlisting}[numbers=left,numberstyle=\tiny,frame=tlrb]
struct DBLARRAY {
  double *data; /* eigentliches Array */
  int length;   /* erstes nicht verwendetes Element */
  int _size;    /* Menge der allokierten Elemente */
};

typedef struct DBLARRAY DBLARRAY;

/* initialisiert eine Array-Datenstruktur */
DBLARRAY *dblarray_init();

/* gibt eine Array-Datenstruktur wieder frei */
void dblarray_free(DBLARRAY *);

/* setzt den Wert an der Stelle i auf val 
 * falls n�tig wird neuer Speicher allokiert und 
 * alle Elemente bis dorthin mit 0 initialisiert */
int dblarray_set(DBLARRAY *, int i, double val);

/* gib den Wert an der Stelle i zur�ck */
double dblarray_get(DBLARRAY *, int i);

\end{lstlisting}
\end{codelisting}
\end{aufg}

\end{document}
