\documentclass{uebungszettel}

\begin{document} 
\fontfamily{arial}\selectfont
\cohead{Tag 5 - �bung 1}

\begin{aufg}
Gegeben sei eine Datei, in der ausschlie�lich Zahlen stehen. In der ersten Zeile stehe eine nat�rliche Zahl, die angibt wie viele Zahlen noch folgen. 
\begin{enumerate}
\item Schreibe ein Programm, dass diese Datei einliest, die Zahlen sortiert und die Datei mit der sortierten Liste �berschreibt. 
\item Modifiziere dein Programm nun so, dass in der ersten Zeile nicht mehr stehen muss wie viele Zeilen noch folgen.
\end{enumerate}
\end{aufg}

\begin{aufg}
Schreibe ein Programm, dass ein Labyrinth aus einer Datei einliest:
{\tt
\lstset{language=Delphi}
\begin{lstlisting}
XXXXXXXXXXXXXXXX
X X XXXXXXXXXX*X
X$X XX     XXX X
X X XX XXX XXX X
X   XX XXX XXX X
XXX X   XX XXX X
XXX   X        X
XXXXXXXXXXXXXXXX
\end{lstlisting}
}
\emph{Bemerkung: } Wir spezifizieren das Labyrinth hier nicht viel n�her, entscheide dich selbst vorher was f�r ein Format die Datei haben soll und welche Einschr�nkungen du daran stellst: Soll die Gr��e des Labyrinths variabel sein oder fest? Soll die Gr��e in der ersten Zeile der Datei stehen oder nicht? Soll das Labyrinth quadratisch sein oder nicht? Soll es au�en herum immer mit $X$en begrenzt sein oder hast du vielleicht eine andere L�sung?\\
Das Programm soll einen Weg vom Startpunkt (dem Stern, dem Geburtsort) zum Dollar (dem Schatz) finden. Die $X$e sind W�nde und Leerzeichen sind Pfade. Markiere einen Weg mit Punkten und gebe das Labyrinth mit Weg in der Konsole aus.

{\tt
\lstset{language=Delphi}
\begin{lstlisting}
XXXXXXXXXXXXXXXX
X X XXXXXXXXXX*X
X$X XX     XXX.X
X.X XX XXX XXX.X
X...XX XXX XXX.X
XXX.X...XX XXX.X
XXX...X........X
XXXXXXXXXXXXXXXX
\end{lstlisting}
}
\end{aufg}


\begin{aufg} *
Diese Aufgabe l�uft auf die Implementierung des Merge-Sort Algorithmus hinaus.
\begin{enumerate}
\item Implementiere eine Funktion \verb|merge|, die zwei bereits sortierte (eventuell verschieden gro�e) Arrays als Argumente erh�lt, diese zu einem sortieren Array kombiniert und dieses zur�ck liefert. 
\item Die Funktion \verb|mergesort| selbst soll ein Array in zwei (m�glichst gleich gro�e) Teilarrays zerlegen, sich f�r diese Teilarrays selbst aufrufen und danach die dann sortierten Teilarrays mit der \verb|merge|-Funktion kombinieren. Erh�lt die Funktion ein Array mit keinem oder einem Element so bel�sst es dieses Array wie es ist, dann ist es n�mlich bereits sortiert.
\item Besorge dir die Datei \verb|daten.h|, sortiere das darin definierte Array und schreibe es sortiert in eine Datei.
\end{enumerate}

Hier als Tipp ein Vorschlag f�r die Signaturen der beiden Funktionen:
\begin{codelisting}
\begin{lstlisting}[numbers=left,numberstyle=\tiny,frame=tlrb]
int *merge(int *list1, int n, int *list2, int m);
void mergesort(int *list, int n);
\end{lstlisting}
\end{codelisting}

\end{aufg}

\end{document}
