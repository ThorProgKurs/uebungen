\documentclass{uebungszettel}

\begin{document} 
\fontfamily{arial}\selectfont
\cohead{Tag 3 - �bung 1}


\begin{aufg}
Erweitere das "`mymath"'-Modul noch um eine Funktion, die zu den drei Koeffizienten $a, b, c \in \R$ einer quadratischen Gleichung $$
a \cdot x^2 + b \cdot x + c = 0 $$
die L�sungen berechnet und beide L�sungen zur�ck gibt.
\end{aufg}

\begin{aufg}
Schreibe Funktionen \verb|square_to| und \verb|root_to|, die einen \verb|double|-Pointer entgegen nehmen, die dort stehende Variable quadrieren bzw. daraus die Wurzel ziehen und das Ergebnis sowohl zur�ck geben als auch an die gleiche Speicherstelle schreiben.
\end{aufg}

\begin{aufg}
\begin{enumerate}
\item Vielleicht kennt ihr das Cantor'sche Diagonalverfahren, man verwendet es beispielsweise um zu zeigen, dass $\N$ und $\Q$ gleichm�chtig sind. Eine Funktion $\pi: \N^2 \rightarrow \N$, die einem P�rchen nat�rlicher Zahlen ihre Nummer auf dem "`Weg"' durch das Diagonalschema zuordnet ist gegeben durch: \[
	\pi(x, y) = \frac{(x+y) \cdot (x+y+1)}{2} + y \]
Implementiere die Funktion $\pi$ in einem eigenen Modul "`CantorDiag"'.
\item Es l�sst sich zeigen, dass diese Funktion invertierbar ist. Die Werte der Inversen $\pi^{-1}: \N \rightarrow \N^2$ lassen sich mit folgender Methode berechnen: Zu einem $z = \pi(x, y)$ erh�lt man $x$ und $y$ zur�ck mit \[
w = \left\lfloor \frac{\sqrt{8\cdot z + 1}-1}{2} \right\rfloor;
t = \frac{w^2 + w}{2};
y = z - t;
x = w -y \]
Implementiere auch diese Funktion indem du die Zahl $z$ und Pointer auf Variablen $x$ und $y$ �bergibst in das gleiche Modul wie $\pi$. 
\item Zeige: $\forall z \in \N \cap [0, 1000]: \pi\left(\pi^{-1}\left(z\right)\right) = z$
\end{enumerate}
\end{aufg}


\end{document}
