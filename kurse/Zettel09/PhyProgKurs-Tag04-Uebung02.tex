\documentclass{uebungszettel}

\begin{document} 
\fontfamily{arial}\selectfont
\cohead{Tag 4 - �bung 2}

\begin{aufg}
\begin{enumerate}
\item Implementiere eine c-Datei zu folgender Header-Datei: 
\begin{codelisting}
\begin{lstlisting}[numbers=left,numberstyle=\tiny,frame=tlrb]
/* gibt die L�nge eines Strings zur�ck */
int str_len(char *s); 

/* gibt 0 zur�ck, wenn zwei strings gleich 
 * sind und 1 sonst */
int str_cmp(char *s1, char *s2);

/* kopiert s nach d und gibt d zur�ck */
char *str_cpy(char *d, char *s);

/* h�nge s2 and s1 an und gib s1 zur�ck */ 
char *str_cat(char* s1, char* s2)

\end{lstlisting}
\end{codelisting}
\item Implementiere nun noch folgende String-Funktionen, die man in der Praxis \emph{niemals} benutzen w�rde:
\begin{codelisting}
\begin{lstlisting}[numbers=left,numberstyle=\tiny,frame=tlrb]
/* schreibt s r�ckw�rts in s und gib es zur�ck */
char *str_reverse(char *s);

/* gibt 1 zur�ck, wenn ein String ein Palindrom 
 * ist und 0 sonst */
int str_ispalin(char *s);

/* h�nge s2 an s1 an und stelle sicher, dass 
 * in s1 genug Speicher zur verf�gung steht */
char *str_smartcat(char *s1, char *s2)

/* verkleinert den Speicher auf den s zeigt
 * auf die L�nge von s */
char *str_compress(char *s);
\end{lstlisting}
\end{codelisting}
\end{enumerate}
\end{aufg}

\begin{aufg}
Implementiere folgende Funktion (ins gleiche Modul), die zu einem gegebenen String einen l�ngsten Teilstring findet, der ein Palindrom ist. Speichere diesen Teilstring wieder in s und gib s zur�ck.
\begin{codelisting}
\begin{lstlisting}[numbers=left,numberstyle=\tiny,frame=tlrb]
/* Beschreibung, selber machen */
char *str_glsp(char *s);
\end{lstlisting}
\end{codelisting}
Das Problem ist in polynomieller Laufzeit l�sbar und auf die Idee kann man auch kommen. 
\end{aufg}


\end{document}
