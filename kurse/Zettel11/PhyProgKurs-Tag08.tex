\documentclass{uebungszettel}
\newcommand{\utitle}{Tag 8}

\begin{document}

\begin{aufg} Erst mal mit den anderen Zetteln fertig werden. \end{aufg}


\begin{aufg} Lese noch einmal im Skript die Sektion 7.5 und implementiere doppelt verkettete Listen, die beliebige Daten speichern k�nnen (als \verb|void *|).

\begin{codelisting}
\begin{lstlisting}[numbers=left,numberstyle=\tiny,frame=tlrb]
/* Definiere hier angemessene Strukturen f�r einen
   einzelnen Listeneintrag und die Liste selbst. */

/* Leere Liste erstellen */
LIST *list_create();

/* Element hinter E einf�gen, NULL hei�t am Anfang */
LISTNODE *list_insert(LIST *L, LISTNODE *E, void *p);

/* Element am Anfang bzw. Ende einf�gen */
LISTNODE *list_unshift(LIST *L, void *p);
LISTNODE *list_push(LIST *L, void *p);

/* Element am Anfang bzw. Ende entfernen und 
   die Daten zur�ck geben */
void *list_shift(LIST *L);
void *list_pop(LIST *L);

/* eine Element aus der Liste entfernen */
void list_delete(LIST *L, LISTNODE *E);

/* zwei Listen zusammenf�gen */
LIST *list_merge(LIST *L, LIST *M);

/* Liste inklusive allen Elementen frei geben */
void list_free(LIST *L);

\end{lstlisting}
\end{codelisting}

\end{aufg}


\end{document}
