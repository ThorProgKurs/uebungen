\documentclass{uebungszettel}
\newcommand{\utitle}{Tag 10}
\setlength{\parindent}{0px}

\begin{document}

\begin{aufg} Installiere die Gnu Multiprecision Library auf deinem System (auf Linux-Rechnern sollte sie bereits verf�gbar sein) und faktorisiere die Zahl
\verb|272963285971849714829857456457|.
\end{aufg}

\begin{aufg} Schreibe eine Funktion
\begin{codelisting}
\begin{lstlisting}[numbers=left,numberstyle=\tiny,frame=tlrb]
mpf_t mpf_exp(mpf_t x, unsigned long int prec);
\end{lstlisting}
\end{codelisting}
welche $e^x$ mit genau \verb|prec| Nachkommastellen Pr�zision als GMP-Float berechnet.
\end{aufg}

\begin{aufg} Implementiere eine parallelisierte Matrixmultiplikation mit der in der Vorlesung vorgestellten Bibliothek OpenMP.
\end{aufg}

\begin{aufg} q3dm17 vs Nikolai.
\end{aufg}


\end{document}
