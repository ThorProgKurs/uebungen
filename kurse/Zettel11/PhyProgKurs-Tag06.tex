\documentclass{uebungszettel}
\newcommand{\utitle}{Tag 6}

\begin{document}

\begin{aufg}
Werde zun�chst mit Aufgabe 4 von Zettel 5 fertig. Implementiere dann noch die folgenden Stringfunktionen:

\begin{codelisting}
\begin{lstlisting}[numbers=left,numberstyle=\tiny,frame=tlrb]
/* Allokiere gen�gend Speicher f�r eine Kopie von s  
 * und kopiere den Inhalt von s dorthin. Liefere als
 * R�ckgabe die Anfangsadresse dieser Kopie. */
char *str_dup(char *s);

/* Setze s auf den String, der aus genau n Leerzeichen
 * besteht. */
char *str_spaceout(char *s, unsigned int n);
\end{lstlisting}
\end{codelisting}
\end{aufg}

\begin{aufg} Implementiere die folgenden Funktionen:
\begin{codelisting}
\begin{lstlisting}[numbers=left,numberstyle=\tiny,frame=tlrb]
/* pieces ist ein Array (von Strings) der L�nge count.
   H�nge diese Strings, getrennt durch sep, aneinander 
   und liefere das Ergebnis. Beispiel:

   char *p[3] = { "Ene", "Mene", "Miste" };
   char *joined = str_join( p, 3, ", ");
   printf("%s", joined);
   free(joined); 

   gibt aus: Ene, Mene, Miste */
char  *str_join(char **pieces, int count, char *sep);

/* Spalte str bei jedem Auftreten von sep auf und 
   liefere das Array der entstehenden Strings.
   W�re etwa str der String "Ene, Mene, Miste" und
   sep der String "," so sollte das zur�ckgegebene
   Array die Strings

     "Ene"     " Mene"     " Miste"
   
  enthalten (man bemerke die Leerzeichen). In count
  wird die L�nge des Arrays gespeichert, in diesem
  Fall drei. Der String str selbst sollte von der 
  Funktion nat�rlich nicht ver�ndert werden. */
char **str_split(char *str, char *sep, int *count);
\end{lstlisting}
\end{codelisting}
\end{aufg}

\begin{aufg} Implementiere einige Funktionen um mit quadratischen Matrizen umzugehen:
\begin{enumerate}
\item Eine Funktion, die Speicher f�r eine quadratische Matrix allokiert, eine um ihn freizugeben, eine um sie auszugeben und eine um sie zur Einheitsmatrix zu initialisieren (das ist die Matrix mit $1$en auf der Hauptdiagonale und $0$en sonst):
\begin{codelisting}
\begin{lstlisting}[numbers=left,numberstyle=\tiny,frame=tlrb]
double **matrix_alloc(int n);
void     matrix_free(double **A, int n);
void     matrix_print(double **A, int n);
double **matrix_id(double **A, int n);
\end{lstlisting}
\end{codelisting}
\item Eine Funktion um eine Matrix zu transponieren (d.h. an der Hauptdiagonale ``zu spiegeln'')
\item Eine Funktion, die zwei solche Matrizen miteinander multipliziert und eine neue Matrix zur�ck gibt. F�r zwei $n \times n$-Matrizen $A = (a_{ij})$ und $B = (b_{ij})$ ist $A \cdot B = C = (c_{ij})$ �ber folgende Formel definiert: \[
c_{ij} = \sum_{k=1}^n a_{ik} b_{kj}
\]
\end{enumerate}
\end{aufg}




\end{document}
