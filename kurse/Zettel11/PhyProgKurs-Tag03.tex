\documentclass{uebungszettel}
\newcommand{\utitle}{Tag 3}

\begin{document}

Aufgaben mit Sternchen sind Zusatzaufgaben, mache diese Aufgaben erst zum Schluss.

\begin{aufg}
\begin{enumerate}
\item Implementiere f�r $x \in \R$ und $n \in \N$ eine Potenzfunktion $x^n=$ \verb|power(x, n)| mit der folgenden Double-and-Add-Methode: \[
	power(x, n) = \left\{ \begin{array}{ll}
	1 & \text{wenn } n = 0 \\
	x \cdot power\left(x^2, \frac{n-1}{2}\right) & \text{wenn } n \text{ ungerade} \\
	power\left(x^2, \frac{n}{2}\right) & \text{wenn } n \text{ gerade} \\
	\end{array}
	\right.
\]
\item Implementiere eine Potenzfunktion \verb|naiv_power(x, n)|, indem du eine Schleife von $1$ bis $n$ laufen l�sst und bei jedem Durchlauf eine mit $1$ Initialisierte Variable mit $x$ multipliziert. Berechne $0,9999999999^{2000000000}$ einmal mit \verb|power(x, n)| von oben und einmal mit \verb|naiv_power(x, n)| (es sollte ca. $0,818731$ raus kommen). 
\item* Implementiere die Double-and-Add-Methode iterativ, also ohne rekursiven Aufruf.
\item* In den �bungen wird an der Tafel stehen, wie man im Programm die Zeit messen kann. Vergleiche die Laufzeiten der 3 Funktionen.
\end{enumerate}
\end{aufg}

\vspace{2cm}
\begin{center}
Flip me.
\end{center}

\newpage

\begin{aufg}
Diese Aufgabe wird auf eine \verb|power(x, y)|-Funktion f�hren, die f�r beliebige $x \in \R^+$ und $y \in \R$ den Wert von $x^y$ berechnet.
\begin{itemize}
\item Implementiere die Exponentialfunktion \verb|expo(x)|, die $e^x$ mit Hilfe folgender Reihendarstellung berechnet: \[
e^x = \sum_{k=0}^\infty \frac{x^k}{k!} \]
\item Implementiere eine Logarithmus-Funktion \verb|logarithm(x)|, die $\ln(x)$ mit Hilfe folgender Reihendarstellung berechnet: \[
\ln(x) = 2\cdot \sum_{k = 0}^\infty \left(\frac{x-1}{x+1}\right)^{2k + 1} \frac{1}{2k + 1} \]
\item Verwende die Formel \[
x^y = e^{y \cdot \ln(x)} \] um \verb|power(x, y)| zu bestimmen.
\end{itemize}
\end{aufg}

\begin{aufg}
Implementiere:
\begin{enumerate}
  \item $\log_a(x)$ mit beliebigen Argumenten $a \in \R_+$ und $x \in \R_+$
  \item Fakult�t $n!$ f�r $n \in \N$
  \item f�r die $k$-te Wurzel aus $x\in \R_{\ge 0}$ mit $k \in \R_+$
  \item eine Funktion, die zu zwei Seiten eines Dreiecks und ihrem eingeschlossenen Winkel die L�nge der dritten Seite zur�ck gibt
\end{enumerate}
\end{aufg}

\end{document}
