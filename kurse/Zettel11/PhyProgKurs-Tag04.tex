\documentclass{uebungszettel}
\newcommand{\utitle}{Tag 4}

\begin{document}

\begin{aufg}
Implementiere die Signumsfunktion \verb|sgn(x)|, den Absolutbetrag \verb|betrag(x)|, \verb|cos(x)| und die Wurzelfunktion \verb|wurzel(x)| (mit dem Heron-Verfahren) als Funktionen.
\end{aufg}

\begin{aufg} Lege ein Modul \verb|mymath.c| / \verb|mymath.h| an, in dem du die bisher geschriebenen Funktionen auslagerst.
\end{aufg}

Wir brauchen im folgenden eine Potenzfunktion, die zwei Flie�kommazahlen als Argumente akzeptiert. Falls du diese Funktion gestern geschrieben hast, sollte sie jetzt im \verb|mymath|-Modul verf�gbar sein. Andernfalls gibt es die funktion 

\begin{verbatim}
double pow(double x, double y);
\end{verbatim}

in der Systemheader \verb|<math.h>|. Im Skript findest du im Anhang eine vollst�ndige Referenz einiger Systembibliotheken.

\begin{aufg}
Implementiere die Riemann'sche Zeta-Funktion f�r $s>1$: 
\[ \zeta(s) := \sum_{k=1}^\infty \frac{1}{k^s} \]
\emph{Warnung}: Diese Reihe konvergiert nicht f�r $s \le 1$.
\end{aufg}

\begin{aufg}
Implementiere ein Programm, dass ein \verb|int|-Array sortiert. Die naheliegenste M�glichkeit besteht wohl darin, zuerst das kleinste Element an die erste Stelle zu tauschen, dann das kleinste unter den Verbleibenden an die zweite Stelle zu tauschen, usw.
\end{aufg}

\end{document}
