\documentclass{uebungszettel}
\newcommand{\utitle}{Tag 7}

\begin{document}

\begin{aufg}
Gegeben sei eine Datei, in der ausschlie�lich Zahlen stehen. In der ersten Zeile stehe eine nat�rliche Zahl, die angibt wie viele Zahlen noch folgen. 
\begin{enumerate}
  \item Schreibe ein Programm, dass diese Datei einliest, die Zahlen sortiert und die Datei mit der sortierten Liste �berschreibt. 
  \item Modifiziere dein Programm nun so, dass in der ersten Zeile nicht mehr stehen muss, wie viele Zeilen noch folgen.
\end{enumerate}
\end{aufg}

\begin{aufg}
Erstelle ein Programm \verb|uniq|, dass zwei Dateinamen als Argumente erh�lt, die erste Datei wortweise\footnote{Worte sind alles, was durch Whitepsace (Leerzeichen, Zeilenumbr�che und Tabulatoren) getrennt ist.} einliest und ohne doppelte Eintr�ge in die zweite Datei schreibt. Es soll auch m�glich sein, dass Ein-- und Ausgabedatei die gleiche sind.
\end{aufg}

\begin{aufg} 
Werde mit den Aufgaben von den vorherigen Zetteln fertig.
\end{aufg}

\begin{aufg} Lies im Skript noch den Rest vom Kapitel �ber Dateieingabe und -ausgabe, einige n�tzliche Funktionen wirst du dort noch finden. Schreibe dann ein Programm, dass ein Labyrinth aus einer Datei einliest:
{\tt
\lstset{language=Delphi}
\begin{lstlisting}
XXXXXXXXXXXXXXXX
X X XXXXXXXXXX*X
X$X XX     XXX X
X X XX XXX XXX X
X   XX XXX XXX X
XXX X   XX XXX X
XXX   X        X
XXXXXXXXXXXXXXXX
\end{lstlisting}
}
\emph{Bemerkung: } Wir spezifizieren das Labyrinth hier nicht viel n�her, entscheide dich selbst vorher f�r ein Format. Soll die Gr��e des Labyrinths variabel sein oder fest? Soll die Gr��e in der ersten Zeile der Datei stehen oder nicht? Soll das Labyrinth quadratisch sein oder nicht? Soll es au�en herum immer mit $X$en begrenzt sein oder hast du vielleicht eine andere L�sung?

\vfill
\begin{center}
Flip me.
\end{center}
\vfill

\newpage

\vspace{1.5ex} Das Programm soll einen Weg vom Startpunkt (dem Stern) zum Schatz (dem Dollarzeichen) finden. Die $X$e sind W�nde und Leerzeichen sind Pfade. Markiere einen Weg mit Punkten und gebe das Labyrinth mit Weg in der Konsole aus.

{\tt
\lstset{language=Delphi}
\begin{lstlisting}
XXXXXXXXXXXXXXXX
X X XXXXXXXXXX*X
X$X XX     XXX.X
X.X XX XXX XXX.X
X...XX XXX XXX.X
XXX.X...XX XXX.X
XXX...X........X
XXXXXXXXXXXXXXXX
\end{lstlisting}
}

\end{aufg}



\end{document}
