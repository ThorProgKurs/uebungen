\documentclass{uebungszettel}
\newcommand{\utitle}{Tag 5}

\begin{document}

\begin{aufg}
Schreibe ein Modul \verb|arrayhelpers|, das einige n�tzliche Funktion zum \verb|int|-Array-Handling enth�lt:
\begin{enumerate}
\item Array zeilenweise oder mit Kommata getrennt ausgeben
\item Array sortieren
\item Alle Felder eines Arrays mit einem Wert initialisieren
\item Array um $1$ rotieren (d.h. das hinterste Element an erste Stelle schreiben und alle anderen Elemente um eins nach hinten schieben)
\item Array um $k$ rotieren
\item Array umdrehen
\item Ein Array in einem anderen suchen und die Position zur�ck geben. Sollte das Array nicht im anderen enthalten sein, so soll der R�ckgabewert $-1$ sein.

\emph{Beispiel:} 
\begin{codelisting}
\begin{lstlisting}[numbers=left,numberstyle=\tiny,frame=tlrb]
int A[10] = {1, 2, 3, 4, 5, 6, 7, 8, 9, 10};
int B[3]  = {4, 5, 6};
int C[2]  = {5, 7};
int D[2]  = {9, 10};
\end{lstlisting}
\end{codelisting}
Hier gilt: \verb|B| ist an $3$-ter Stelle in \verb|A| enthalten und \verb|D| an $8$-ter. Das 

Array \verb|C| ist garnicht in \verb|A| enthalten, darum wird der R�ckgabewert $-1$ sein.

\end{enumerate}
\end{aufg}


\begin{aufg} Wir sagen, ein Array der L�nge $n$ ist eine \emph{Permutation}, wenn jede Zahl zwischen $1$ und $n$ an irgendeiner Stelle im Array vorkommt.
\begin{enumerate}
\item Schreibe eine Funktion, die pr�ft, ob ein Array eine Permutation ist.
\item Schreibe eine Funktion, die ein Array \verb|a| und eine Permutation \verb|p| entgegennimmt und \verb|a| so umsortiert, dass das \verb|i|-te Element an die Stelle \verb|p[i]| geschrieben wird.
\end{enumerate}
\end{aufg}

\begin{aufg}
Schreibe eine Funktion zur L�sung einer quadratischen Gleichung $aX^2 + cX + b = 0$, welche beide L�sungen berechnet und mit Hilfe von Pointern zur�ckgibt.
\end{aufg}

\begin{aufg}
\begin{enumerate}
\item Implementiere eine c-Datei zu folgender Header-Datei: 
\begin{codelisting}
\begin{lstlisting}[numbers=left,numberstyle=\tiny,frame=tlrb]
/* gibt die L�nge eines Strings zur�ck */
int str_len(char *s); 

/* gibt 0 zur�ck, wenn zwei strings gleich 
 * sind und 1 sonst */
int str_cmp(char *s1, char *s2);

/* kopiert s nach d und gibt d zur�ck */
char *str_cpy(char *d, char *s);

/* h�nge s2 and s1 an und gib s1 zur�ck */ 
char *str_cat(char* s1, char* s2)

\end{lstlisting}
\end{codelisting}
\item Implementiere nun noch folgende String-Funktionen, die man in der Praxis \emph{niemals} benutzen w�rde:
\begin{codelisting}
\begin{lstlisting}[numbers=left,numberstyle=\tiny,frame=tlrb]
/* schreibt s r�ckw�rts in s und gib es zur�ck */
char *str_reverse(char *s);

/* gibt 1 zur�ck, wenn ein String ein Palindrom 
 * ist und 0 sonst */
int str_ispalin(char *s);

\end{lstlisting}
\end{codelisting}
\end{enumerate}
\end{aufg}





\end{document}
