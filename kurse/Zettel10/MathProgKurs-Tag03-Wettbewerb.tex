\documentclass{uebungszettel}
\newcommand{\utitle}{Tag 3}

\begin{document}

\newcommand{\ah}[2]{\ \\* \emph{(#1, #2)}\\}

\begin{aufg} Werde zun�chst mit den Aufgaben von Tag 2 fertig, mit oder ohne Hilfe der Musterl�sungen. Im zweiten Fall geht es insbesondere darum, die Musterl�sungen zu verstehen.
\end{aufg}

\begin{aufg}
Berechne die Summe der ersten $n$ ungeraden Zahlen mit einer \texttt{for}-Schleife. Wie ist die Ausgabe f�r $n = 1, \ldots ,15$. Was f�llt dir auf? K�nnte man diese Aufgabe nun also effizienter programmieren?
\end{aufg}

\begin{aufg}
Implementiere den Cosinus �ber seine Reihendarstellung mit einer for-Schleife. Du kannst die Formel bei Wikipedia nachschlagen, sie selbst entwickeln oder diese hier verwenden:
\[ \cos(x) = \sum_{k = 0}^{\infty}{(-1)^k \cdot \frac{x^{2k}}{(2k)!}} \]
\end{aufg}

\begin{aufg}
Schreibe ein Programm, um den Wert der Reihe
\[ \sum_{k = 1}^\infty \frac{1}{k^2} \]
zu berechnen. Er sollte $\frac{\pi^2}{6}$ sein. Wichtig ist, sich ein geeignetes Abbruchkriterium zu �berlegen.
\end{aufg}

\end{document}
