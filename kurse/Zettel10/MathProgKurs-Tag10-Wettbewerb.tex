\documentclass{uebungszettel}
\newcommand{\utitle}{Tag 10}
\begin{document}

\begin{aufg} Implementiere eine Funktion die zu einem gegebenen Funktionenpointer $f:\R \rightarrow \R$, einen Dateinamen, einer Schrittweite $s \in \R$, einer Startstelle $x_1$ und einer Endstelle $x_2$ die Wertetabelle der Funktion zwischen $x_1$ und $x_2$ zur Schrittweite $s$ speichert. Dabei sollen $x$ und $f(x)$ durch einen Tabulator getrennt werden und jedes Paar $(x, f(x))$ in einer eigenen Zeile stehen.
\end{aufg}

\begin{aufg}
In dieser Aufgabe geht es um numerische Integration.
\begin{enumerate}
\item Implementiere eine Integrationsfunktion, die das Intervall $[a, b]$ in $n$ gleich gro�e Teile aufteilt, f�r diese jeweils die Trapzsumme (aus der Vorlesung) berechnet und diese aufsummiert:

\begin{codelisting}
\begin{lstlisting}[numbers=left,numberstyle=\tiny,frame=tlrb]
double integrate(
  double a, 
  double b, 
  double (*f)(double), 
  unsigned int n
); 
\end{lstlisting}
\end{codelisting}

\item[*b)] Schreibe nun eine Funktion, die nicht die Anzahl der Teilintervalle erh�lt, sondern eine ``Fehlertoleranz'' $e$. Die Funktion die Aufteilung solange verfeinern, bis sich der approximierte Wert f�r das Integral durch eine Verfeinerung nur noch um weniger als $e$ �ndern w�rde. 
\end{enumerate}
\end{aufg}

\begin{aufg} Schreibe eine Funktion, die mithilfe von \verb|qsort| ein Array von Strings lexikographisch (wie im Telefonbuch) sortiert. Das hei�t, dass erst nach der ersten Stelle sortiert wird, dann nach der Zweiten usw. In der \verb|<string.h>| liegt eine Vergleichsfunktion f�r diese Situation vor: \verb|strcmp|. Als Beispiel hier eine lexikographisch sortierte Liste:
\begin{codelisting}
\begin{lstlisting}[numbers=left,numberstyle=\tiny,frame=tlrb]
1
10
133
2
2344
Hallo
Thor
Tor
\end{lstlisting}
\end{codelisting}

\end{aufg}


\begin{aufg}
Brainfuck ist eine sogenannte esoterische Programmiersprache, das sind Sprachen, die meist zu wissenschaftlichen oder theoretischen Zwecken, oder einfach zum Spa� entwickelt wurden. Sbesteht nur aus $8$ Befehlen: $>$ $<$ $+$ $-$ $,$ $.$ $[$ $]$ alle anderen Zeichen werden als Kommentar interpretiert. Diese Befehle werden, wie bei C auch, nacheinander ausgef�hrt. Sie operieren auf einem (potentiell unendlich langen) Band (welches aus Zellen besteht in denen jeweils ein \verb|char| steht) indem sie einen Lese-/Schreibkopf �ber das Band bewegen und Zeichen lesen / schreiben lassen. Das Band ist �berall mit \verb|'\0'| vorinitialisiert und der Lese-/Schreibkopf startet an ``Position $0$'' des Bandes. Die Befehle haben folgende Bedeutung:

\begin{tabular}{|c|p{10cm}|} \hline
$>$ bzw. $<$ & schiebt den Lese-/Schreibkopf eins nach rechts bzw. links \\\hline
$+$ bzw. $-$ & in- bzw. dekrementiert den Bandwert unter dem Lese-/Schreibkopf um $1$ \\\hline
$.$ & gibt den Wert unter dem Lese-/Schreibkopf aus \\\hline
$,$ & liest ein Zeichen vom Benutzer ein und schreibt es unter den Lese-/Schreibkopf \\\hline
$[$ & springt zum zugeh�rigen $]$-Befehl, wenn der Wert unter dem Lese-/Schreibkopf $0$ ist, sonst soll nichts passieren\\\hline
$]$ & springt zum zugeh�rigen $[$-Befehl, wenn der Wert unter dem Lese-/Schreibkopf verschieden von $0$ ist\\\hline
\end{tabular}

So sieht ein ``Hallo-Welt''-Programm in Brainfuck aus:

\begin{codelisting}
\begin{lstlisting}[numbers=left,numberstyle=\tiny,frame=tlrb,mathescape=true]
++++++++++
[
   >+++++++>++++++++++>+++>+<<<<-
]
>++.
>+.
+++++++..
+++.>++.
<<+++++++++++++++.
>.+++.
- - - - - - . - - - - - - - - .
>+.>.
\end{lstlisting}
\end{codelisting}
Deine Aufgabe ist es nun, ein Programm zu schreiben, welches Brainfuck-Programme einlesen und ausf�hren kann. 
\end{aufg}
\end{document}
