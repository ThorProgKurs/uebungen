\documentclass{uebungszettel}
\newcommand{\utitle}{Tag 5}

\begin{document}

\begin{aufg} Lege ein Modul \verb|mymath.c| / \verb|mymath.h| an, in dem du die bisher geschriebenen Funktionen auslagerst.
\end{aufg}

Wir brauchen im folgenden eine Potenzfunktion, die zwei Flie�kommazahlen als Argumente akzeptiert. Falls du diese Funktion gestern geschrieben hast, sollte sie jetzt im \verb|mymath|-Modul verf�gbar sein. Andernfalls gibt es die funktion 

\begin{verbatim}
double pow(double x, double y);
\end{verbatim}

in der Systemheader \verb|<math.h>|. Im Skript findest du im Anhang eine vollst�ndige Referenz einiger Systembibliotheken.

\begin{aufg}
Implementiere die Riemann'sche Zeta-Funktion f�r $s>1$: 
\[ \zeta(s) := \sum_{k=1}^\infty \frac{1}{k^s} \]
\emph{Warnung}: Diese Reihe konvergiert nicht f�r $s \ge 1$.
\end{aufg}

\begin{aufg}
Erweitere das \verb|mymath|-Modul noch um eine Funktion, die zu drei beliebigen Koeffizienten $a, b, c \in \R$ einer quadratischen Gleichung 
\[ a \cdot x^2 + b \cdot x + c = 0 \]
die L�sungen berechnet und die gr��ere L�sung zur�ck gibt.
\end{aufg}

\pagebreak

\begin{aufg}
\begin{enumerate}
\item Implementiere eine Funktion \verb|presentation(n, r)|, die zu einer nat�rlichen Zahl (im Zehnersystem) $n$ und einer Basis $0 < r < 10$ mit nachfolgendem Algorithmus die Darstellung von $n$ in der Basis $r$ ausgibt: Dividiere durch r, merke dir den Rest, den die Zahl bei der Division l�sst. Verfahre mit dem Ergebniss der Division (abgerundet) so weiter. Die Reste geben die Zahl in der neuen Basis. Hier ein Beispiel, wie der Algorithmus f�r die 99 in der Basis 8 aussehen soll: 
\begin{eqnarray}
99 & = & 12 \cdot 8 + 3 \\
12 & = & 1 \cdot 8 + 4 \\
1 & = & 0 \cdot 8 +1
\end{eqnarray}
Die Darstellung von 99 zur Basis 8 ist also 341 (oder "`richtig"' 143). Lagere die Funktion in ein eigenes Modul aus!
\item Was macht dieser Programmschnipsel? Warum? \\

\begin{codelisting}
\begin{lstlisting}[numbers=left,numberstyle=\tiny,frame=tlrb]
for(; n != 0; n /= r) {
	printf("%i\n",n%r);
}
\end{lstlisting}
\end{codelisting}

\item{*} Implementiere \verb|presentation2(n, r)| rekursiv s.d. die Ziffern in richtiger (umgekehrter) Reihenfolge ausgegeben werden. Implementiere diese Funktion im gleichen Modul wie \verb|presentation(n, r)|!
\end{enumerate}
\end{aufg}


\end{document}
