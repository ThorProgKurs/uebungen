\documentclass{uebungszettel}

\newcommand{\utitle}{Tag 1}
\begin{document}
\newcommand{\ah}[2]{\ \\* \emph{(#1, #2)}\\}

\begin{aufg}
Installiere einen Compiler auf deinem Computer und kompiliere ein Hallo-Welt-Programm. Informationen dazu findest du im Skript auf der Webseite des Kurse:
\begin{center}
	\verb|http://www.ah-effect.net/ |
\end{center}
Du findest dort auch einen Download-Link f�r Cygwin.
\end{aufg}

\begin{aufg}
Was machen folgende Algorithmen (kein C-Code)?
\begin{enumerate}
	\item 
		\emph{Eingabe: } $c \in \mathbb{N}$ \\
		\emph{Ausgabe: } \verb*|Ja| oder \verb|Nein|
		\begin{enumerate}
  		\item Sei $n = 2$
  		\item solange $n \leq \sqrt{c}$ 
  		\item \hspace{1em} falls $c$ durch $n$ teilbar, gebe \verb|Nein| aus und terminiere
  		\item \hspace{1em} vergr��ere $n$ um $1$
  		\item gebe \verb|Ja| aus und terminiere
		\end{enumerate}
	\item
		\emph{Eingabe: } $a \in \mathbb{R}^+_0$ \\
		\emph{Ausgabe: } $x \in \mathbb{R}$
		\begin{enumerate}
  		\item Sei $x = 2$ und $y = 1$
  		\item solange $\left|x - y\right| \geq 10^{-10}$
  		\item \hspace{1em} Setze $x = y$
  		\item \hspace{1em} Setze $y = \frac{1}{2}\left(x + \frac{a}{x}\right)$
  		\item gebe $x$ aus und terminiere
		\end{enumerate}
	\item 
		\emph{Eingabe: } $a, b \in \mathbb{N}$ \\
		\emph{Ausgabe: } $k \in \mathbb{N}$
		\begin{enumerate}
  		\item wenn $a = 0$ dann gebe $b$ aus und terminiere
  		\item sonst 
  		\item \hspace{1em} solange $b \neq 0$
  		\item \hspace{1em} \hspace{1em} wenn $a > b$ setze $a = a - b$
  		\item \hspace{1em} \hspace{1em} sonst setze $b = b - a$
  		\item gebe $a$ aus und terminiere
		\end{enumerate}
\end{enumerate}
\end{aufg}

\end{document}
