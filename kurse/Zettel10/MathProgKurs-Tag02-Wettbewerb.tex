\documentclass{uebungszettel}

\newcommand{\utitle}{Tag 2}
\begin{document}
\newcommand{\ah}[2]{\ \\* \emph{(#1, #2)}\\}
\begin{aufg}

Schreibe ein Programm, dass den Wert der folgenden Funktion ausgibt (f�r eine fest in den Quellcode geschriebene \verb|int|-Variable):
\[
	f(n) = \left\{ \begin{array}{ll}
	\frac{n}{2} & \text{wenn } n \text{ gerade} \\
	\frac{n+1}{2} & \text{wenn } n \text{ ungerade} \\
	\end{array}
	\right.
\]
Und das geht nat�rlich nur mit Wissen aus der Vorlesung.
\end{aufg}

Lies dir als n�chstes im Skript durch, wie Schleifen funktionieren (Abschnitt 2.8) und implementiere die Algorithmen vom ersten Tag in C:

\begin{aufg} .
\begin{enumerate}
\item Implementiere den Primzahltest von gestern.
\item Schreibe ein Programm, dass jeweils die n�chste Primzahl nach $20000$, $30000$ und $40000$ findet.
\end{enumerate}
\end{aufg}

\begin{aufg}
F�r $a \in \mathbb{R}^+$ konvergiert die Folge $(a_n)$ mit $a_0 = a$ und
\[ a_{n+1} = \frac{1}{2}\left(a_n + \frac{a}{a_n}\right) \]
gegen $\sqrt{a}$. Implementiere damit einen Wurzellalgorithmus.
\end{aufg}

\begin{aufg}
Implementiere den Algorithmus von gestern aus Teil c), welcher den gr��ten gemeinsamen Teiler zweier Zahlen berechnet.
\end{aufg}

\end{document}
