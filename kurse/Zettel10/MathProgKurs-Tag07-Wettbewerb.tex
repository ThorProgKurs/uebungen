\documentclass{uebungszettel}
\newcommand{\utitle}{Tag 7}

\begin{document}

\begin{aufg}
Denke dir einen sinnvollen Namen aus f�r ein Modul, dass Vektorrechnung auf $\R^n$ implementiert. Vektoren sollen \verb|double|-Arrays mit der L�nge $n$ sein. Implementiere die nachfolgenden Funktionen:
\begin{enumerate}
	\item eine Funktion, die gen�gend Speicher f�r einen Vektor reserviert und einen Pointer darauf zur�ck gibt
	\item Vektoraddition
	\item Vektorsubtraktion
	\item Produkt eines Vektors mit einer skalaren Gr��e
	\item Skalarprodukt zweier Vektoren
	\item eine Funktion, die einen Vektor auf der Konsole aus gibt
\end{enumerate}
Zur Verdeutlichung hier ein Beispiel einer Funktion, die einen Vektor aufnimmt und ihn mit dem Nullvektor initialisiert:
\begin{codelisting}
\begin{lstlisting}[numbers=left,numberstyle=\tiny,frame=tlrb]
void vector_make0(double *v, int n) { 
	int i;
	for(i=0; i<n; i++) v[i] = 0;
}
\end{lstlisting}
\end{codelisting}
\end{aufg}

\begin{aufg}
Diese Aufgabe l�uft auf die Implementierung des Merge-Sort Algorithmus hinaus.
\begin{enumerate}
\item Implementiere eine Funktion \verb|merge|, die zwei bereits sortierte (eventuell verschieden gro�e) Arrays als Argumente erh�lt, diese zu einem sortieren Array kombiniert und dieses zur�ck liefert. 
\item Die Funktion \verb|mergesort| selbst soll ein Array in zwei (m�glichst gleich gro�e) Teilarrays zerlegen, diese Teilarrays rekursiv durch einen weiteren Aufruf von \verb|mergesort| sortieren und die so sortierten Teilarrays mit der \verb|merge|-Funktion kombinieren. 

Wenn die Funktion ein Array der L�nge $1$ oder $0$ �bergeben bekommt, so tut die Funktion nichts, da das Array bereits sortiert ist: Dies beendet die Rekursion.
\end{enumerate}

Hier als Tipp ein Vorschlag f�r die Signaturen der beiden Funktionen:
\begin{codelisting}
\begin{lstlisting}[numbers=left,numberstyle=\tiny,frame=tlrb]
int *merge(int *list1, int n, int *list2, int m);
void mergesort(int *list, int n);
\end{lstlisting}
\end{codelisting}
\end{aufg}

\begin{aufg} Implementiere einige Funktionen um mit quadratischen Matrizen umzugehen:
\begin{enumerate}
\item Eine Funktion, die Speicher f�r eine quadratische Matrix allokiert, eine um ihn freizugeben, eine um sie auszugeben und eine um sie zur Einheitsmatrix zu initialisieren (das ist die Matrix mit $1$en auf der Hauptdiagonale und $0$en sonst):
\begin{codelisting}
\begin{lstlisting}[numbers=left,numberstyle=\tiny,frame=tlrb]
double **matrix_alloc(int n);
void     matrix_free(double **A, int n);
void     matrix_print(double **A, int n);
double **matrix_id(double **A, int n);
\end{lstlisting}
\end{codelisting}
\item Eine Funktion um eine Matrix zu transponieren (d.h. an der Hauptdiagonale ``zu spiegeln'')
\item Eine Funktion, die zwei solche Matrizen miteinander multipliziert und eine neue Matrix zur�ck gibt. F�r zwei $n \times n$-Matrizen $A = (a_{ij})$ und $B = (b_{ij})$ ist $A \cdot B = C = (c_{ij})$ �ber folgende Formel definiert: \[
c_{ij} = \sum_{k=1}^n a_{ik} b_{kj}
\]
\end{enumerate}
\end{aufg}


\end{document}
